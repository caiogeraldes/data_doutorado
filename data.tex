\setlrmarginsandblock{1cm}{1cm}{*}
\setulmarginsandblock{1cm}{1cm}{*}
\checkandfixthelayout%
{\tiny
\begin{longtable}{p{0.2\linewidth}p{0.2\linewidth}lp{0.5\linewidth}}
\toprule
AUTHOR & WORK & LOCATION & TEXT\\
\midrule
Aeschines (0026) & Against Timarchus (001) & 30.3 & τὸν γὰρ τὴν ἰδίαν οἰκίαν κακῶς οἰκήσαντα, καὶ τὰ κοινὰ τῆς πόλεως παραπλησίως ἡγήσατο διαθήσειν, καὶ οὐκ ἐδόκει οἷόν τ’ εἶναι τῷ νομοθέτῃ τὸν αὐτὸν ἄνθρωπον ἰδίᾳ μὲν εἶναι πονηρόν, δημοσίᾳ δὲ χρηστόν, οὐδ’ ᾤετο δεῖν τὸν ῥήτορα ἥκειν ἐπὶ τὸ βῆμα τῶν λόγων ἐπιμεληθέντα πρότερον, ἀλλ’ οὐ τοῦ βίου.\\
Aeschines (0026) & Against Timarchus (001) & 51.5 & ὁ γὰρ πρὸς ἕνα τοῦτο πράττων, ἐπὶ μισθῷ δὲ τὴν πρᾶξιν ποιούμενος, αὐτῷ μοι δοκεῖ τούτῳ ἔνοχος εἶναι.\\
Aeschines (0026) & Against Timarchus (001) & 52.9 & ὁ γὰρ εἰκῇ τοῦτο καὶ πρὸς πολλοὺς πράττων καὶ μισθοῦ, αὐτῷ μοι δοκεῖ τούτῳ ἔνοχος εἶναι.\\
Aeschines (0026) & Against Ctesiphon (003) & t.1 & ἐβουλόμην μὲν οὖν, ὦ ἄνδρες Ἀθηναῖοι, καὶ τὴν βουλὴν τοὺς πεντακοσίους καὶ τὰς ἐκκλησίας ὑπὸ τῶν ἐφεστηκότων ὀρθῶς διοικεῖσθαι, καὶ τοὺς νόμους οὓς ἐνομοθέτησεν ὁ Σόλων περὶ τῆς τῶν ῥητόρων εὐκοσμίας ἰσχύειν, ἵνα ἐξῆν πρῶτον μὲν τῷ πρεσβυτάτῳ τῶν πολιτῶν, ὥσπερ οἱ νόμοι προστάττουσι, σωφρόνως ἐπὶ τὸ βῆμα παρελθόντι ἄνευ θορύβου καὶ ταραχῆς ἐξ ἐμπειρίας τὰ βέλτιστα τῇ πόλει συμβουλεύειν, δεύτερον δ’ ἤδη καὶ τῶν ἄλλων πολιτῶν τὸν βουλόμενον καθ’ ἡλικίαν χωρὶς καὶ ἐν μέρει περὶ ἑκάστου γνώμην ἀποφαίνεσθαι·\\
Aeschines (0026) & Against Timarchus (001) & 150.1 & ὡς τοίνυν ἐξῆν αὐτῷ σωθῆναι μὴ τιμωρησαμένῳ τὸν τοῦ Πατρόκλου θάνατον, ἀνάγνωθι ἃ λέγει ἡ Θέτις.\\
\addlinespace
Aeschines (0026) & The Speech on the Embassy (002) & 4.1 & ἐμοὶ δέ, ἄνδρες Ἀθηναῖοι, συμβέβηκε τῆς Δημοσθένους ἀκούοντι κατηγορίας μήτε δεῖσαι πώποθ’ οὕτως ὡς ἐν τῇδε τῇ ἡμέρᾳ, μήτ’ ἀγανακτῆσαι μᾶλλον ἢ νῦν, μήτ’ εἰς ὑπερβολὴν ὁμοίως ἡσθῆναι.\\
Aeschines (0026) & The Speech on the Embassy (002) & 28.1 & ἀφικομένου δ’ εἰς τοὺς τόπους Ἰφικράτους μετ’ ὀλίγων τὸ πρῶτον νεῶν, ἐπὶ κατασκοπῇ μᾶλλον τῶν πραγμάτων ἢ πολιορκία τῆς πόλεως, ἐνταῦθα, ἔφην ἐγώ, μετεπέμψατο αὐτὸν Εὐρυδίκη ἡ μήτηρ ἡ σή, καὶ ὥς γε δὴ λέγουσιν οἱ παρόντες πάντες, Περδίκκαν μὲν τὸν ἀδελφὸν τὸν σὸν καταστήσασα εἰς τὰς χεῖρας τὰς Ἰφικράτους, σὲ δὲ εἰς τὰ γόνατα τὰ ἐκείνου θεῖσα παιδίον ὄντα, εἶπεν ὅτι Ἀμύντας ὁ πατὴρ τῶν παιδίων τούτων, ὅτ’ ἔζη, υἱὸν ἐποιήσατό σε, τῇ δὲ Ἀθηναίων πόλει οἰκείως ἐχρήσατο, ὥστε συμβαίνει σοι καὶ ἰδίᾳ τῶν παίδων τούτων γεγενῆσθαι ἀδελφῷ, καὶ δημοσίᾳ φίλῳ ἡμῖν εἶναι.\\
Aeschines (0026) & The Speech on the Embassy (002) & 147.5 & συμβέβηκε δὲ αὐτῷ νέῳ μὲν ὄντι, πρὶν τὴν οὐσίαν ἀπολέσαι διὰ τὸν πόλεμον, ἀθλεῖν τῷ σώματι, ἐκπεσόντι δὲ ὑπὸ τῶν τριάκοντα στρατεύεσθαι μὲν ἐν τῇ Ἀσίᾳ, ἀριστεύειν δ’ ἐν τοῖς κινδύνοις, εἶναι δ’ ἐκ φρατρίας τὸ γένος ἣ τῶν αὐτῶν βωμῶν Ἐτεοβουτάδαις μετέχει, ὅθεν ἡ τῆς Ἀθηνᾶς τῆς Πολιάδος ἐστὶν ἱέρεια, συγκατάγειν δὲ τὸν δῆμον, ὥσπερ καὶ ὀλίγῳ πρότερον εἶπον.\\
Aeschines (0026) & Against Ctesiphon (003) & 91.6 & οὐδὲν γὰρ ἦν τὸ μέσον, εἰ μνησθεὶς τῶν προτέρων ἀδικημάτων ὁ δῆμος μὴ προσδέξαιτο τὴν συμμαχίαν, ἀλλ’ ὑπῆρχεν αὐτῷ ἢ φεύγειν ἐκ Χαλκίδος, ἢ τεθνάναι ἐγκαταληφθέντι·\\
Aeschines (0026) & Against Timarchus (001) & 72.1 & τίς οὖν οὕτω ταλαίπωρός ἐστιν ἄνθρωπος, ὅστις ἂν ἐθελήσειε σαφῶς τοιαύτην μαρτυρίαν μαρτυρῆσαι, ἐξ ἧς ὑπάρχει αὐτῷ, ἐὰν τἀληθῆ μαρτυρήσῃ, ἐπιδεικνύναι ἔνοχον ὄντα ἑαυτὸν τοῖς ἐσχάτοις ἐπιτιμίοις;\\
\addlinespace
Aeschines (0026) & Against Ctesiphon (003) & 158.3 & ὑμεῖς δέ, ὦ ἄνδρες Ἀθηναῖοι, οὐκ αἰσχύνεσθε, εἰ ἐπὶ μὲν τοὺς πορθμέας τοὺς εἰς Σαλαμῖνα πορθμεύοντας νόμον ἔθεσθε, ἐάν τις αὐτῶν ἄκων ἐν τῷ πόρῳ πλοῖον ἀνατρέψῃ, τούτῳ μὴ ἐξεῖναι πάλιν πορθμεῖ γενέσθαι, ἵνα μηδεὶς αὐτοσχεδιάζῃ εἰς τὰ τῶν Ἑλλήνων σώματα, τὸν δὲ τὴν Ἑλλάδα καὶ τὴν πόλιν ἄρδην ἀνατετροφότα, τοῦτον ἐάσετε πάλιν ἀπευθύνειν τὰ κοινά;\\
Aeschines (0026) & Against Ctesiphon (003) & 199.1 & ὅλως δ’ ἔγωγε, ὦ ἄνδρες Ἀθηναῖοι, ὀλίγου δέω εἰπεῖν ὡς καὶ νόμον δεῖ τεθῆναι ἐπὶ ταῖς γραφαῖς μόναις ταῖς τῶν παρανόμων, μὴ ἐξεῖναι μήτε τῷ κατηγόρῳ συνηγόρους παρασχέσθαι, μήτε τῷ τὴν γραφὴν φεύγοντι.\\
Aeschines (0026) & Against Timarchus (001) & 37.4 & δέομαι δ’ ὑμῶν, ὦ ἄνδρες Ἀθηναῖοι, συγγνώμην ἔχειν, ἐὰν ἀναγκαζόμενος λέγειν περὶ ἐπιτηδευμάτων φύσει μὲν μὴ καλῶν, τούτῳ δὲ πεπραγμένων, ἐξαχθῶ τι ῥῆμα εἰπεῖν ὅ ἐστιν ὅμοιον τοῖς ἔργοις τοῖς Τιμάρχου.\\
Aeschines (0026) & The Speech on the Embassy (002) & t.1 & δέομαι ὑμῶν, ὦ ἄνδρες Ἀθηναῖοι, ἐθελῆσαί μου μετ’ εὐνοίας ἀκοῦσαι λέγοντος, ὑπολογιζομένους τό τε μέγεθος τοῦ κινδύνου καὶ τὸ πλῆθος τῶν αἰτιῶν πρὸς ἂς ἀπολογήσασθαί με δεῖ, καὶ τὰς τέχνας καὶ τὰς κατασκευὰς τοῦ κατηγόρου καὶ τὴν ὠμότητα, ὃς ἐτόλμησε παρακελεύσασθαι πρὸς ἄνδρας ὀμωμοκότας τῶν ἀντιδίκων ὁμοίως ἀμφοτέρων ἀκούσεσθαι τοῦ κινδυνεύοντος φωνὴν μὴ ὑπομένειν.\\
Aeschines (0026) & The Speech on the Embassy (002) & 6.6 & περὶ δὲ τῆς ἄλλης κατηγορίας δέομαι ὑμῶν, ὦ ἄνδρες Ἀθηναῖοι, ἐάν τι παραλίπω καὶ μὴ μνησθῶ, ἐπερωτᾶν με καὶ δηλοῦν ὅ τι ἂν ποθῆτε ἀκοῦσαι, μηδὲν προκατεγνωκότας, ἀλλ’ ἴσῃ τῇ εὐνοίᾳ ἀκούοντας.\\
\addlinespace
Aeschines (0026) & The Speech on the Embassy (002) & 44.5 & δέομαι δὲ ὑμῶν προσεπιπονῆσαι ἀκούοντας καὶ τὴν λοιπὴν διήγησιν.\\
Aeschylus (0085) & Eumenides (007) & 890 & ἔξεστι γάρ σοι τῆσδε γαμόρῳ χθονὸς εἶναι δικαίως ἐς τὸ πᾶν τιμωμένῃ.\\
Aeschylus (0085) & Libation Bearers (006) & 140 & αὐτῇ τέ μοι δὸς σωφρονεστέραν πολὺ μητρὸς γενέσθαι χεῖρά τ’ εὐσεβεστέραν.\\
Aeschylus (0085) & Prometheus Bound (003) & 701 & μαθεῖν γὰρ τῆσδε πρῶτ’ ἐχρῄζετε τὸν ἀμφ’ ἑαυτῆς ἆθλον ἐξηγουμένης·\\
Andocides (0027) & On the Mysteries (001) & 28.1 & ἔδοξεν οὖν τῷ δήμῳ ἐν τῷ τῶν θεσμοθετῶν δικαστηρίω τοὺς μεμυημένους, ἀκούσαντας τὰς μηνύσεις ἅς ἕκαστος ἐμήνυσε, διαδικάσαι.\\
\addlinespace
Andocides (0027) & On the Mysteries (001) & 88.5 & ὅπου οὖν ἔδοξεν ὑμῖν δοκιμάσαι μὲν τοὺς νόμους, δοκιμάσαντας δὲ ἀναγράψαι, ἀγράφῳ δὲ νόμῳ τὰς ἀρχὰς μὴ χρῆσθαι μηδὲ περὶ ἑνός, ψήφισμα δὲ μηδὲν μήτε βουλῆς μήτε δήμου νόμου κυριώτερον εἶναι, μηδ’ ἐπ’ ἀνδρὶ νόμον ἐξεῖναι τιθέναι ἐὰν μὴ τὸν αὐτὸν ἐπὶ πᾶσιν Ἀθηναίοις, τοῖς δὲ νόμοις τοῖς κειμένοις χρῆσθαι ἀπ’ Εὐκλείδου ἄρχοντος, ἐνταυθοῖ ἔστιν ὅ τι ὑπολείπεται ἢ μεῖζον ἢ ἔλαττον τῶν γενομένων πρότερον ψηφισμάτων, πρὶν Εὐκλείδην ἄρξαι, ὅπως κύριον ἔσται;\\
Andocides (0027) & On the Mysteries (001) & 3.7 & αὐτίκα ἐγὼ πολλῶν μοι ἀπαγγελλόντων ὅτι λέγοιεν οἱ ἐχθροὶ ὡς ἄρα ἐγὼ οὔτ’ ἂν ὑπομείναιμι οἰχήσομαί τε φεύγων, —τί γὰρ ἂν καὶ βουλόμενος Ἀνδοκίδης ἀγῶνα τοσοῦτον ὑπομείνειεν, ᾧ ἔξεστι μὲν ἀπελθόντι ἐντεῦθεν ἔχειν πάντα τὰ ἐπιτήδεια, ἔστι δὲ πλεύσαντι εἰς Κύπρον, ὅθεν περ ἥκει, γῆ πολλὴ καὶ ἀγαθὴ διδομένη καὶ δωρεὰν ὑπάρχουσα;\\
Andocides (0027) & On the Mysteries (001) & 136.2 & ὡς γὰρ πλείστους εἶναι ὑμῖν ἤθελον ἂν τοιούσδε οἷόσπερ ἐγώ, τούτους δὲ μάλιστα μὲν ἀπολωλέναι, εἰ δὲ μή, εἶναι τοὺς μὴ ἐπιτρέψοντας αὐτοῖς, οἷς καὶ προσήκει ἀνδράσιν εἶναι καὶ ἀγαθοῖς καὶ δικαίοις περὶ τὸ πλῆθος τὸ ὑμέτερον, καὶ βουλόμενοι δυνήσονται εὖ ποιεῖν ὑμᾶς.\\
Andocides (0027) & On the Mysteries (001) & 149.3 & ὑμεῖς με παρ’ ὑμῶν αὐτῶν αἰτησάμενοι σῴσατε, καὶ μὴ βούλεσθε Θετταλοὺς καὶ Ἀνδρίους πολίτας ποιεῖσθαι δι’ ἀπορίαν ἀνδρῶν, τοὺς δὲ ὄντας πολίτας ὁμολογουμένως, οἷς προσήκει ἀνδράσιν ἀγαθοῖς εἶναι καὶ βουλόμενοι δυνήσονται, τούτους δὲ ἀπόλλυτε.\\
Andocides (0027) & On His Return (002) & 26.9 & ὥστ’ ἔμοιγε καὶ διὰ τὰ τῶν προγόνων ἔργα εἰκότως ὑπάρχει δημοτικῷ εἶναι, εἴπερ τι ἀλλὰ νῦν γε φρονῶν τυγχάνω.\\
\addlinespace
Andocides (0027) & On the Mysteries (001) & 84.3 & ἐξεῖναι δὲ καὶ ἰδιώτῃ τῷ βουλομένῳ, εἰσιόντι εἰς τὴν βουλὴν συμβουλεύειν ὅ τι ἂν ἀγαθὸν ἔχῃ περὶ τῶν νόμων.\\
Andocides (0027) & On His Return (002) & 26.3 & τάδε γὰρ οὐ ψευσαμένῳ μοι λαθεῖν οἷόν τ’ ἐστι τούς γε πρεσβυτέρους ὑμῶν, ὅτι ὁ τοῦ ἐμοῦ πατρὸς πάππος Λεωγόρας στασιάσας πρὸς τοὺς τυράννους ὑπὲρ τοῦ δήμου, ἐξὸν αὐτῷ διαλλαχθέντι τῆς ἔχθρας καὶ γενομένῳ κηδεστῇ ἄρξαι μετ’ ἐκείνων τῶν ἀνδρῶν τῆς πόλεως, εἵλετο μᾶλλον ἐκπεσεῖν μετὰ τοῦ δήμου καὶ φεύγων κακοπαθεῖν μᾶλλον ἢ προδότης αὐτῶν καταστῆναι.\\
Andocides (0027) & Against Alcibiades (004) & 7.3 & δέομαι δ’ ὑμῶν τῶν λόγων ἴσους καὶ κοινοὺς ἡμῖν ἐπιστάτας γενέσθαι, καὶ πάντας ἄρχοντας περὶ τούτων καταστῆναι, καὶ μήτε τοῖς λοιδορουμένοις μήτε τοῖς ὑπὲρ καιρὸν χαριζομένοις ἐπιτρέπειν, ἀλλὰ τῷ μὲν θέλοντι λέγειν καὶ ἀκούειν εὐμενεῖς εἶναι, τῷ δὲ ἀσελγαίνοντι καὶ θορυβοῦντι χαλεπούς.\\
Andocides (0027) & On the Mysteries (001) & 9.7 & τάδε δὲ ὑμῶν δέομαι, μετ’ εὐνοίας μου τὴν ἀκρόασιν τῆς ἀπολογίας ποιήσασθαι, καὶ μήτ’ ἐμοὶ ἀντιδίκους καταστῆναι μήτε ὑπονοεῖν τὰ λεγόμενα μήτε ῥήματα θηρεύειν, ἀκροασαμένους δὲ διὰ τέλους τῆς ἀπολογίας τότε ἤδη ψηφίζεσθαι τοῦτο ὅ τι ἂν ὑμῖν αὐτοῖς ἄριστον καὶ εὐορκότατον νομίζητε εἶναι.\\
Andocides (0027) & On the Mysteries (001) & 37.5 & καὶ τούτοις, ὦ ἄνδρες, δέομαι ὑμῶν προσέχοντας τὸν νοῦν ἀναμιμνῄσκεσθαι, ἐὰν ἀληθῆ λέγω, καὶ διδάσκειν ἀλλήλους·\\
\addlinespace
Andocides (0027) & On the Mysteries (001) & 141.1 & δέομαι οὖν ἁπάντων ὑμῶν περὶ ἐμοῦ τὴν αὐτὴν γνώμην ἔχειν, ἥνπερ καὶ περὶ τῶν ἐμῶν προγόνων, ἵνα κἀμοὶ ἐγγένηται ἐκείνους μιμήσασθαι, ἀναμνησθέντας αὐτῶν ὅτι ὅμοιοι τοῖς πλείστων καὶ μεγίστων ἀγαθῶν αἰτίοις τῇ πόλει γεγένηνται, πολλῶν ἕνεκα σφᾶς αὐτοὺς παρέχοντες τοιούτους, μάλιστα δὲ τῆς εἰς ὑμᾶς εὐνοίας, καὶ ὅπως, εἴ ποτέ τις αὐτοῖς ἢ τῶν ἐξ ἐκείνων τινὶ κίνδυνος γένοιτο ἢ συμφορά, σῴζοιντο συγγνώμης παρ’ ὑμῶν τυγχάνοντες.\\
Antiphon (0028) & First Tetralogy (002) & 1.8.1 & εἴ τε καὶ ἁλοίη, τιμωρησαμένῳ κάλλιον ἔδοξεν αὐτῷ ταῦτα πάσχειν, ἢ ἀνάνδρως μηδὲν ἀντιδράσαντα ὑπὸ τῆς γραφῆς διαφθαρῆναι.\\
Antiphon (0028) & Prosecution Of The Stepmother For Poisoning (001) & 17.2 & ἔδοξεν οὖν αὐτῇ βουλευομένῃ βέλτιον εἶναι μετὰ δεῖπνον δοῦναι, τῆς Κλυταιμνήστρας ταύτης τῆς τούτου μητρὸς ταῖς ὑποθήκαις ἅμα διακονοῦσαν.\\
Antiphon (0028) & On the Choreutes (006) & 9.2 & ἵνα μὲν ἐξῆν αὐτοῖς, εἴ τι ἠδίκουν ἐγὼ τὴν πόλιν ἢ ἐν χορηγίᾳ ἢ ἐν ἄλλοις τισίν, ἀποφήνασι καὶ ἐξελέγξασιν ἄνδρα τε ἐχθρὸν τιμωρήσασθαι καὶ τὴν πόλιν ὠφελῆσαι, ἐνταῦθα μὲν οὐδεὶς πώποτε οἷός τε ἐγένετο αὐτῶν οὔτε μικρὸν οὔτε μέγα ἐξελέγξαι ἀδικοῦντα τόνδε τὸν ἄνδρα τὸ πλῆθος τὸ ὑμέτερον·\\
Antiphon (0028) & On the Murder of Herodes (005) & 90.7 & καὶ φεισαμένοις μὲν ὑμῖν ἐμοῦ νῦν ἔξεστι τότε χρῆσθαι ὅ τι ἂν βούλησθε, ἀπολέσασι δὲ οὐδὲ βουλεύσασθαι ἔτι περὶ ἐμοῦ ἐγχωρεῖ.\\
\addlinespace
Antiphon (0028) & On the Choreutes (006) & 2.3 & ὑπάρχει μὲν γὰρ αὐτοῖς ἀρχαιοτάτοις εἶναι ἐν τῇ γῇ ταύτῃ, ἔπειτα τοὺς αὐτοὺς αἰεὶ περὶ τῶν αὐτῶν, ὅπερ μέγιστον σημεῖον νόμων καλῶς κειμένων·\\
Antiphon (0028) & On the Murder of Herodes (005) & 14.3 & ὑπάρχει μέν γε αὐτοῖς ἀρχαιοτάτοις εἶναι ἐν τῇ γῇ ταύτῃ, ἔπειτα τοὺς αὐτοὺς ἀεὶ περὶ τῶν αὐτῶν, ὅπερ μέγιστόν ἐστι σημεῖον νόμων καλῶς κειμένων·\\
Antiphon (0028) & On the Choreutes (006) & 44.1 & ἐπειδὴ δὲ οὑτοσὶ ὁ βασιλεὺς εἰσῆλθεν, ἐξὸν ἀυτοῖς ἀπὸ τῆς πρώτης ἡμέρας ἀρξαμένοις τοῦ Ἑκατομβαιῶνος μηνὸς τριάκονθ’ ἡμέρας συνεχῶς τούτων ᾗτινι ἐβούλοντο ἀπογράφεσθαι, ἀπεγράφοντο οὐδεμιᾷ·\\
Antiphon (0028) & On the Choreutes (006) & 44.1 & καὶ αὖθις τοῦ Μεταγειτνιῶνος μηνὸς ἀπὸ τῆς πρώτης ἡμέρας ἀρξαμένοις ἐξὸν αὐτοῖς ἀπογράφεσθαι <ᾗ>τινι ἐβούλοντο, οὐδ’ αὖ πω ἐνταῦθα ἀπεγράψαντο, ἀλλὰ παρεῖσαν καὶ τούτου τοῦ μηνὸς εἴκοσιν ἡμέρας·\\
Antiphon (0028) & On the Murder of Herodes (005) & 61.2 & ἐξὸν γὰρ αὐτῷ ἐν ἀγῶνι καὶ κινδύνῳ μεγάλῳ καταστήσαντι μετὰ τῶν νόμων τῶν ὑμετέρων ἀπολέσαι ἐκεῖνον, εἴπερ προωφείλετο αὐτῷ κακόν, καὶ τό τε ἴδιον τὸ αὑτοῦ διαπράξασθαι καὶ τῇ πόλει τῇ ὑμετέρᾳ χάριν καταθέσθαι, εἰ ἐπέδειξεν ἀδικοῦντα ἐκεῖνον, οὐκ ἠξίωσεν, ἀλλ’ οὐδ’ ἦλθεν ἐπὶ τοῦτον.\\
\addlinespace
Antiphon (0028) & First Tetralogy (002) & 2.13.5 & δέομαι δ’ ὑμῶν, ὦ ἄνδρες, τῶν μεγίστων κριταὶ καὶ κύριοι, ἐλεήσαντας τὴν ἀτυχίαν μου ἰατροὺς γενέσθαι αὐτῆς, καὶ μὴ συνεπιβάντας τῇ τούτων ἐπιθέσει περιιδεῖν ἀδίκως καὶ ἀθέως διαφθαρέντα με ὑπ’ αὐτῶν.\\
Antiphon (0028) & Second Tetralogy (003) & 2.1.4 & ὑπὸ δὲ σκληρᾶς ἀνάγκης βιαζόμενος, καὶ αὐτὸς εἰς τὸν ὑμέτερον ἔλεον, ὦ ἄνδρες δικασταί, καταπεφευγὼς δέομαι ὑμῶν, ἐὰν ἀκριβέστερον ἢ ὡς σύνηθες ὑμῖν δόξω εἰπεῖν, μὴ διὰ τὰς προειρημένας τύχας λτδυσχερῶσγτ ἀποδεξαμένους μου τὴν ἀπολογίαν δόξῃ καὶ μὴ ἀληθείᾳ τὴν κρίσιν ποιήσασθαι·\\
Antiphon (0028) & Second Tetralogy (003) & 3.3.4 & ἐγὼ δὲ δράσας μὲν οὐδὲν κακόν, παθὼν δὲ ἄθλια καὶ δεινά, καὶ νῦν ἔτι δεινότερα τούτων, ἔργῳ καὶ οὐ λόγῳ εἰς τὸν ὑμέτερον ἔλεον καταπεφευγὼς δέομαι ὑμῶν, ὦ ἄνδρες ἀνοσίων ἔργων τιμωροί, ὁσίων δὲ διαγνώμονες, μὴ λτπαρ’γτ ἔργα φανερὰ ὑπὸ πονηρᾶς λόγων ἀκριβείας πεισθέντας ψευδῆ τὴν ἀλήθειαν τῶν πραχθέντων ἡγήσασθαι·\\
Antiphon (0028) & Third Tetralogy (004) & 3.7.3 & ἡμεῖς δὲ τόν <τε> θάνατον φανερὸν ἀποδεικνύντες, τήν τε πληγὴν ὁμολογουμένην ἐξ ἧς ἀπέθανε, τόν τε νόμον εἰς τὸν πατάξαντα τὸν φόνον ἀνάγοντα, ἀντὶ τοῦ παθόντος ἐπισκήπτομεν ὑμῖν, τῷ τούτου φόνῳ τὸ μήνιμα τῶν ἀλιτηρίων ἀκεσαμένους πᾶσαν τὴν πόλιν καθαρὰν τοῦ μιάσματος καταστῆσαι.\\
Antiphon (0028) & Third Tetralogy (004) & 4.10.1 & οὑτωσὶ δὲ ἐκ παντὸς τρόπου τῶν ἐγκλημάτων ἀπολυομένου τοῦ ἀνδρός, ἡμεῖς ὁσιώτερον ὑμῖν ἐπισκήπτομεν ὑπὲρ αὐτοῦ, μὴ τὸν φονέα ζητοῦντας κολάζειν τὸν καθαρὸν ἀποκτείνειν.\\
\addlinespace
Antiphon (0028) & Third Tetralogy (004) & 3.t.2 & τοῦτόν τε οὐ θαυμάζω ἀνόσια δράσαντα ὅμοια οἷς εἴργασται λέγειν, ὑμῖν τε συγγιγνώσκω βουλομένοις τὴν ἀκρίβειαν τῶν πραχθέντων μαθεῖν τοιαῦτα ἀνέχεσθαι ἀκούοντας αὐτοῦ, ἃ ἐκβάλλεσθαι ἄξιά ἐστι.\\
Aristophanes (0019) & Lysistrata (007) & 524 & μετὰ ταῦθ’ ἡμῖν εὐθὺς ἔδοξεν σῶσαι τὴν Ἑλλάδα κοινῇ ταῖσι γυναιξὶν συλλεχθείσαις.\\
Aristophanes (0019) & Plutus (011) & 1150 & ταὐτομολεῖν ἀστεῖον εἶναί σοι δοκεῖ;\\
Aristophanes (0019) & Thesmophoriazusae (008) & 428 & νῦν οὖν ἐμοὶ τούτῳ δοκεῖ ὄλεθρόν τιν’ ἡμᾶς κυρκανᾶν ἁμωσγέπως, ἢ φαρμάκοισιν ἢ μιᾷ γέ τῳ τέχνῃ, ὅπως ἀπολεῖται.\\
Aristophanes (0019) & Wasps (004) & 31 & ἔδοξέ μοι περὶ πρῶτον ὕπνον ἐν τῇ πυκνὶ ἐκκλησιάζειν πρόβατα συγκαθήμενα, βακτηρίας ἔχοντα καὶ τριβώνια·\\
\addlinespace
Aristophanes (0019) & Lysistrata (007) & 521 & πῶς ὀρθῶς ὦ κακόδαιμον, εἰ μηδὲ κακῶς βουλευομένοις ἐξῆν ὑμῖν ὑποθέσθαι;\\
Aristophanes (0019) & Thesmophoriazusae (008) & 352 & ξυνευχόμεσθα τέλεα μὲν πόλει τέλεα δὲ δήμῳ τάδ’ εὔγματα γενέσθαι, τὰ δ’ ἄρισθ’ ὅσαις προσήκει νικᾶν λεγούσαις·\\
Demosthenes (0014) & Against Timocrates (024) & 112.1 & δικαίως δ’ ἂν ἐμοὶ δοκεῖ παθεῖν ὁτιοῦν, ὅστις οἴεται δεῖν, εἰ μέν τις ἀγορανόμος ἢ ἀστυνόμος ἢ δικαστὴς κατὰ δήμους γενόμενος κλοπῆς ἐν ταῖς εὐθύναις ἑάλωκεν, ἄνθρωπος πένης καὶ ἰδιώτης καὶ πολλῶν ἄπειρος καὶ κληρωτὴν ἀρχὴν ἄρξας, τούτῳ μὲν τὴν δεκαπλασίαν εἶναι, καὶ νόμον οὐδένα τοῖς τοιούτοις ἐπικουροῦντα τίθησιν·\\
Demosthenes (0014) & On The Trierarchic Crown (051) & 12.1 & ἔτι τοίνυν ἔμοιγε δοκεῖ κἀκεῖν’ ἀλόγως ἔχειν, τὸν μὲν εἰπόντα τι μὴ κατὰ τοὺς νόμους, ἐὰν ἁλῷ τὸ τρίτον, μέρος ἠτιμῶσθαι τοῦ σώματος, τοὺς δὲ μὴ λόγον, ἀλλ’ ἔργον παράνομον πεποιηκότας μηδεμίαν δοῦναι δίκην.\\
Demosthenes (0014) & On the Public Fund (013) & 1.1 & περὶ μὲν τοῦ παρόντος ἀργυρίου καὶ ὧν ἕνεκα τὴν ἐκκλησίαν ποιεῖσθ’, ὦ ἄνδρες Ἀθηναῖοι, οὐδέτερόν μοι δοκεῖ τῶν χαλεπῶν εἶναι, οὔτ’ ἐπιτιμήσαντα τοῖς νέμουσι καὶ διδοῦσι τὰ κοινὰ εὐδοκιμῆσαι παρὰ τοῖς βλάπτεσθαι διὰ τούτων ἡγουμένοις τὴν πόλιν, οὔτε συνειπόντα καὶ παραινέσανθ’ ὡς δεῖ λαμβάνειν, χαρίσασθαι τοῖς σφόδρ’ ἐν χρείᾳ τοῦ λαβεῖν οὖσιν·\\
\addlinespace
Demosthenes (0014) & The Funeral Speech 51-61 (060) & 5.1 & δοκεῖ δέ μοι καὶ τὸ τοὺς καρπούς, οἷς ζῶσιν ἅνθρωποι, παρ’ ἡμῖν πρώτοις φανῆναι, χωρὶς τοῦ μέγιστον εὐεργέτημ’ εἰς πάντας γενέσθαι, ὁμολογούμενον σημεῖον ὑπάρχειν τοῦ μητέρα τὴν χώραν εἶναι τῶν ἡμετέρων προγόνων.\\
Demosthenes (0014) & Third Philippic (009) & 63.3 & ὅπερ καὶ παρ’ ὑμῖν, ὅτι τοῖς μὲν ὑπὲρ τοῦ βελτίστου λέγουσιν οὐδὲ βουλομένοις ἔνεστιν ἐνίοτε πρὸς χάριν οὐδὲν εἰπεῖν·\\
Demosthenes (0014) & Against Aphobus 1 (027) & 58.3 & τούτῳ γὰρ ἐξῆν μηδὲν ἔχειν τούτων τῶν πραγμάτων, μισθώσαντι τὸν οἶκον κατὰ τουτουσὶ τοὺς νόμους.\\
Demosthenes (0014) & Against Aristocrates (023) & 28.1 & καὶ λαβοῦσιν ἐκείνοις ἐξέσται στρεβλοῦν, αἰκίσασθαι, χρήματα πράξασθαι.\\
Demosthenes (0014) & Against Callicles (055) & 5.1 & καίτοι, ὦ Καλλίκλεις, ἐξῆν δήπου τόθ’ ὑμῖν, ὁρῶσιν ἀποικοδομουμένην τὴν χαράδραν, ἐλθοῦσιν εὐθὺς ἀγανακτεῖν καὶ λέγειν πρὸς τὸν πατέρα Τεισία, τί ταῦτα ποιεῖς;\\
\addlinespace
Demosthenes (0014) & Against Dionysodorus (056) & 49.1 & εἰ μέντοι ἐξέσται τοῖς ναυκλήροις, συγγραφὴν γραψαμένοις ἐφ’ ᾧ τε καταπλεῖν εἰς Ἀθήνας, ἔπειτα κατάγειν τὴν ναῦν εἰς ἕτερα ἐμπόρια, φάσκοντας ῥαγῆναι καὶ τοιαύτας προφάσεις ποριζομένους οἵαισπερ καὶ Διονυσόδωρος οὑτοσὶ χρῆται, καὶ τοὺς τόκους μερίζειν πρὸς τὸν πλοῦν ὃν ἂν φήσωσιν πεπλευκέναι, καὶ μὴ πρὸς τὴν συγγραφήν, οὐδὲν κωλύσει ἅπαντα τὰ συμβόλαια διαλύεσθαι.\\
Demosthenes (0014) & Against Eubulides (057) & 53.2 & ἐξῆν δὲ δήπου τούτοις, εἰ νόθος ἢ ξένος ἦν ἐγώ, κληρονόμοις εἶναι τῶν ἐμῶν πάντων.\\
Demosthenes (0014) & Third Olynthiac (003) & 23.4 & οὐ γὰρ ἀλλοτρίοις ὑμῖν χρωμένοις παραδείγμασιν, ἀλλ’ οἰκείοις, ὦ ἄνδρες Ἀθηναῖοι, εὐδαίμοσιν ἔξεστι γενέσθαι.\\
Demosthenes (0014) & Against Leptines (020) & 157.3 & ὑμῖν δ’ οὐχὶ πρέπει τὰ τοιαῦτα μιμεῖσθαι, οὐδ’ ἀνάξια φαίνεσθαι φρονοῦντας ὑμῶν αὐτῶν.\\
Demosthenes (0014) & Against Androtion (022) & 34.4 & κληρονόμον γάρ σε καθίστησ’ ὁ νόμος τῆς ἀτιμίας τῆς τοῦ πατρός, ὄντι δ’ ἀτίμῳ σοι λέγειν οὐ προσῆκεν οὐδὲ γράφειν.\\
\addlinespace
Demosthenes (0014) & Against Androtion (022) & 58.4 & ὧν προσῆκέ σοι τὴν ὀργὴν οὐκ εἰς τῶν πολιτῶν τὸν τυχόντ’ ἀφιέναι οὐδ’ εἰς τὰς ὁμοτέχνους πόρνας, ἀλλ’ εἰς τὸν τοῦτον τὸν τρόπον σε θρέψαντα.\\
Demosthenes (0014) & Against Boeotus 1 (039) & 38.2 & καίτοι ἐξ ἀρχῆς τ’ ἔδει ἐᾶν αὐτὸν τελέσασθαι τὴν δίκην κατὰ Βοιωτοῦ, εἴπερ μηδὲν προσῆκεν αὐτῷ τοὐνόματος, ὕστερόν τε μὴ αὐτὸν φαίνεσθαι ἐπὶ τῷ ὀνόματι τούτῳ ἀντιλαγχάνοντα τὴν μὴ οὖσαν.\\
Demosthenes (0014) & Against Eubulides (057) & 32.4 & προσήκει τοίνυν ὑμῖν βοηθοῦσι τοῖς νόμοις μὴ τοὺς ἐργαζομένους ξένους νομίζειν, ἀλλὰ τοὺς συκοφαντοῦντας πονηρούς.\\
Demosthenes (0014) & Against Eubulides (057) & 36.4 & ἀλλ’ ἀκούσαντες, ἐὰν ὑμῖν ἐπιδεικνύω τῆς μητρὸς τοὺς οἰκείους οἵους προσήκει εἶναι ἀνθρώποις ἐλευθέροις, ἃ καταιτιᾶται περὶ αὐτῆς, ταύτας τὰς διαβολὰς ἐξομνυμένους, καὶ μαρτυροῦντας αὐτὴν ἀστὴν οὖσαν εἰδέναι, οὓς ὑμεῖς φήσετε πιστοὺς εἶναι, δικαίαν ἡμῖν θέσθε τὴν ψῆφον.\\
Demosthenes (0014) & Against Leptines (020) & 90.1 & οὐ γὰρ ᾤετο δεῖν ὁ Σόλων, ὁ τοῦτον τὸν τρόπον προστάξας νομοθετεῖν, τοὺς μὲν θεσμοθέτας τοὺς ἐπὶ τοὺς νόμους κληρουμένους δὶς δοκιμασθέντας ἄρχειν, ἔν τε τῇ βουλῇ καὶ παρ’ ὑμῖν ἐν τῷ δικαστηρίῳ, τοὺς δὲ νόμους αὐτούς, καθ’ οὓς καὶ τούτοις ἄρχειν καὶ πᾶσι τοῖς ἄλλοις πολιτεύεσθαι προσήκει, ἐπὶ καιροῦ τεθέντας, ὅπως ἔτυχον, μὴ δοκιμασθέντας κυρίους εἶναι.\\
\addlinespace
Demosthenes (0014) & Against Meidias (021) & 72.9 & σκέψασθε δὴ πρὸς Διὸς καὶ θεῶν, ὦ ἄνδρες Ἀθηναῖοι, καὶ λογίσασθε παρ’ ὑμῖν αὐτοῖς, ὅσῳ πλείον’ ὀργὴν ἐμοὶ προσῆκε παραστῆναι πάσχοντι τοιαῦθ’ ὑπὸ Μειδίου ἢ τότ’ ἐκείνῳ τῷ Εὐαίωνι τῷ τὸν Βοιωτὸν ἀποκτείναντι.\\
Demosthenes (0014) & For the Megalopolitans (016) & 23.5 & εἰ μὲν γὰρ ὑπὲρ ἐκείνων, οὐδετέροις ὡς μαινομένοις πείθεσθαι προσήκει·\\
Demosthenes (0014) & Against Aristocrates (023) & 8.6 & συμβέβηκε γὰρ ἐκ τούτου αὑτοῖς μὲν ἀντιπάλους εἶναι τούτους, ὑμᾶς δ’ ὑπέρχεσθαι καὶ θεραπεύειν.\\
Demosthenes (0014) & Against Aristocrates (023) & 143.9 & τοῦτο τοίνυν ἐπ’ ἐκείνου μέν, εὖ ποιοῦν, οὐ συνέβη φενακισθεῖσιν ὑμῖν αἰσχύνην ὀφλεῖν·\\
Demosthenes (0014) & Against Aristocrates (023) & 171.1 & ὡς δ’ ἐν ἀρχαιρεσίαις ὑμεῖς Χαβρίαν ἐπὶ τὸν πόλεμον τοῦτον κατεστήσατε, καὶ τῷ μὲν Ἀθηνοδώρῳ συνέβη διαφεῖναι τὴν δύναμιν χρήματ’ οὐκ ἔχοντι παρ’ ὑμῶν οὐδ’ ἀφορμὴν τῷ πολέμῳ, τῷ Χαβρίᾳ δὲ μίαν ναῦν ἔχοντι μόνην ἐκπλεῖν, τί ποιεῖ πάλιν οὗτος ὁ Χαρίδημος;\\
\addlinespace
Demosthenes (0014) & Against Conon (054) & 8.1 & καὶ ἡμῖν συμβαίνει ἀναστρέφουσιν ἀπὸ τοῦ Φερρεφαττίου καὶ περιπατοῦσιν πάλιν κατ’ αὐτό πως τὸ Λεωκόριον εἶναι, καὶ τούτοις περιτυγχάνομεν.\\
Demosthenes (0014) & Against Conon (054) & 44.5 & εἰ γὰρ δὴ ὁμολογουμένως ἔτι τούτων καὶ ἀχρηστοτέροις καὶ πονηροτέροις ἡμῖν εἶναι συνέβαινεν, οὐ τυπτητέοι, οὐδ’ ὑβριστέοι δήπου ἐσμέν.\\
Demosthenes (0014) & Against Meidias (021) & 89.1 & πάνυ γε, ἀλλὰ δάκνει καὶ τοῦτο, φαίη τις ἄν, ὅταν ἐκτίνειν ἀδίκως δέῃ, συνέβη δ’ ὑπερημέρῳ γενομένῳ λαθεῖν αὐτῷ διὰ τὸ ἀδικηθῆναι.\\
Demosthenes (0014) & Apollodorus Against Timotheus (049) & 23.5 & ἀπολελυμένῳ τοίνυν τῆς αἰτίας πολλὴ συνέβαινεν αὐτῷ μετὰ ταῦτα χρημάτων ἀπορία εἴς τε τὰς ἰδίας χρείας καὶ εἰς τὰς δημοσίας εἰσφοράς, ἃ ὁρῶν ὁ πατὴρ ὁ ἐμὸς οὐκ ἐτόλμα τοῦτον εὐθὺς ἀπαιτεῖν τὸ ἀργύριον·\\
Demosthenes (0014) & On the Crown (018) & 46.1 & εἶτ’, οἶμαι, συμβέβηκε τοῖς μὲν πλήθεσιν ἀντὶ τῆς πολλῆς καὶ ἀκαίρου ῥᾳθυμίας τὴν ἐλευθερίαν ἀπολωλεκέναι, τοῖς δὲ προεστηκόσι καὶ τἄλλα πλὴν ἑαυτοὺς οἰομένοις πωλεῖν πρώτους ἑαυτοὺς πεπρακόσιν αἰσθέσθαι·\\
\addlinespace
Demosthenes (0014) & On the Crown (018) & 146.3 & συνέβαινε δ’ αὐτῷ τῷ πολέμῳ κρατοῦντι τοὺς ὁποιουσδήποθ’ ὑμεῖς ἐξεπέμπετε στρατηγούς (ἐῶ γὰρ τοῦτό γε) αὐτῇ τῇ φύσει τοῦ τόπου καὶ τῶν ὑπαρχόντων ἑκατέροις κακοπαθεῖν.\\
Demosthenes (0014) & On the Crown (018) & 160.1 & συμβέβηκε τοίνυν μοι τῶν κατὰ τῆς πατρίδος τούτῳ πεπραγμένων ἁψαμένῳ εἰς ἃ τούτοις ἐναντιούμενος αὐτὸς πεπολίτευμαι ἀφῖχθαι·\\
Demosthenes (0014) & Third Philippic (009) & 4.1 & εἶθ’ ὑμῖν συμβέβηκεν ἐκ τούτου ἐν μὲν ταῖς ἐκκλησίαις τρυφᾶν καὶ κολακεύεσθαι πάντα πρὸς ἡδονὴν ἀκούουσιν, ἐν δὲ τοῖς πράγμασι καὶ τοῖς γιγνομένοις περὶ τῶν ἐσχάτων ἤδη κινδυνεύειν.\\
Demosthenes (0014) & Against Aristocrates (023) & 138.1 & ὅτι τοίνυν ἄνευ τοῦ τοῖς πράγμασι μὴ συμφέρειν τὸ ψήφισμα, οὐδὲ πρὸς δόξαν συμφέρει τῇ πόλει τοιοῦτον οὐδὲν ἐψηφισμένῃ φαίνεσθαι, καὶ τοῦτο δεῖ μαθεῖν ὑμᾶς.\\
Demosthenes (0014) & Against Macartatus (043) & 66.8 & συμφέρει Ἀθηναίοις περὶ τοῦ σημείου τοῦ ἐν τῷ οὐρανῷ γενομένου θύοντας καλλιερεῖν Διὶ ὑπάτῳ, Ἀθηνᾷ ὑπάτῃ, Ἡρακλεῖ, Ἀπόλλωνι σωτῆρι, καὶ ἀποπέμπειν Ἀμφιόνεσσι·\\
\addlinespace
Demosthenes (0014) & For the Liberty of the Rhodians (015) & 13.1 & ἐπεὶ καὶ βασιλέα γε, ὅ τι μὲν ποιήσει μὰ Δί’ οὐκ ἂν εἴποιμ’ ἔγωγ’ ὡς οἶδα, ὅτι μέντοι συμφέρει τῇ πόλει δῆλον ἤδη γενέσθαι πότερ’ ἀντιποιήσεται τῆς πόλεως τῆς Ῥοδίων ἢ οὔ, τοῦτ’ ἂν ἰσχυρισαίμην·\\
Demosthenes (0014) & On the False Embassy (019) & 343.1 & οὔτε γὰρ πρὸς δόξαν οὔτε πρὸς εὐσέβειαν οὔτε πρὸς ἀσφάλειαν οὔτε πρὸς ἄλλ’ οὐδὲν ὑμῖν συμφέρει τοῦτον ἀφεῖναι, ἀλλὰ τιμωρησαμένους παράδειγμα ποιῆσαι πᾶσι, καὶ τοῖς πολίταις καὶ τοῖς ἄλλοις Ἕλλησιν.\\
Demosthenes (0014) & Against Aristocrates (023) & 102.6 & ἐκ γὰρ τοῦ ταῦθ’ οὕτως ἔχειν ἡμῖν ὑπάρχει μεγίστοις οὖσιν ἀσφαλῶς οἰκεῖν.\\
Demosthenes (0014) & On the Crown (018) & 68.1 & καὶ μὴν οὐδὲ τοῦτό γ’ οὐδεὶς ἂν εἰπεῖν τολμήσαι, ὡς τῷ μὲν ἐν Πέλλῃ τραφέντι, χωρίῳ ἀδόξῳ τότε γ’ ὄντι καὶ μικρῷ, τοσαύτην μεγαλοψυχίαν προσῆκεν ἐγγενέσθαι ὥστε τῆς τῶν Ἑλλήνων ἀρχῆς ἐπιθυμῆσαι καὶ τοῦτ’ εἰς τὸν νοῦν ἐμβαλέσθαι, ὑμῖν δ’ οὖσιν Ἀθηναίοις καὶ κατὰ τὴν ἡμέραν ἑκάστην ἐν πᾶσι καὶ λόγοις καὶ θεωρήμασι τῆς τῶν προγόνων ἀρετῆς ὑπομνήμαθ’ ὁρῶσι τοσαύτην κακίαν ὑπάρξαι, ὥστε τῆς ἐλευθερίας αὐτεπαγγέλτους ἐθελοντὰς παραχωρῆσαι Φιλίππῳ.\\
Demosthenes (0014) & On the Crown (018) & 257.1 & ἐμοὶ μὲν τοίνυν ὑπῆρξεν, Αἰσχίνη, παιδὶ μὲν ὄντι φοιτᾶν εἰς τὰ προσήκοντα διδασκαλεῖα, καὶ ἔχειν ὅσα χρὴ τὸν μηδὲν αἰσχρὸν ποιήσοντα δι’ ἔνδειαν, ἐξελθόντι δ’ ἐκ παίδων ἀκόλουθα τούτοις πράττειν, χορηγεῖν, τριηραρχεῖν, εἰσφέρειν, μηδεμιᾶς φιλοτιμίας μήτ’ ἰδίας μήτε δημοσίας ἀπολείπεσθαι, ἀλλὰ καὶ τῇ πόλει καὶ τοῖς φίλοις χρήσιμον εἶναι, ἐπειδὴ δὲ πρὸς τὰ κοινὰ προσελθεῖν ἔδοξέ μοι, τοιαῦτα πολιτεύμαθ’ ἑλέσθαι ὥστε καὶ ὑπὸ τῆς πατρίδος καὶ ὑπ’ ἄλλων Ἑλλήνων πολλῶν πολλάκις ἐστεφανῶσθαι, καὶ μηδὲ τοὺς ἐχθροὺς ὑμᾶς, ὡς οὐ καλά γ’ ἦν ἃ προειλόμην, ἐπιχειρεῖν λέγειν.\\
\addlinespace
Demosthenes (0014) & On the Crown (018) & 266.1 & ἐῶ τἄλλα, ἀλλὰ νυνὶ τήμερον ἐγὼ μὲν ὑπὲρ τοῦ στεφανωθῆναι δοκιμάζομαι, τὸ δὲ μηδ’ ὁτιοῦν ἀδικεῖν ἀνωμολόγημαι, σοὶ δὲ συκοφάντῃ μὲν εἶναι δοκεῖν ὑπάρχει, κινδυνεύεις δ’ εἴτε δεῖ σ’ ἔτι τοῦτο ποιεῖν, εἴτ’ ἤδη πεπαῦσθαι μὴ μεταλαβόντα τὸ πέμπτον μέρος τῶν ψήφων.\\
Demosthenes (0014) & The Funeral Speech 51-61 (060) & 7.1 & οἱ γὰρ τῆς κατὰ τὸν παρόντα χρόνον γενεᾶς πρόγονοι, καὶ πατέρες καὶ τούτων ἐπάνω τὰς προσηγορίας ἔχοντες, αἷς ὑπὸ τῶν ἐν γένει γνωρίζονται, ἠδίκησαν μὲν οὐδένα πώποτ’ οὔθ’ Ἕλλην’ οὔτε βάρβαρον, ἀλλ’ ὑπῆρχεν αὐτοῖς πρὸς ἅπασι τοῖς ἄλλοις καλοῖς κἀγαθοῖς καὶ δικαιοτάτοις εἶναι, ἀμυνόμενοι δὲ πολλὰ καὶ καλὰ διεπράξαντο.\\
Demosthenes (0014) & Against Androtion (022) & 8.1 & περὶ τοίνυν τοῦ νόμου τοῦ διαρρήδην οὐκ ἐῶντος ἐξεῖναι μὴ ποιησαμένῃ τῇ βουλῇ τὰς τριήρεις αἰτῆσαι τὴν δωρειάν, ἄξιόν ἐστιν ἀκοῦσαι τὴν ἀπολογίαν ἣν ποιήσεται, καὶ θεωρῆσαι τὴν ἀναίδειαν τοῦ τρόπου δι’ ὧν ἐγχειρεῖ λέγειν.\\
Demosthenes (0014) & Against Androtion (022) & 11.1 & διὰ ταῦτα γάρ, ὦ ἄνδρες Ἀθηναῖοι, τοῦτον ἔχει τὸν τρόπον ὁ νόμος, μὴ ἐξεῖναι τῇ βουλῇ μὴ ποιησαμένῃ τὰς τριήρεις αἰτῆσαι τὴν δωρειάν, ἵνα μηδὲ πεισθῆναι μηδ’ ἐξαπατηθῆναι γένοιτ’ ἐπὶ τῷ δήμῳ.\\
Demosthenes (0014) & Against Aristogeiton 1 (025) & 29.2 & καὶ μὴν ὅ τι βούλεσθε τούτων ἧττόν ἐστι δεινὸν ἢ εἴ τις ἐξ ὧν οὗτός ἐστι μερῶν εἴποι τοῖς βιαζομένοις ἐξεῖναι λέγειν, ἢ τοῖς ἐκ τοῦ δεσμωτηρίου, ἢ τοῖς ὧν ἀπέκτεινεν ὁ δῆμος τοὺς πατέρας, ἢ τοῖς ἀποδεδοκιμασμένοις ἄρχειν λαχοῦσιν, ἢ τοῖς ὀφείλουσι τῷ δημοσίῳ, ἢ τοῖς καθάπαξ ἀτίμοις, ἢ τοῖς πονηροτάτοις καὶ δοκοῦσι καὶ οὖσι·\\
\addlinespace
Demosthenes (0014) & On the Accession of Alexander (017) & 25.1 & ἕως γὰρ ἂν ἐξῇ τῶν κατὰ θάλατταν καὶ μόνοις ἀναμφισβητήτως εἶναι κυρίοις, τοῖς γε κατὰ γῆν πρὸς τῇ ὑπαρχούσῃ δυνάμει ἔστι προβολὰς ἑτέρας ἰσχυροτέρας εὑρέσθαι, ἄλλως τε καὶ πεπαυμένων ὑπὸ τῆς τύχης τῶν δορυφορουμένων ὑπὸ τῶν τυραννικῶν στρατοπέδων, καὶ τῶν μὲν ἐφθαρμένων, τῶν δὲ ἐξεληλεγμένων οὐδενὸς ἀξίων ὄντων.\\
Demosthenes (0014) & Against Boeotus 2 (040) & 5.1 & οἵτινες, ὦ γῆ καὶ θεοί, ἐξὸν αὐτοῖς τὰ δίκαια ποιήσασι μὴ εἰσιέναι εἰς δικαστήριον, οὐκ αἰσχύνονται μὲν ἀναμιμνῄσκοντες ὑμᾶς εἴ τι ἢ ὁ πατὴρ ἡμῶν μὴ ὀρθῶς διεπράξατο ἢ οὗτοι εἰς ἐκεῖνον ἥμαρτον, ἀναγκάζουσι δ’ ἐμὲ δικάζεσθαι αὑτοῖς.\\
Demosthenes (0014) & Against Dionysodorus (056) & 17.8 & ταῦτα τοίνυν, ὦ ἄνδρες δικασταί, προκαλεσαμένων ἡμῶν Διονυσόδωρον τουτονὶ πολλάκις, καὶ ἐπὶ πολλὰς ἡμέρας ἐκτιθέντων τὴν πρόκλησιν, εὐήθεις ἔφη παντελῶς ἡμᾶς εἶναι, εἰ ὑπολαμβάνομεν αὐτὸν οὕτως ἀλογίστως ἔχειν ὥστ’ ἐπὶ διαιτητὴν βαδίζειν, προδήλου ὄντος ὅτι καταγνώσεται αὐτοῦ ἀποτεῖσαι τὰ χρήματα, ἐξὸν αὐτῷ ἐπὶ τὸ δικαστήριον ἥκειν φέροντα τὸ ἀργύριον, εἶτ’ ἐὰν μὲν δύνηται ὑμᾶς παρακρούσασθαι, ἀπιέναι τἀλλότρια ἔχοντα, εἰ δὲ μή, τηνικαῦτα καταθεῖναι τὰ χρήματα, ὡς ἂν ἄνθρωπος οὐ τῷ δικαίῳ πιστεύων, ἀλλὰ διάπειραν ὑμῶν λαμβάνειν βουλόμενος.\\
Demosthenes (0014) & Against Evergus And Mnesibulus (047) & 5.1 & ἐξὸν γὰρ αὐτοῖς ἀπηλλάχθαι πραγμάτων καὶ μὴ κινδυνεύειν εἰσιόντας εἰς ὑμᾶς, ἔργῳ βεβαιώσαντας ὡς ἀληθής ἐστιν ἡ μαρτυρία, οὐκ ἠθελήκασιν παραδοῦναι τὴν ἄνθρωπον, ἣν μεμαρτυρήκασιν προκαλέσασθαι πρὸς τῷ διαιτητῇ Πυθοδώρῳ ἐκ Κηδῶν παραδιδόναι ἕτοιμον εἶναι Θεόφημον, ἠξίουν δ’ ἐγὼ παραλαμβάνειν, ὡς οἱ μάρτυρες ὑμῖν οἱ τότε παραγενόμενοι ἐμαρτύρησαν ἐν τῷ δικαστηρίῳ, καὶ νῦν δὲ μαρτυρήσουσιν.\\
Demosthenes (0014) & Against Leochares (044) & 68.1 & ὅσοι μὴ ἐπεποίηντο φησίν ὅτε Σόλων εἰσῄει εἰς τὴν ἀρχήν, ἐξεῖναι αὐτοῖς διαθέσθαι ὅπως ἂν ἐθέλωσιν ὡς τοῖς γε ποιηθεῖσιν οὐκ ἐξὸν διαθέσθαι, ἀλλὰ ζῶντας ἐγκαταλιπόντας υἱὸν γνήσιον ἐπανιέναι, ἢ τελευτήσαντας ἀποδιδόναι τὴν κληρονομίαν τοῖς ἐξ ἀρχῆς οἰκείοις οὖσι τοῦ ποιησαμένου.\\
\addlinespace
Demosthenes (0014) & Against Leptines (020) & 12.6 & πῶς οὖν οὐ δεινόν, ὦ ἄνδρες Ἀθηναῖοι, εἰ τότε μὲν τοῖς ἠδικηκόσιν ὑμᾶς ὑπὲρ τοῦ μὴ ψεύσασθαι τὰ χρήματ’ εἰσφέρειν ἠθελήσατε, νῦν δ’ ἐξὸν ὑμῖν ἄνευ δαπάνης τὰ δίκαια ποιῆσαι τοῖς εὐεργέταις, λύσασι τὸν νόμον, ψεύδεσθαι μᾶλλον αἱρήσεσθε;\\
Demosthenes (0014) & Philip's Letter (012) & 18.1 & πάντων δέ μοι δοκεῖ παραλογώτατον εἶναι, διότι πέμψαντος ἐμοῦ πρέσβεις ἀπὸ τῆς συμμαχίας πάσης, ἵν’ ὦσι μάρτυρες, καὶ βουλομένου ποιήσασθαι πρὸς ὑμᾶς δικαίας ὁμολογίας ὑπὲρ τῶν Ἑλλήνων, οὐδὲ τοὺς περὶ τούτων λόγους ἐδέξασθε παρὰ τῶν πρεσβευόντων, ἐξὸν ὑμῖν ἢ τῶν κινδύνων ἀπαλλάξαι τοὺς δυσχερὲς ὑποπτεύοντάς τι καθ’ ἡμῶν, ἢ φανερῶς ἐξελέγξαι με φαυλότατον ὄντα τῶν ἁπάντων.\\
Demosthenes (0014) & Second Philippic (006) & 10.5 & εὑρίσκει γάρ, οἶμαι, καὶ ἀκούει τοὺς μὲν ὑμετέρους προγόνους, ἐξὸν αὐτοῖς τῶν λοιπῶν ἄρχειν Ἑλλήνων ὥστ’ αὐτοὺς ὑπακούειν βασιλεῖ, οὐ μόνον οὐκ ἀνασχομένους τὸν λόγον τοῦτον, ἡνίκ’ ἦλθεν Ἀλέξανδρος ὁ τούτων πρόγονος περὶ τούτων κῆρυξ, ἀλλὰ καὶ τὴν χώραν ἐκλιπεῖν προελομένους καὶ παθεῖν ὁτιοῦν ὑπομείναντας, καὶ μετὰ ταῦτα πράξαντας ταῦθ’ ἃ πάντες ἀεὶ γλίχονται λέγειν, ἀξίως δ’ οὐδεὶς εἰπεῖν δεδύνηται, διόπερ κἀγὼ παραλείψω, δικαίως (ἔστι γὰρ μείζω τἀκείνων ἔργα ἢ ὡς τῷ λόγῳ τις ἂν εἴποι) τοὺς δὲ Θηβαίων καὶ Ἀργείων προγόνους τοὺς μὲν συστρατεύσαντας τῷ βαρβάρῳ, τοὺς δ’ οὐκ ἐναντιωθέντας.\\
Demosthenes (0014) & Against Aphobus 1 (027) & 68.1 & δέομαι οὖν ὑμῶν, ὦ ἄνδρες δικασταί, καὶ ἱκετεύω καὶ ἀντιβολῶ, μνησθέντας καὶ τῶν νόμων καὶ τῶν ὅρκων οὓς ὀμόσαντες δικάζετε, βοηθῆσαι ἡμῖν τὰ δίκαια, καὶ μὴ περὶ πλείονος τὰς τούτου δεήσεις ἢ τὰς ἡμετέρας ποιήσασθαι.\\
Demosthenes (0014) & Against Eubulides (057) & 1.5 & δέομαι δ’ ἁπάντων ὑμῶν, ὦ ἄνδρες δικασταί, καὶ ἱκετεύω καὶ ἀντιβολῶ, λογισαμένους τό τε μέγεθος τοῦ παρόντος ἀγῶνος καὶ τὴν αἰσχύνην μεθ’ ἧς ὑπάρχει τοῖς ἁλισκομένοις ἀπολωλέναι, ἀκοῦσαι καὶ ἐμοῦ σιωπῇ, μάλιστα μέν, εἰ δυνατόν, μετὰ πλείονος εὐνοίας ἢ τούτου (τοῖς γὰρ ἐν κινδύνῳ καθεστηκόσιν εἰκὸς εὐνοϊκωτέρους ὑπάρχειν) εἰ δὲ μή, μετά γε τῆς ἴσης.\\
\addlinespace
Demosthenes (0014) & Against Leochares (044) & 3.1 & δέομαι δ’ ὑμῶν, ὦ ἄνδρες δικασταί, βοηθῆσαι τῷ τε πατρὶ τούτῳ καὶ ἐμοί, ἐὰν λέγωμεν τὰ δίκαια, καὶ μὴ περιιδεῖν πένητας ἀνθρώπους καὶ ἀσθενεῖς καταστασιασθέντας ὑπὸ παρατάξεως ἀδίκου.\\
Demosthenes (0014) & Against Olympiodorus (048) & 3.1 & δέομαι οὖν ὑμῶν, ὦ ἄνδρες δικασταί, ἀκούσαντας ἀμφοτέρων ἡμῶν καὶ αὐτοὺς δοκιμαστὰς τοῦ πράγματος γενομένους μάλιστα μὲν διαλλάξαντας ἀποπέμψαι καὶ εὐεργέτας ἡμῶν ἀμφοτέρων ὑμᾶς γενέσθαι, ἐὰν δ’ ἄρα μὴ ἐπιτυγχάνητε τούτου, ἐκ τῶν ὑπολοίπων τῷ τὰ δίκαια λέγοντι, τούτῳ τὴν ψῆφον ὑμᾶς προσθέσθαι.\\
Demosthenes (0014) & Apollodorus Against Callipus (052) & 32.3 & δέομαι δὲ ὑμῶν αὐτός τε ὑπὲρ ἐμαυτοῦ καὶ ὑπὲρ τοῦ πατρός, ἀναμνησθέντας ὅτι πάντων μὲν ὑμῖν καὶ μάρτυρας καὶ τεκμήρια καὶ νόμους καὶ πίστεις παρεσχόμην ὧν εἴρηκα, τούτῳ δὲ ἐπιδείκνυμι ὅτι ἐξόν, εἴπερ τι αὐτῷ προσῆκεν τοῦ ἀργυρίου, ἐπὶ τὸν Κηφισιάδην βαδίζειν τὸν ὁμολογοῦντα κεκομίσθαι καὶ ἔχειν τὸ ἀργύριον, καὶ ταῦτα μηδὲν ἧττον τὰ πιστὰ παρ’ ἡμῶν λαβόντα, οὐκ ἔρχεται, εἰδὼς ὅτι οὐκ ἔστιν παρ’ ἡμῖν τὸ ἀργύριον, δέομαι ὑμῶν ἀποψηφίσασθαί μου.\\
Demosthenes (0014) & Apollodorus Against Polycles (050) & 64.1 & ἀνθ’ ὧν ἁπάντων νῦν ὑμῶν δέομαι, ὥσπερ ἐγὼ ὑμῖν εὔτακτον καὶ χρήσιμον ἐμαυτὸν παρέσχον, οὕτω καὶ ὑμᾶς νυνὶ περὶ ἐμοῦ πρόνοιαν ποιησαμένους, καὶ ἀναμνησθέντας ἁπάντων ὧν τε διηγησάμην πρὸς ὑμᾶς, τῶν τε μαρτυριῶν ὧν παρεσχόμην καὶ τῶν ψηφισμάτων, βοηθῆσαι μὲν ἐμοὶ ἀδικουμένῳ, τιμωρήσασθαι δ’ ὑπὲρ ὑμῶν αὐτῶν εἰσπρᾶξαι δὲ τὰ ὑπὲρ τούτου ἀνηλωμένα.\\
Demosthenes (0014) & Apollodorus Against Timotheus (049) & 24.3 & μέλλων τοίνυν ἀποδημεῖν ὡς βασιλέα, καὶ διαπραξάμενος ἐκπλεῦσαι ὡς βασιλεῖ στρατηγήσων τὸν ἐπ’ Αἴγυπτον πόλεμον, ἵνα μὴ δῷ ἐνθάδε λόγον καὶ εὐθύνας τῆς στρατηγίας, μεταπεμψάμενος τὸν πατέρα τὸν ἐμὸν εἰς τὸ Παράλιον τῶν τε προϋπηργμένων εἰς αὑτὸν ἐπῄνει, καὶ ἐδεῖτο αὐτοῦ συστήσας Φιλώνδαν, ἄνδρα τὸ μὲν γένος Μεγαρέα, μετοικοῦντα δ’ Ἀθήνησιν, πιστῶς δὲ τούτῳ διακείμενον καὶ ὑπηρετοῦντα ἐν ἐκείνῳ τῷ χρόνῳ, ἐπειδὰν ἀφίκηται ἐκ Μακεδονίας ὁ Φιλώνδας, ὃν συνίστη οὗτος τῷ πατρὶ τῷ ἐμῷ, ἄγων ξύλα τὰ δοθέντα τούτῳ ὑπὸ Ἀμύντου, τὸ ναῦλον τῶν ξύλων παρασχεῖν καὶ ἐᾶσαι ἀνακομίσαι τὰ ξύλα εἰς τὴν οἰκίαν τὴν ἑαυτοῦ τὴν ἐν Πειραιεῖ·\\
\addlinespace
Demosthenes (0014) & On the False Embassy (019) & 1.4 & δεήσομαι δὲ πάντων ὑμῶν, ἃ καὶ τοῖς μὴ δεηθεῖσι δίκαιόν ἐστιν ὑπάρχειν, μηδεμίαν μήτε χάριν μήτ’ ἄνδρα ποιεῖσθαι περὶ πλείονος ἢ τὸ δίκαιον καὶ τὸν ὅρκον ὃν εἰσελήλυθεν ὑμῶν ἕκαστος ὀμωμοκώς, ἐνθυμουμένους ὅτι ταῦτα μέν ἐσθ’ ὑπὲρ ὑμῶν καὶ ὅλης τῆς πόλεως, αἱ δὲ τῶν παρακλήτων αὗται δεήσεις καὶ σπουδαὶ τῶν ἰδίων πλεονεξιῶν εἵνεκα γίγνονται, ἃς ἵνα κωλύηθ’ οἱ νόμοι συνήγαγον ὑμᾶς, οὐχ ἵνα κυρίας τοῖς ἀδικοῦσι ποιῆτε.\\
Demosthenes (0014) & Against Olympiodorus (048) & 51.2 & ἐγὼ γὰρ αὐτῷ ἐπήγγειλα ἥκειν ἔχοντι τὰς συνθήκας.\\
Demosthenes (0014) & Against Aphobus 1 (027) & 68.1 & δέομαι οὖν ὑμῶν, ὦ ἄνδρες δικασταί, καὶ ἱκετεύω καὶ ἀντιβολῶ, μνησθέντας καὶ τῶν νόμων καὶ τῶν ὅρκων οὓς ὀμόσαντες δικάζετε, βοηθῆσαι ἡμῖν τὰ δίκαια, καὶ μὴ περὶ πλείονος τὰς τούτου δεήσεις ἢ τὰς ἡμετέρας ποιήσασθαι.\\
Demosthenes (0014) & Against Eubulides (057) & 1.5 & δέομαι δ’ ἁπάντων ὑμῶν, ὦ ἄνδρες δικασταί, καὶ ἱκετεύω καὶ ἀντιβολῶ, λογισαμένους τό τε μέγεθος τοῦ παρόντος ἀγῶνος καὶ τὴν αἰσχύνην μεθ’ ἧς ὑπάρχει τοῖς ἁλισκομένοις ἀπολωλέναι, ἀκοῦσαι καὶ ἐμοῦ σιωπῇ, μάλιστα μέν, εἰ δυνατόν, μετὰ πλείονος εὐνοίας ἢ τούτου (τοῖς γὰρ ἐν κινδύνῳ καθεστηκόσιν εἰκὸς εὐνοϊκωτέρους ὑπάρχειν) εἰ δὲ μή, μετά γε τῆς ἴσης.\\
Demosthenes (0014) & Against Conon (054) & 1.5 & πάντων δὲ τῶν φίλων καὶ τῶν οἰκείων, οἷς συνεβουλευόμην, ἔνοχον μὲν φασκόντων αὐτὸν ἐκ τῶν πεπραγμένων εἶναι καὶ τῇ τῶν λωποδυτῶν ἀπαγωγῇ καὶ ταῖς τῆς ὕβρεως γραφαῖς, συμβουλευόντων δέ μοι καὶ παραινούντων μὴ μείζω πράγματ’ ἢ δυνήσομαι φέρειν ἐπάγεσθαι, μηδ’ ὑπὲρ τὴν ἡλικίαν περὶ ὧν ἐπεπόνθειν ἐγκαλοῦντα φαίνεσθαι, οὕτως ἐποίησα καὶ δι’ ἐκείνους ἰδίαν ἔλαχον δίκην, ἥδιστ’ ἄν, ὦ ἄνδρες Ἀθηναῖοι, θανάτου κρίνας τουτονί.\\
\addlinespace
Dinarchus (0029) & Against Demosthenes (004) & 71.5 & σὲ δὲ τὴν μὲν πατρῴαν γῆν πεπρακέναι, τοὺς δ’ οὐ γεγενημένους υἱεῖς σαυτῷ προσποιεῖσθαι παρὰ τοὺς νόμους τῶν ἐν ταῖς κρίσεσιν ἕνεκα γιγνομένων ὅρκων, ἐπιτάττειν δὲ τοῖς ἄλλοις στρατεύεσθαι λιπόντ’ αὐτὸν τὴν κοινὴν τάξιν.\\
Euripides (0006) & Orestes (049) & 46 & ἔδοξε δ’ Ἄργει τῷδε μήθ’ ἡμᾶς στέγαις, μὴ πυρὶ δέχεσθαι, μήτε προσφωνεῖν τινα μητροκτονοῦντας·\\
Euripides (0006) & Hecuba (040) & 301 & ἃ δ’ εἶπον εἰς ἅπαντας οὐκ ἀρνήσομαι, Τροίας ἁλούσης ἀνδρὶ τῷ πρώτῳ στρατοῦ σὴν παῖδα δοῦναι σφάγιον ἐξαιτουμένῳ.\\
Euripides (0006) & Hecuba (040) & 538 & πρευμενὴς δ’ ἡμῖν γενοῦ λῦσαί τε πρύμνας καὶ χαλινωτήρια νεῶν δὸς ἡμῖν πρευμενοῦς τ’ ἀπ’ Ἰλίου νόστου τυχόντας πάντας ἐς πάτραν μολεῖν.\\
Herodotus (0016) & Histories (001) & 9.60.7 & νῦν ὦν δέδοκται τὸ ἐνθεῦτεν τὸ ποιητέον ἡμῖν, ἀμυνομένους γὰρ τῇ δυνάμεθα ἄριστα περιστέλλειν ἀλλήλους.\\
\addlinespace
Herodotus (0016) & Histories (001) & 1.3.5 & οὕτω δὴ ἁρπάσαντος αὐτοῦ Ἑλένην, τοῖσι Ἕλλησι δόξαι πρῶτὸν πέμψαντας ἀγγέλους ἀπαιτέειν τε Ἑλένην καὶ δίκας τῆς ἁρπαγῆς αἰτέειν.\\
Herodotus (0016) & Histories (001) & 1.19.5 & μακροτέρης δέ οἱ γινομένης τῆς νούσου πέμπει ἐς Δελφοὺς θεοπρόπους, εἴτε δὴ συμβουλεύσαντός τευ, εἴτε καὶ αὐτῷ ἔδοξε πέμψαντα τὸν θεὸν ἐπειρέσθαι περὶ τῆς νούσου.\\
Herodotus (0016) & Histories (001) & 1.207.23 & νῦν ὦν μοι δοκέει διαβάντας προελθεῖν ὅσον ἂν ἐκεῖνοι ὑπεξίωσι, ἐνθεῦτεν δὲ τάδε ποιεῦντας πειρᾶσθαι ἐκείνων περιγενέσθαι.\\
Herodotus (0016) & Histories (001) & 2.151.10 & οἳ δὲ ἐν φρενὶ λαβόντες τό τε ποιηθὲν ἐκ Ψαμμητίχου καὶ τὸ χρηστήριον, ὅτι ἐκέχρηστό σφι τὸν χαλκέῃ σπείσαντα αὐτῶν φιάλῃ τοῦτον βασιλέα ἔσεσθαι μοῦνον Αἰγύπτου, ἀναμνησθέντες τοῦ χρησμοῦ κτεῖναι μὲν οὐκ ἐδικαίωσαν Ψαμμήτιχον, ὡς ἀνεύρισκον βασανίζοντες ἐξ οὐδεμιῆς προνοίης αὐτὸν ποιήσαντα, ἐς δὲ τὰ ἕλεα ἔδοξέ σφι διῶξαι ψιλώσαντας τὰ πλεῖστα τῆς δυνάμιος, ἐκ δὲ τῶν ἑλέων ὁρμώμενον μὴ ἐπιμίσγεσθαι τῇ ἄλλῃ Αἰγύπτῳ.\\
Herodotus (0016) & Histories (001) & 3.62.17 & νῦν ὦν μοι δοκέει μεταδιώξαντας τὸν κήρυκα ἐξετάζειν εἰρωτεῦντας παρ’ ὅτευ ἥκων προαγορεύει ἡμῖν Σμέρδιος βασιλέος ἀκούειν. ʺ\\
\addlinespace
Herodotus (0016) & Histories (001) & 4.3.12 & νῦν ὦν μοι δοκέει αἰχμὰς μὲν καὶ τόξα μετεῖναι, λαβόντα δὲ ἕκαστον τοῦ ἵππου τὴν μάστιγα ἰέναι ἆσσον αὐτῶν.\\
Herodotus (0016) & Histories (001) & 4.11.19 & ὡς δὲ δόξαι σφι ταῦτα, διαστάντας καὶ ἀριθμὸν ἴσους γενομένους μάχεσθαι πρὸς ἀλλήλους.\\
Herodotus (0016) & Histories (001) & 4.81.19 & κομισθῆναι τε δὴ χρῆμα πολλὸν ἀρδίων καί οἱ δόξαι ἐξ αὐτέων μνημόσυνον ποιήσαντι λιπέσθαι.\\
Herodotus (0016) & Histories (001) & 5.92,gamma.19 & ἀποδόντες ὦν ὀπίσω τῇ τεκούσῃ τὸ παιδίον καὶ ἐξελθόντες ἔξω, ἑστεῶτες ἐπὶ τῶν θυρέων ἀλλήλων ἅπτοντο καταιτιώμενοι, καὶ μάλιστα τοῦ πρώτου λαβόντος, ὅτι οὐκ ἐποίησε κατὰ τὰ δεδογμένα, ἐς ὃ δή σφι χρόνου ἐγγινομένου ἔδοξε αὖτις παρελθόντας πάντας τοῦ φόνου μετίσχειν.\\
Herodotus (0016) & Histories (001) & 5.96.10 & οὐκ ἐνδεκομένοισι δέ σφι ἐδέδοκτο ἐκ τοῦ φανεροῦ τοῖσι Πέρσῃσι πολεμίους εἶναι.\\
\addlinespace
Herodotus (0016) & Histories (001) & 6.22.1 & Σαμίων δὲ τοῖσί τι ἔχουσι τὸ μὲν ἐς τοὺς Μήδους ἐκ τῶν στρατηγῶν τῶν σφετέρων ποιηθὲν οὐδαμῶς ἤρεσκε, ἐδόκεε δὲ μετὰ τὴν ναυμαχίην αὐτίκα βουλευομένοισι, πρὶν ἤ σφι ἐς τὴν χώρην ἀπικέσθαι τὸν τύραννον Αἰάκεα, ἐς ἀποικίην ἐκπλέειν μηδὲ μένοντας Μήδοισί τε καὶ Αἰάκεϊ δουλεύειν.\\
Herodotus (0016) & Histories (001) & 6.82.14 & ταῦτά τε ὦν ἐπιλεγομένῳ καὶ βουλευομένῳ ἔδοξέ μοι τὰ ἡμίσεα πάσης τῆς οὐσίης ἐξαργυρώσαντα θέσθαι παρὰ σέ, εὖ ἐξεπισταμένῳ ὥς μοι κείμενα ἔσται παρὰ σοὶ σόα.\\
Herodotus (0016) & Histories (001) & 9.12.1 & ἐς τοῦτον δὴ τὸν χῶρον καὶ ἐπὶ τὴν κρήνην τὴν Γαργαφίην τὴν ἐν τῷ χώρῳ τούτῳ ἐοῦσαν ἔδοξέ σφι χρεὸν εἶναι ἀπικέσθαι καὶ διαταχθέντας στρατοπεδεύεσθαι.\\
Herodotus (0016) & Histories (001) & 9.87.2 & ἄνδρες Θηβαῖοι, ἐπειδὴ οὕτω δέδοκται τοῖσι Ἕλλησι, μὴ πρότερον ἀπαναστῆναι πολιορκέοντας ἢ ἐξέλωσι Θήβας ἢ ἡμέας αὐτοῖσι παραδῶτε, νῦν ὦν ἡμέων εἵνεκα γῆ ἡ Βοιωτίη πλέω μὴ ἀναπλήσῃ, ἀλλ’ εἰ μὲν χρημάτων χρηίζοντες πρόσχημα ἡμέας ἐξαιτέονται, χρήματά σφι δῶμεν ἐκ τοῦ κοινοῦ (σὺν γὰρ τῷ κοινῷ καὶ ἐμηδίσαμεν οὐδὲ μοῦνοι ἡμεῖς), εἰ δὲ ἡμέων ἀληθέως δεόμενοι πολιορκέουσι, ἡμεῖς ἡμέας αὐτοὺς ἐς ἀντιλογίην παρέξομεν. ʺ\\
Herodotus (0016) & Histories (001) & 9.45.14 & πρὸς ταῦτα Πελοποννησίων μὲν τοῖσι ἐν τέλεϊ ἐοῦσι ἐδόκεε τῶν μηδισάντων ἐθνέων τῶν Ἑλληνικῶν τὰ ἐμπολαῖα ἐξαναστήσαντας δοῦναι τὴν χώρην Ἴωσι ἐνοικῆσαι, Ἀθηναίοισι δὲ οὐκ ἐδόκεε ἀρχὴν Ἰωνίην γενέσθαι ἀνάστατον οὐδὲ Πελοποννησίοισι περὶ τῶν σφετερέων ἀποικιέων βουλεύειν·\\
\addlinespace
Herodotus (0016) & Histories (001) & 9.36.2 & τοῖσι μέν νυν ἀμφὶ Λευτυχίδην Πελοποννησίοις ἔδοξε ἀποπλέειν ἐς τὴν Ἑλλάδα, Ἀθηναίοισι δὲ καὶ Ξανθίππῳ τῷ στρατηγῷ αὐτοῦ ὑπομείναντας πειρᾶσθαι τῆς Χερσονήσου.\\
Herodotus (0016) & Histories (001) & 3.142.2 & τῷ δικαιοτάτῳ ἀνδρῶν βουλομένῳ γενέσθαι οὐκ ἐξεγένετο.\\
Herodotus (0016) & Histories (001) & 1.54.5 & Δελφοὶ δὲ ἀντὶ τούτων ἔδοσαν Κροίσῳ καὶ Λυδοῖσι προμαντηίην καὶ ἀτελείην καὶ προεδρίην, καὶ ἐξεῖναι τῷ βουλομένῳ αὐτῶν γίνεσθαι Δελφὸν ἐς τὸν αἰεὶ χρόνον.\\
Herodotus (0016) & Histories (001) & 2.120.16 & οὐ μὲν οὐδὲ ἡ βασιληίη ἐς Ἀλέξανδρον περιήιε, ὥστε γέροντος Πριάμου ἐόντος ἐπ’ ἐκείνῳ τὰ πρήγματα εἶναι, ἀλλὰ Ἕκτωρ καὶ πρεσβύτερος καὶ ἀνὴρ ἐκείνου μᾶλλον ἐὼν ἔμελλε αὐτὴν Πριάμου ἀποθανόντος παραλάμψεσθαι, τὸν οὐ προσῆκε ἀδικέοντι τῷ ἀδελφεῷ ἐπιτρέπειν, καὶ ταῦτα μεγάλων κακῶν δι’ αὐτὸν συμβαινόντων ἰδίῃ τε αὐτῷ καὶ τοῖσι ἄλλοισι πᾶσι Τρωσί.\\
Herodotus (0016) & Histories (001) & 1.81.3 & οἱ μὲν γὰρ πρότεροι διεπέμποντο ἐς πέμπτον μῆνα προερέοντες συλλέγεσθαι ἐς Σάρδις, τούτους δὲ ἐξέπεμπε τὴν ταχίστην δέεσθαι βοηθέειν ὡς πολιορκεομένου Κροίσου.\\
\addlinespace
Herodotus (0016) & Histories (001) & 5.80.4 & τουτέων ἀδελφεῶν ἐουσέων, δοκέω ἡμῖν Αἰγινητέων δέεσθαι τὸν θεὸν χρῇσαι τιμωρητήρων γενέσθαι. ʺ\\
Herodotus (0016) & Histories (001) & 6.100.1 & Ἐρετριέες δὲ πυνθανόμενοι τὴν στρατιὴν τὴν Περσικὴν ἐπὶ σφέας ἐπιπλέουσαν Ἀθηναίων ἐδεήθησαν σφίσι βοηθοὺς γενέσθαι.\\
Herodotus (0016) & Histories (001) & 1.46.15 & ἐντειλάμενος δὲ τοῖσι Λυδοῖσι τάδε ἀπέπεμπε ἐς τὴν διάπειραν τῶν χρηστηρίων, ἀπ’ ἧς ἂν ἡμέρης ὁρμηθέωσι ἐκ Σαρδίων, ἀπὸ ταύτης ἡμερολογέοντας τὸν λοιπὸν χρόνον ἑκατοστῇ ἡμέρῃ χρᾶσθαι τοῖσι χρηστηρίοισι, ἐπειρωτῶντας ὅ τι ποιέων τυγχάνοι ὁ Λυδῶν βασιλεὺς Κροῖσος ὁ Ἀλυάττεω·\\
Herodotus (0016) & Histories (001) & 1.117.15 & παραδίδωμι μέντοι τῷδε κατὰ τάδε ἐντειλάμενος, θεῖναὶ μιν ἐς ἔρημον ὄρος καὶ παραμένοντα φυλάσσειν ἄχρι οὗ τελευτήσῃ, ἀπειλήσας παντοῖα τῷδε ἢν μὴ τάδε ἐπιτελέα ποιήσῃ.\\
Herodotus (0016) & Histories (001) & 1.123.16 & ἀπορράψας δὲ τοῦ λαγοῦ τὴν γαστέρα, καὶ δίκτυα δοὺς ἅτε θηρευτῇ τῶν οἰκετέων τῷ πιστοτάτῳ, ἀπέστελλε ἐς τοὺς Πέρσας, ἐντειλάμενὸς οἱ ἀπὸ γλώσσης διδόντα τὸν λαγὸν Κύρῳ ἐπειπεῖν αὐτοχειρίῃ μιν διελεῖν καὶ μηδένα οἱ ταῦτα ποιεῦντι παρεῖναι.\\
\addlinespace
Herodotus (0016) & Histories (001) & 2.121,gamma.6 & τοῦ φωρὸς τὸν νέκυν κατὰ τοῦ τείχεος κατακρεμάσαι, φυλάκους δὲ αὐτοῦ καταστήσαντα ἐντείλασθαί σφι, τὸν ἂν ἴδωνται ἀποκλαύσαντα ἢ κατοικτισάμενον, συλλαβόντας ἄγειν πρὸς ἑωυτόν.\\
Herodotus (0016) & Histories (001) & 2.162.1 & πυθόμενος δὲ ταῦτα ὁ Ἀπρίης ἔπεμπε ἐπ’ Ἄμασιν ἄνδρα δόκιμον τῶν περὶ ἑωυτὸν Αἰγυπτίων, τῷ οὔνομα ἦν Πατάρβημις, ἐντειλάμενος αὐτῷ ζῶντα Ἄμασιν ἀγαγεῖν παρ’ ἑωυτόν.\\
Herodotus (0016) & Histories (001) & 3.25.9 & ἐπείτε δὲ στρατευόμενος ἐγένετο ἐν Θήβῃσι, ἀπέκρινε τοῦ στρατοῦ ὡς πέντε μυριάδας, καὶ τούτοισι μὲν ἐνετέλλετο Ἀμμωνίους ἐξανδραποδισαμένους τὸ χρηστήριον τὸ τοῦ Διὸς ἐμπρῆσαι, αὐτὸς δὲ τὸν λοιπὸν ἄγων στρατὸν ἤιε ἐπὶ τοὺς Αἰθίοπας.\\
Herodotus (0016) & Histories (001) & 3.36.20 & ὁ δὲ ἐπείτε τοξεῦσαι οὐκ εἶχε, ἐνετείλατο τοῖσι θεράπουσι λαβόντας μιν ἀποκτεῖναι.\\
Herodotus (0016) & Histories (001) & 3.135.2 & ἐπείτε γὰρ τάχιστα ἡμέρη ἐπέλαμψε, καλέσας Περσέων ἄνδρας δοκίμους πεντεκαίδεκα ἐνετέλλετό σφι ἑπομένους Δημοκήδεϊ διεξελθεῖν τὰ παραθαλάσσια τῆς Ἑλλάδος, ὅκως τε μὴ διαδρήσεται σφέας ὁ Δημοκήδης, ἀλλά μιν πάντως ὀπίσω ἀπάξουσι.\\
\addlinespace
Herodotus (0016) & Histories (001) & 4.133.5 & πυνθανόμεθα γὰρ Δαρεῖον ἐντείλασθαι ὑμῖν ἑξήκοντα ἡμέρας μούνας φρουρήσαντας τὴν γέφυραν, αὐτοῦ μὴ παραγενομένου ἐν τούτῳ τῷ χρόνῳ, ἀπαλλάσσεσθαι ἐς τὴν ὑμετέρην.\\
Herodotus (0016) & Histories (001) & 5.12.1 & τελεωθέντων δὲ ἀμφοτέροισι, οὗτοι μὲν κατὰ τὰ εἵλοντο ἐτράποντο, Δαρεῖον δὲ συνήνεικε πρῆγμα τοιόνδε ἰδόμενον ἐπιθυμῆσαι ἐντείλασθαι Μεγαβάζῳ Παίονας ἑλόντα ἀνασπάστους ποιῆσαι ἐς τὴν Ἀσίην ἐκ τῆς Εὐρώπης.\\
Herodotus (0016) & Histories (001) & 3.73.6 & ὅσοι τε ὑμέων Καμβύσῃ νοσέοντι παρεγένοντο, πάντως κου μέμνησθε τὰ ἐπέσκηψε Πέρσῃσι τελευτῶν τὸν βίον μὴ πειρωμένοισι ἀνακτᾶσθαι τὴν ἀρχήν·\\
Herodotus (0016) & Histories (001) & 4.33.18 & ἐπεὶ δὲ τοῖσι Ὑπερβορέοισι τοὺς ἀποπεμφθέντας ὀπίσω οὐκ ἀπονοστέειν, δεινὰ ποιευμένους εἰ σφέας αἰεὶ καταλάμψεται ἀποστέλλοντας μὴ ἀποδέκεσθαι, οὕτω δὴ φέροντας ἐς τοὺς οὔρους τὰ ἱρὰ ἐνδεδεμένα ἐν πυρῶν καλάμῃ τοὺς πλησιοχώρους ἐπισκήπτειν κελεύοντας προπέμπειν σφέα ἀπὸ ἑωυτῶν ἐς ἄλλο ἔθνος.\\
Herodotus (0016) & Histories (001) & 4.89.3 & Δαρεῖος δὲ δωρησάμενος Μανδροκλέα διέβαινε ἐς τὴν Εὐρώπην, τοῖσι Ἴωσι παραγγείλας πλέειν ἐς τὸν Πόντον μέχρι Ἴστρου ποταμοῦ, ἐπεὰν δὲ ἀπίκωνται ἐς τὸν Ἴστρον, ἐνθαῦτα αὐτὸν περιμένειν ζευγνύντας τὸν ποταμόν.\\
\addlinespace
Herodotus (0016) & Histories (001) & 6.50.12 & μαθὼν δὲ ὁ Κλεομένης ποιεῦντας τοὺς Ἀργείους ὁκοῖόν τι ὁ σφέτερος κῆρυξ σημήνειε, παραγγέλλει σφι, ὅταν σημήνῃ ὁ κῆρυξ ποιέεσθαι ἄριστον, τότε ἀναλαβόντας τὰ ὅπλα χωρέειν ἐς τοὺς Ἀργείους.\\
Herodotus (0016) & Histories (001) & 9.7.17 & καὶ οἳ μὲν περὶ τὸ Ἥραιον ἐστρατοπεδεύοντο, Παυσανίης δὲ ὁρῶν σφεας ἀπαλλασσομένους ἐκ τοῦ στρατοπέδου παρήγγελλε καὶ τοῖσι Λακεδαιμονίοισι ἀναλαβόντας τὰ ὅπλα ἰέναι κατὰ τοὺς ἄλλους τοὺς προϊόντας, νομίσας αὐτοὺς ἐς τὸν χῶρον ἰέναι ἐς τὸν συνεθήκαντο.\\
Herodotus (0016) & Histories (001) & 3.53.1 & ἐπεὶ δὲ τοῦ χρόνου προβαίνοντος ὅ τε Περίανδρος παρηβήκεε καὶ συνεγινώσκετο ἑωυτῷ οὐκέτι εἶναι δυνατὸς τὰ πρήγματα ἐπορᾶν τε καὶ διέπειν, πέμψας ἐς τὴν Κέρκυραν ἀπεκάλεε τὸν Λυκόφρονα ἐπὶ τὴν τυραννίδα·\\
Herodotus (0016) & Histories (001) & 2.107.8 & τὸν δὲ ὡς μαθεῖν τοῦτο, αὐτίκα συμβουλεύεσθαι τῇ γυναικί·\\
Herodotus (0016) & Histories (001) & 3.21.14 & βασιλεὺς ὁ Αἰθιόπων συμβουλεύει τῷ Περσέων βασιλέι, ἐπεὰν οὕτω εὐπετέως ἕλκωσι τὰ τόξα Πέρσαι ἐόντα μεγάθεϊ τοσαῦτα, τότε ἐπ’ Αἰθίοπας τοὺς μακροβίους πλήθεϊ ὑπερβαλλόμενον στρατεύεσθαι·\\
\addlinespace
Herodotus (0016) & Histories (001) & 6.108.13 & συμβουλεύομεν δὲ ὑμῖν δοῦναι ὑμέας αὐτοὺς Ἀθηναίοισι, πλησιοχώροισι τε ἀνδράσι καὶ τιμωρέειν ἐοῦσι οὐ κακοῖσι. ʺ\\
Herodotus (0016) & Histories (001) & 7.120.1 & ἔνθα δὴ Μεγακρέοντος ἀνδρὸς Ἀβδηρίτεω ἔπος εὖ εἰρημένον ἐγένετο, ὃς συνεβούλευσε Ἀβδηρίτῃσι πανδημεί, αὐτοὺς καὶ γυναῖκας, ἐλθόντας ἐς τὰ σφέτερα ἱρὰ ἵζεσθαι ἱκέτας τῶν θεῶν παραιτεομένους καὶ τὸ λοιπόν σφι ἀπαμύνειν τῶν ἐπιόντων κακῶν τὰ ἡμίσεα, τῶν τε παροιχομένων ἔχειν σφι μεγάλην χάριν, ὅτι βασιλεὺς Ξέρξης οὐ δὶς ἑκάστης ἡμέρης ἐνόμισε σῖτον αἱρέεσθαι·\\
Herodotus (0016) & Histories (001) & 7.141.2 & προβάλλουσι δὲ σφέας αὐτοὺς ὑπὸ τοῦ κακοῦ τοῦ κεχρησμένου, Τίμων ὁ Ἀνδροβούλου, τῶν Δελφῶν ἀνὴρ δόκιμος ὅμοια τῷ μάλιστα, συνεβούλευέ σφι ἱκετηρίην λαβοῦσι δεύτερα αὖτις ἐλθόντας χρᾶσθαι τῷ χρηστηρίῳ ὡς ἱκέτας.\\
Herodotus (0016) & Histories (001) & 7.173.15 & ἀπικόμενοι γὰρ ἄγγελοι παρὰ Ἀλεξάνδρου τοῦ Ἀμύντεω ἀνδρὸς Μακεδόνος συνεβούλευόν σφι ἀπαλλάσσεσθαι μηδὲ μένοντας ἐν τῇ ἐσβολῇ καταπατηθῆναι ὑπὸ τοῦ στρατοῦ τοῦ ἐπιόντος, σημαίνοντες τὸ πλῆθός τε τῆς στρατιῆς καὶ τὰς νέας.\\
Herodotus (0016) & Histories (001) & 7.53.3 & ὦ Πέρσαι, τῶνδ’ ἐγὼ ὑμέων χρηίζων συνέλεξα, ἄνδρας τε γενέσθαι ἀγαθοὺς καὶ μὴ καταισχύνειν τὰ πρόσθε ἐργασμένα Πέρσῃσι, ἐόντα μεγάλα τε καὶ πολλοῦ ἄξια, ἀλλ’ εἷς τε ἕκαστος καὶ οἱ σύμπαντες προθυμίην ἔχωμεν·\\
\addlinespace
Isaeus (0017) & On the Estate of Dicaeogenes (005) & 18.9 & ἡμεῖς τοίνυν ταῦτα παθόντες ὑπὸ Λεωχάρους, καὶ ἐγγενόμενον ἡμῖν αὐτὸν ἐπειδὴ εἵλομεν τῶν ψευδομαρτυρίων ἀτιμῶσαι, οὐκ ἐβουλήθημεν, ἀλλ’ ἐξήρκεσε τὰ ἡμέτερα ἡμῖν κομισαμένοις ἀπηλλάχθαι.\\
Isaeus (0017) & On The Estate Of Aristarchus (010) & 12.4 & θαυμαστὸν γὰρ ἂν ἦν, εἰ τὴν ἐμὴν μητέρα ἔχοντι Ἀπολλοδώρῳ ἢ Ἀριστομένει οὐκ ἂν οἷόν τε ἦν τῶν ἐκείνης κυρίῳ γενέσθαι, κατὰ τὸν νόμον ὃς οὐκ ἐᾷ τῶν τῆς ἐπικλήρου κύριον εἶναι, ἀλλ’ ἢ τοὺς παῖδας ἐπὶ δίετες ἡβήσαντας κρατεῖν τῶν χρημάτων, ἀλλ’ ἑτέρῳ αὐτὴν ἐκδόντι ἐξέσται εἰς τὰ ταύτης χρήματα ὑὸν εἰσποιῆσαι.\\
Isaeus (0017) & On The Estate Of Pyrrhus (003) & 20.4 & παρὰ δὲ τῶν ἀσθενούντων ἢ τῶν ἀποδημεῖν μελλόντων ὅταν τις ἐκμαρτυρίαν ποιῆται, τοὺς ἐπιεικεστάτους τῶν πολιτῶν καὶ τοὺς ἡμῖν γνωριμωτάτους ἕκαστος ἡμῶν παρακαλεῖ μάλιστα, καὶ οὐ μεθ’ ἑνὸς οὐδὲ μετὰ δυοῖν, ἀλλ’ ὡς ἂν μετὰ πλείστων δυνώμεθα τὰς ἐκμαρτυρίας πάντες ποιούμεθα, ἵνα τῷ τε ἐκμαρτυρήσαντι μὴ ἐξείη ὕστερον ἐξάρνῳ γενέσθαι τὴν μαρτυρίαν, ὑμεῖς τε πολλοῖς καὶ καλοῖς κἀγαθοῖς ταὐτὰ μαρτυροῦσι πιστεύοιτε μᾶλλον.\\
Isaeus (0017) & On The Estate of Apollodorus (007) & 24.3 & καὶ τῶνδε ἐξῆν αὐτῷ κατὰ ταύτην τὴν συγγένειαν λαγχάνειν, ὄντι προτέρῳ ταύτης, εἴπερ τὰ πεπραγμένα μὴ κυρίως ἔχειν ἐνόμιζεν.\\
Isaeus (0017) & On The Estate Of Pyrrhus (003) & 58.1 & οὐκοῦν δυοῖν τὰ ἕτερα προσῆκε τῇ γυναικί, ἢ γυναικί ἢ ζῶντι τῷ Ἐνδίῳ ἀμφισβητῆσαι τῶν πατρῴων, ἢ ἐπειδὴ τετελευτηκὼς ἦν ὁ εἰσποίητος, τῶν τοῦ ἀδελφοῦ τὴν ἐπιδικασίαν ἀξιοῦν ποιεῖσθαι, ἄλλως τε καὶ εἰ, ὥς φασιν οὗτοι, ἠγγυήκει αὐτὴν τῷ Ξενοκλεῖ ὡς γνησίαν ἀδελφὴν οὖσαν αὑτοῦ.\\
\addlinespace
Isaeus (0017) & On The Estate of Ciron (008) & 45.5 & ἔχετε δὲ πίστεις ἱκανὰς ἐκ μαρτυριῶν, ἐκ βασάνων, ἐξ αὐτῶν τῶν νόμων, ὅτι τ’ ἐσμὲν λτἐκγτ θυγατρὸς γνησίας Κίρωνος, καὶ ὅτι προσήκει ἡμῖν μᾶλλον ἢ τούτοις κληρονομεῖν τῶν ἐκείνου χρημάτων, ἐκγόνοις οὖσι τοῦ πάππου.\\
Isaeus (0017) & On The Estate of Apollodorus (007) & 4.5 & δέομαι δὲ ὑμῶν, ὦ ἄνδρες, πάντων ὁμοίως εὔνοιάν τέ μοι παρασχεῖν, κἂν ἐπὶ τὸν κλῆρον ἀναιδῶς αὐτοὺς ἰόντας ἐξελέγχω, βοηθεῖν μοι τὰ δίκαια.\\
Isocrates (0010) & Against Euthynus (001) & 9.6 & ἐπεὶ ἔμοιγε δοκεῖ, εἰδότι τὴν τούτων οἰκειότητα, οὐδ’ ἂν Εὐθύνους Νικίαν ἀδικῆσαι, εἰ ἐξῆν ἄλλον τινὰ τοσαῦτα χρήματα ἀποστερῆσαι.\\
Isocrates (0010) & Panathenaicus (021) & 200.1 & ἐπειδὴ δὲ διεξιοῦσιν ἡμῖν ἐδόκει καλῶς ἔχειν καὶ προσδεῖσθαι τελευτῆς μόνον, ἔδοξέ μοι μεταπέμψασθαί τινα τῶν ἐμοὶ μὲν πεπλησιακότων, ἐν ὀλιγαρχίᾳ δὲ πεπολιτευμένον, προῃρημένον δὲ Λακεδαιμονίους ἐπαινεῖν, ἵν’ εἴ τι παρέλαθεν ἡμᾶς ψεῦδος εἰρημένον, ἐκεῖνος κατιδὼν δηλώσειεν ἡμῖν.\\
Isocrates (0010) & Panathenaicus (021) & 233.1 & ἐν τοιαύτῃ δὲ μοι ταραχῇ καθεστηκότι καὶ μεταβολὰς ποιουμένῳ πολλὰς ἔδοξε κράτιστον εἶναι παρακαλέσαντι τῶν πεπλησιακότων τοὺς ἐπιδημοῦντας βουλεύσασθαι μετ’ αὐτῶν, πότεροι ἀφανιστέος παντάπασίν ἐστιν ἢ διαδοτέος τοῖς βουλομένοις λαμβάνειν, ὁπότερα δ’ ἂν ἐκείνοις δόξῃ, ταῦτα ποιεῖν.\\
\addlinespace
Isocrates (0010) & To Philip (020) & 89.1 & οἶμαι δὲ τῶν μὲν ἄλλων εἴ τισι δόξειε περὶ τῆς στρατείας τῆς εἰς τὴν Ἀσίαν συμβουλεύειν, ἐπὶ ταύτην ἂν ἐπιπεσεῖν τὴν παράκλησιν, λέγοντας ὡς ὅσοι περ ἐπεχείρησαν πρὸς τὸν βασιλέα πολεμεῖν, ἅπασι συνέπεσεν ἐξ ἀδόξων μὲν γενέσθαι λαμπροῖς, ἐκ πενήτων δὲ πλουσίοις, ἐκ ταπεινῶν δὲ πολλῆς χώρας καὶ πόλεων δεσπόταις.\\
Isocrates (0010) & Letter 2 (027) & 11.1 & ἀλλὰ τῶν μὲν βαρβάρων, πρὸς οὓς νῦν πολεμεῖς, ἐπὶ τοσοῦτον ἐξαρκέσει σοι κρατεῖν, ὅσον ἐν ἀσφαλείᾳ καταστῆσαι τὴν σαυτοῦ χώραν, τὸν δὲ βασιλέα τὸν νῦν μέγαν προσαγορευόμενον καταλύειν ἐπιχειρήσεις, ἵνα τήν τε σαυτοῦ δόξαν μείζω ποιήσῃς καὶ τοῖς Ἕλλησιν ὑποδείξῃς πρὸς ὃν χρὴ πολεμεῖν.\\
Isocrates (0010) & Aegineticus (006) & 49.1 & ἄξιον δ’ ἐστὶ καὶ τῷ νόμῳ βοηθεῖν καθ’ ὃν ἔξεστιν ἡμῖν καὶ παῖδας εἰσποιήσασθαι καὶ βουλεύσασθαι περὶ τῶν ἡμετέρων αὐτῶν, ἐνθυμηθέντας ὅτι τοῖς ἐρήμοις τῶν ἀνθρώπων ἀντὶ παίδων οὗτός ἐστιν·\\
Isocrates (0010) & Against Callimachus (002) & 39.1 & πρὸς δὲ τοὺτοις ἔτι καὶ νῦν ἔξεστιν αὐτῷ, πρὶν ἀποπειραθῆναι τῆς ὑμετὲρας γνώμης, ἁφέντι τὴν δίκην ἀπηλλάχθαι πάντων τῶν πραγμάτων.\\
Isocrates (0010) & Against Lochites (003) & 2.8 & περὶ δὲ τῆς ὕβρεως, ὡς κοινοῦ τοῦ πράγματος ὄντος, ἔξεστι τῷ βουλομένῳ τῶν πολιτῶν γραψαμένῳ πρὸς τοὺς θεσμοθέτας εἰσελθεῖν εἰς ὑμᾶς.\\
\addlinespace
Isocrates (0010) & Antidosis (019) & 237.1 & ἔχω δὲ δεῖξαι καὶ τόπους ἐν οἷς ἔξεστιν ἰδεῖν τοῖς βουλομένοις τοὺς πολυπράγμονας καὶ τοὺς ταῖς αἰτίαις ἐνόχους ὄντας ἃς οὗτοι τοῖς σοφισταῖς ἐπιφέρουσιν.\\
Isocrates (0010) & Nicocles or the Cyprians (014) & 64.3 & ὑμῖν δ’ ἔξεστι μηδὲν ταλαιπωρηθεῖσιν, ἀλλὰ πιστοῖς μόνον καὶ δικαίοις οὖσιν, ἅπαντα ταῦτα διαπράξασθαι.\\
Isocrates (0010) & Antidosis (019) & 104.3 & τοῖς μὲν γὰρ ἰδιώταις ὑπὲρ ὧν ἕκαστος ἔπραξε προσήκει διαλεχθεῖσι καταβαίνειν ἢ δοκεῖν περιεργάζεσθαι, τοῖς δ’ ὑπολαμβανομένοις συμβούλοις εἶναι καὶ διδασκάλοις ὁμοίως ὑπὲρ τῶν συγγεγενημένων ὥσπερ ὑπὲρ αὑτῶν ἀναγκαῖον ποιεῖσθαι τὴν ἀπολογίαν, ἄλλως τ’ ἢν καὶ τύχῃ τις διὰ τὴν αἰτίαν ταύτην κρινόμενος·\\
Isocrates (0010) & Antidosis (019) & 293.1 & ὥσθ’ ἅπασι μὲν βούλεσθαι προσήκει πολλοὺς εἶναι τοὺς ἐκ παιδείας δεινοὺς εἰπεῖν γιγνομένους, μάλιστα δ’ ὑμῖν·\\
Isocrates (0010) & Letter 9 (024) & 18.3 & σοὶ δὲ προσήκει προσέχοντι τὸν νοῦν τοῖς ὑπ’ ἐμοῦ λεγομένοις βουλεύσασθαι, πότερον ὀλιγωρητέον ἐστὶ τῶν Ἑλληνικῶν πραγμάτων γεγονότι μέν, ὥσπερ ὀλίγῳ πρότερον ἐγὼ διῆλθον, ἡγεμόνι δὲ Λακεδαιμονίων ὄντι, βασιλεῖ δὲ προσαγορευομένῳ, μεγίστην δὲ τῶν Ἑλλήνων ἔχοντι δόξαν, ἢ τῶν μὲν ἐνεστώτων πραγμάτων ὑπεροπτέον, μείζοσι δ’ ἐπιχειρητέον.\\
\addlinespace
Isocrates (0010) & Panegyricus (011) & 117.4 & οὓς ἡμεῖς διαβῆναι τολμήσαντας εἰς τὴν Εὐρώπην καὶ μεῖζον ἢ προσῆκεν αὐτοῖς φρονήσαντας οὕτω διέθεμεν, ὥστε μὴ μόνον παύσασθαι στρατείας ἐφ’ ἡμᾶς ποιουμένους ἀλλὰ καὶ τὴν αὑτῶν χώραν ἀνέχεσθαι πορθουμένην, καὶ διακοσίαις καὶ χιλίαις ναυσὶ περιπλέοντας εἰς τοσαύτην ταπεινότητα κατεστήσαμεν, ὥστε μακρὸν πλοῖον ἐπὶ τάδε Φασήλιδος μὴ καθέλκειν, ἀλλ’ ἡσυχίαν ἄγειν καὶ τοὺς καιροὺς περιμένειν, ἀλλὰ μὴ τῇ παρούσῃ δυνάμει πιστεύειν.\\
Isocrates (0010) & To Philip (020) & 89.1 & οἶμαι δὲ τῶν μὲν ἄλλων εἴ τισι δόξειε περὶ τῆς στρατείας τῆς εἰς τὴν Ἀσίαν συμβουλεύειν, ἐπὶ ταύτην ἂν ἐπιπεσεῖν τὴν παράκλησιν, λέγοντας ὡς ὅσοι περ ἐπεχείρησαν πρὸς τὸν βασιλέα πολεμεῖν, ἅπασι συνέπεσεν ἐξ ἀδόξων μὲν γενέσθαι λαμπροῖς, ἐκ πενήτων δὲ πλουσίοις, ἐκ ταπεινῶν δὲ πολλῆς χώρας καὶ πόλεων δεσπόταις.\\
Isocrates (0010) & Trapeziticus (005) & 9.8 & πανταχόθεν δέ μοι τοσούτων κακῶν προσπεπτωκότων τίν’ οἴεσθέ με γνώμην ἔχειν, ᾧ γ’ ὑπῆρχε σιγῶντι μὲν ὑπὸ τούτου ἀπεστερῆσθαι τῶν χρημάτων, λέγοντι δὲ ταῦτα μὲν μηδὲν μᾶλλον κομίσασθαι, πρὸς Σάτυρον δ’ εἰς τὴν μεγίστην διαβολὴν καὶ ἐμαυτὸν καὶ τὸν πατέρα καταστῆσαι;\\
Isocrates (0010) & Antidosis (019) & 34.1 & οὐ γὰρ δὴ τοῦτό γ’ ἐστὶν οὔτ’ εἰκὸς οὔτε δυνατόν, ἐμὲ μὲν περὶ πολλοὺς ἡμαρτηκέναι, τοὺς δὲ ταῖς συμφοραῖς δι’ ἐμὲ περιπεπτωκότας ἡσυχίαν ἔχειν καὶ μὴ τολμᾶν ἐγκαλεῖν, ἀλλὰ πραοτέρους ἐν τοῖς ἐμοῖς εἶναι κινδύνοις τῶν μηδὲν ἠδικημένων, ἐξὸν αὐτοῖς δηλώσασιν ἃ πεπόνθασι τὴν μεγίστην παρ’ ἐμοῦ λαβεῖν τιμωρίαν.\\
Isocrates (0010) & Antidosis (019) & 289.1 & οἵτινες ἐν ταύταις μὲν ταῖς ἀκμαῖς ὄντες ὑπερεῖδον τὰς ἡδονάς, ἐν αἷς οἱ πλεῖστοι τῶν τηλικούτων μάλιστ’ αὐτῶν ἐπιθυμοῦσιν, ἐξὸν δ’ αὐτοῖς ῥᾳθυμεῖν μηδὲν δαπανωμένοις εἵλοντο πονεῖν χρήματα τελέσαντες, ἄρτι δ’ ἐκ παίδων ἐξεληλυθότες ἔγνωσαν ἃ πολλοὶ τῶν πρεσβυτέρων οὐκ ἴσασιν, ὅτι δεῖ τὸν ὀρθῶς καὶ πρεπόντως προεστῶτα τῆς ἡλικίας καὶ καλὴν ἀρχὴν τοῦ βίου ποιούμενον αὑτοῦ πρότερον ἢ τῶν αὑτοῦ ποιήσασθαι τὴν ἐπιμέλειαν, καὶ μὴ σπεύδειν μηδὲ ζητεῖν ἑτέρων ἄρχειν πρὶν ἂν τῆς αὑτοῦ διανοίας λάβῃ τὸν ἐπιστατήσοντα, μηδ’ οὕτω χαίρειν μηδὲ μέγα φρονεῖν ἐπὶ τοῖς ἄλλοις ἀγαθοῖς ὡς ἐπὶ τοῖς ἐν τῇ ψυχῇ διὰ τὴν παιδείαν ἐγγιγνομένοις.\\
\addlinespace
Isocrates (0010) & Panegyricus (011) & 131.1 & ἐπεὶ καὶ τοῦτ’ ἔχομεν αὐτοῖς ἐπιτιμᾶν, ὅτι τῇ μὲν αὑτῶν πόλει τοὺς ὁμόρους εἱλωτεύειν ἀναγκάζουσι, τῷ δὲ κοινῷ τῷ τῶν συμμάχων οὐδὲν τοιοῦτον κατασκευάζουσιν, ἐξὸν αὐτοῖς τὰ πρὸς ἡμᾶς διαλυσαμένοις ἅπαντας τοὺς βαρβάρους περιοίκους ὅλης τῆς Ἑλλάδος καταστῆσαι.\\
Isocrates (0010) & To Philip (020) & 14.1 & ἀλλὰ τοὺς μὲν ἄλλους ἑώρων τοὺς ἐνδόξους τῶν ἀνδρῶν ὑπὸ πόλεσι καὶ νόμοις οἰκοῦντας, καὶ οὐδὲν ἐξὸν αὐτοῖς ἄλλο πράττειν πλὴν τὸ προσταττόμενον, ἔτι δὲ πολὺ καταδεεστέρους ὄντας τῶν πραγμάτων τῶν ῥηθησομένων, σοὶ δὲ μόνῳ πολλὴν ἐξουσίαν ὑπὸ τῆς τύχης δεδομένην καὶ πρέσβεις πέμπειν πρὸς οὕς τινας ἂν βουληθῇς, καὶ δέχεσθαι παρ’ ὧν ἄν σοι δοκῇ, καὶ λέγειν ὅ τι ἂν ἡγῇ συμφέρειν, πρὸς δὲ τούτοις καὶ πλοῦτον καὶ δύναμιν κεκτημένον ὅσην οὐδεὶς τῶν Ἑλλήνων, ἃ μόνα τῶν ὄντων καὶ πείθειν καὶ βιάζεσθαι πέφυκεν·\\
Isocrates (0010) & Aegineticus (006) & 51.3 & δέομαι οὖν ὑμῶν καὶ τούτων μεμνημένους καὶ τῶν ἄλλων τῶν εἰρημένων τὰ δίκαια ψηφίσασθαι, καὶ τοιούτους μοι γενέσθαι δικαστάς, οἵων περ ἂν αὐτοὶ τυχεῖν ἀξιώσαιτε.\\
Isocrates (0010) & Antidosis (019) & 17.1 & δέομαι οὖν ὑμῶν μήτε πιστεύειν πω μήτ’ ἀπιστεῖν τοῖς εἰρημένοις, πρὶν ἂν διὰ τέλους ἀκούσητε καὶ τὰ παρ’ ἡμῶν, ἐνθυμουμένους ὅτι οὐδὲν ἂν ἔδει δίδοσθαι τοῖς φεύγουσιν ἀπολογίαν, εἴπερ οἷόν τ’ ἦν ἐκ τῶν τοῦ διώκοντος λόγων ἐψηφίσθαι τὰ δίκαια.\\
Isocrates (0010) & Letter 8 (025) & 9.3 & εὑρήσετε γὰρ ἐμὲ μὲν οἰκειότατα κεχρημένον τοῖς μεγίστων ἀγαθῶν αἰτίοις γεγενημένοις ὑμῖν τε καὶ τοῖς ἄλλοις, ὑπὲρ ὧν δὲ δέομαι τοιούτους ὄντας, οἵους τοὺς μὲν πρεσβυτέρους καὶ τοὺς περὶ τὴν πολιτείαν ὄντας μὴ λυπεῖν, τοῖς δὲ νεωτέροις διατριβὴν παρέχειν ἡδεῖαν καὶ χρησίμην καὶ πρέπουσαν τοῖς τηλικούτοις.\\
\addlinespace
Isocrates (0010) & Trapeziticus (005) & 16.7 & ἐπειδὴ τοίνυν ἐκ τῶν συνόδων, ὦ ἄνδρες δικασταί, πάντες αὐτοῦ κατεγίγνωσκον ἀδικεῖν καὶ δεινὰ ποιεῖν, ὅστις τὸν παῖδα, ὃ ἔφασκον ἐγὼ συνειδέναι περὶ τῶν χρημάτων, πρῶτον μὲν αὐτὸς ἀφανίσας ὑφ’ ἡμῶν αὐτὸν ᾐτιᾶτ’ ἠφανίσθαι, ἔπειτα δὲ συλληφθέντα ὡς ἐλεύθερον ὄντα διεκώλυσε βασανίζεσθαι, μετὰ δὲ ταῦθ’ ὡς δοῦλον ἐκδοὺς καὶ βασανιστὰς ἑλόμενος λόγῳ μὲν ἐκέλευσε βασανίζειν, ἔργῳ δ’ οὐκ εἴα, διὰ ταῦθ’ ἡγούμενος οὐδεμίαν αὑτῷ σωτηρίαν εἶναι, ἐάνπερ εἰς ὑμᾶς εἰσέλθῃ, προσπέμπων ἐδεῖτό μου εἰς ἱερὸν ἐλθόνθ’ ἑαυτῷ συγγενέσθαι.\\
Isocrates (0010) & Trapeziticus (005) & 56.1 & ἐγὼ οὖ ὑμῶν δέομαι μεμνημένους τούτων καταψηφίσασθαι Πασίωνος, καὶ μὴ τοσαύτην πονηρίαν ἐμοῦ καταγνῶναι, ὡς οἰκῶν ἐν τῷ Πόντῳ καὶ τοσαύτην οὐσίαν κεκτημένος ὥστε καὶ ἑτέρους εὖ ποιεῖν δύνασθαι, Πασίων) ἦλθον συκοφαντήσων καὶ ψευδεῖς αὐτῷ παρακαταθήκας ἐγκαλῶν.\\
Isocrates (0010) & To Demonicus (007) & 45.8 & τὸν γὰρ αὑτῷ τὰ βέλτιστα πράττειν ἐπιτάττοντα, τοῦτον εἰκὸς καὶ τῶν ἄλλων τοὺς ἐπὶ τὴν ἀρετὴν παρακαλοῦντας ἀποδέχεσθαι.\\
Isocrates (0010) & Archidamus (016) & 11.1 & καίτοι λίαν προθύμως οἱ σύμμαχοι συμβεβουλεύκασιν ὑμῖν ὡς χρὴ Μεσσήνην ἀφέντας ποιήσασθαι τὴν εἰρήνην.\\
Isocrates (0010) & Archidamus (016) & 33.4 & λέγουσι δ’ οἱ συμβουλεύοντες ἡμῖν ποιεῖσθαι τὴν εἰρήνην, ὡς χρὴ τοὺς εὖ φρονοῦντας μὴ τὴν αὐτὴν γνώμην ἔχειν περὶ τῶν πραγμάτων εὐτυχοῦντας καὶ δυστυχοῦντας, ἀλλὰ πρὸς τὸ παρὸν ἀεὶ βουλεύεσθαι καὶ ταῖς τύχαις ἐπακολουθεῖν καὶ μὴ μεῖζον φρονεῖν τῆς δυνάμεως, μηδὲ τὸ δίκαιον ἐν τοῖς τοιούτοις καιροῖς ἀλλὰ τὸ συμφέρον ζητεῖν.\\
\addlinespace
Isocrates (0010) & Areopagiticus (018) & 72.1 & ἐγὼ δὲ καὶ τῶν ἰδιωτῶν τοὺς ὀλίγα μὲν κατορθοῦντας πολλὰ δ’ ἐξαμαρτάνοντας μέμφομαι καὶ νομίζω φαυλοτέρους εἶναι τοῦ δέοντος, καὶ πρός γε τούτοις τοὺς γεγονότας ἐκ καλῶν κἀγαθῶν ἀνδρῶν, καὶ μικρῷ μὲν ὄντας ἐπιεικεστέρους τῶν ὑπερβαλλόντων ταῖς πονηρίαις, πολὺ δὲ χείρους τῶν πατέρων, λοιδορῶ, καὶ συμβουλεύσαιμ’ ἂν αὐτοῖς παύσασθαι τοιούτοις οὖσιν.\\
Isocrates (0010) & On the Peace (017) & 145.1 & πολλῶν δὲ καὶ καλῶν λόγων ἐνόντων περὶ τὴν ὑπόθεσιν ταύτην, ἐμοὶ μὲν ἀμφότερα συμβουλεύει παύσασθαι λέγοντι, καὶ τὸ μῆκος τοῦ λόγου καὶ τὸ πλῆθος τῶν ἐτῶν τῶν ἐμῶν·\\
Isocrates (0010) & Panathenaicus (021) & 24.1 & ἀλλὰ μὴν οὐδ’ ἐκεῖνο ποιεῖν οὐδεὶς ἄν μοι συμβουλεύσειεν, ἀμελήσαντι τούτων καὶ μεταξὺ καταβαλόντι περαίνειν τὸν λόγον, ὃν προῄρημαι βουλόμενος ἐπιδεῖξαι τὴν πόλιν ἡμῶν πλειόνων ἀγαθῶν αἰτίαν γεγενημένην τοῖς Ἕλλησιν ἢ τὴν Λακεδαιμονίων·\\
Isocrates (0010) & Panathenaicus (021) & 168.5 & ὧν ἀκούσας οὐδένα χρόνον ἐπισχὼν ὁ δῆμος ἔπεμψε πρεσβείαν εἰς Θήβας, περί τε τῆς ἀναιρέσεως συμβουλεύσοντας αὐτοῖς ὁσιώτερον βουλεύσασθαι καὶ τὴν ἀπόκρισιν νομιμωτέραν ποιήσασθαι τῆς πρότερον γενομένης, κἀκεῖνο ὑποδείξοντας, ὡς ἡ πόλις αὐτοῖς οὐκ ἐπιτρέψει παραβαίνουσι τὸν νόμον τὸν κοινὸν ἁπάντων τῶν Ἑλλήνων.\\
Isocrates (0010) & Panathenaicus (021) & 262.5 & συμβουλεύω γάρ σοι μήτε κατακάειν τὸν λόγον μήτ’ ἀφανίζειν, ἀλλ’ εἴ τινος ἐνδεής ἐστι, διορθώσαντα καὶ προσγράψαντα πάσας τὰς διατριβὰς τὰς περὶ αὐτὸν γεγενημένας διδόναι τοῖς βουλομένοις λαμβάνειν, εἴπερ βούλει χαρίσασθαι μὲν τοῖς ἐπιεικεστάτοις τῶν Ἑλλήνων καὶ τοῖς ὡς ἀληθῶς φιλοσοφοῦσιν ἀλλὰ μὴ προσποιουμένοις, λυπῆσαι δὲ τοὺς θαυμάζοντας μὲν τὰ σὰ μᾶλλον τῶν ἄλλων, λοιδορουμένους δὲ τοῖς λόγοις τοῖς σοῖς ἐν τοῖς ὄχλοις τοῖς πανηγυρικοῖς, ἐν οἷς πλείους εἰσὶν οἱ καθεύδοντες τῶν ἀκροωμένων, καὶ προσδοκῶντας, ἢν παρακρούσωνται τοὺς τοιούτους, ἐναμίλλους τοὺς αὑτῶν γενήσεσθαι τοῖς ὑπὸ σοῦ γεγραμμένοις, κακῶς εἰδότας ὅτι πλέον ἀπολελειμμένοι τῶν σῶν εἰσιν ἢ τῆς Ὁμήρου δόξης οἱ περὶ τὴν αὐτὴν ἐκείνῳ ποίησιν γεγονότες. ʺ\\
\addlinespace
Isocrates (0010) & Plataicus (012) & 6.1 & δεόμεθ’ οὖν ὑμῶν, ὦ ἄνδρες Ἀθηναῖοι, μετ’ εὐνοίας ἀκροάσασθαι τῶν λεγομένων, ἐνθυμηθέντας ὅτι πάντων ἂν ἡμῖν ἀλογώτατον εἴη συμβεβηκός, εἰ τοῖς μὲν ἅπαντα τὸν χρόνον δυσμενῶς πρὸς τὴν πόλιν ὑμῶν διακειμένοις αἴτιοι γεγένησθε τῆς ἐλευθερίας, ἡμεῖς δὲ μηδ’ ἱκετεύοντες ὑμᾶς τῶν αὐτῶν τοῖς ἐχθίστοις τύχοιμεν.\\
Lycurgus (0034) & Against Leocrates (001) & 83.6 & ἐπὶ Κόδρου γὰρ βασιλεύοντος Πελοποννησίοις γενομένης ἀφορίας κατὰ τὴν χώραν αὐτῶν ἔδοξε στρατεύειν ἐπὶ τὴν πόλιν ἡμῶν, καὶ ἡμῶν τοὺς προγόνους ἐξαναστήσαντας κατανείμασθαι τὴν χώραν.\\
Lycurgus (0034) & Against Leocrates (001) & 123.1 & ἆρά γ’ ὑμῖν δοκεῖ βουλομένοις μιμεῖσθαι τοὺς προγόνους πάτριον εἶναι Λεωκράτην μὴ οὐκ ἀποκτεῖναι;\\
Lycurgus (0034) & Against Leocrates (001) & 125.1 & ἐψηφίσαντο γὰρ καὶ ὤμοσαν, ἐάν τις τυραννίδι ἐπιτιθῆται ἢ τὴν πόλιν προδιδῷ ἢ τὸν δῆμον καταλύῃ, τὸν αἰσθανόμενον καθαρὸν εἶναι ἀποκτείναντα, καὶ κρεῖττον ἔδοξεν αὐτοῖς τοὺς τὴν αἰτίαν ἔχοντας τεθνάναι μᾶλλον ἢ πειραθέντας μετὰ ἀληθείας αὐτοὺς δουλεύειν·\\
Lysias (0540) & Against Eratosthenes (012) & t.3 & οὐκ ἄρξασθαί μοι δοκεῖ ἄπορον εἶναι, ὦ ἄνδρες δικασταί, τῆς κατηγορίας, ἀλλὰ παύσασθαι λέγοντι·\\
\addlinespace
Lysias (0540) & Against Eratosthenes (012) & 15.1 & ἐκείνου δὲ διαλεγομένου Θεόγνιδι (ἔμπειρος γὰρ ὢν ἐτύγχανον τῆς οἰκίας, καὶ ᾔδη ὅτι ἀμφίθυρος εἴη) ἐδόκει μοι ταύτῃ πειρᾶσθαι σωθῆναι, ἐνθυμουμένῳ ὅτι, ἐὰν μὲν λάθω, σωθήσομαι, ἐὰν δὲ ληφθῶ, ἡγούμην μέν, εἰ Θέογνις εἴη πεπεισμένος ὑπὸ τοῦ Δαμνίππου χρήματα λαβεῖν, οὐδὲν ἧττον ἀφεθήσεσθαι, εἰ δὲ μή, ὁμοίως ἀποθανεῖσθαι.\\
Lysias (0540) & Against Pancleon (023) & 10.1 & τῇ δ’ ὑστεραίᾳ τῆς τε ἀντιγραφῆς ἕνεκα ταυτησὶ καὶ αὐτῆς τῆς δίκης ἔδοξέ μοι χρῆναι μάρτυρας λαβόντι παραγενέσθαι, ἵν’ εἰδείην τόν τ’ ἐξαιρησόμενον αὐτὸν καὶ ὅ τι λέγων ἀφαιρήσοιτο.\\
Lysias (0540) & On The Refusal Of A Pension (024) & 15.4 & ἐγὼ δ’ ὑμᾶς, ὦ βουλή, σαφῶς οἶμαι δεῖν διαγιγνώσκειν οἷς τ’ ἐγχωρεῖ τῶν ἀνθρώπων ὑβρισταῖς εἶναι καὶ οἷς οὐ προσήκει.\\
Lysias (0540) & On The Refusal Of A Pension (024) & 18.1 & καὶ τοῖς μὲν ἰσχυροῖς ἐγχωρεῖ μηδὲν αὐτοῖς πάσχουσιν, οὓς ἂν βουληθῶσιν, ὑβρίζειν, τοῖς δὲ ἀσθενέσιν οὐκ ἔστιν οὔτε ὑβριζομένοις ἀμύνεσθαι τοὺς ὑπάρξαντας οὔτε ὑβρίζειν βουλομένοις περιγίγνεσθαι τῶν ἀδικουμένων.\\
Lysias (0540) & Funeral Oration (002) & 6.2 & μόναις δ’ αὐταῖς οὐκ ἐξεγένετο ἐκ τῶν ἡμαρτημένων μαθούσαις περὶ τῶν λοιπῶν ἄμεινον βουλεύσασθαι, οὐδ’ οἴκαδε ἀπελθούσαις ἀπαγγεῖλαι τήν τε σφετέραν αὐτῶν δυστυχίαν καὶ τὴν τῶν ἡμετέρων προγόνων ἀρετήν·\\
\addlinespace
Lysias (0540) & Against Diogeiton (032) & 23.1 & καίτοι εἰ ἐβούλετο δίκαιος εἶναι περὶ τοὺς παῖδας, ἐξῆν αὐτῷ κατὰ τοὺς νόμους, οἳ κεῖνται περὶ τῶν ὀρφανῶν καὶ τοῖς ἀδυνάτοις τῶν ἐπιτρόπων καὶ τοῖς δυναμένοις, μισθῶσαι τὸν οἶκον ἀπηλλαγμένον πολλῶν πραγμάτων, ἢ γῆν πριάμενον ἐκ τῶν προσιόντων τοὺς παῖδας τρέφειν·\\
Lysias (0540) & Against Philon (031) & 16.1 & ἵνα οὖν μὴ ἐγγένηται αὐτῷ ψευσαμένῳ ἐξαπατῆσαι, καὶ περὶ τούτων ἤδη σαφῶς ὑμῖν ἀποδείξω, ἐπειδὴ ὕστερον οὐκ ἐξέσται μοι παρελθόντι ἐνθάδ’ ἐλέγχειν αὐτόν.\\
Lysias (0540) & For Polystratus (020) & 23.1 & καὶ ἐξὸν αὐτῷ τὴν οὐσίαν ἀφανῆ καταστήσαντι μηδὲν ὑμᾶς ὠφελεῖν, εἵλετο μᾶλλον συνειδέναι ὑμᾶς, ἵν’ εἰ καὶ βούλοιτο κακὸς εἶναι, μὴ ἐξείη αὐτῷ, ἀλλ’ εἰσφέροι τε τὰς εἰσφορὰς καὶ λῃτουργοίη.\\
Lysias (0540) & On A Wound By Premeditation (004) & 9.2 & ὁ δ’ εἰς τοῦτο βαρυδαιμονίας ἥκει, ὥστε οὐκ αἰσχύνεται τραύματ’ ὀνομάζων τὰ ὑπώπια καὶ ἐν κλίνῃ περιφερόμενος καὶ δεινῶς προσποιούμενος διακεῖσθαι ἕνεκα πόρνης ἀνθρώπου, ἣν ἔξεστιν αὐτῷ ἀναμφισβητήτως ἔχειν ἐμοὶ ἀποδόντι τἀργύριον.\\
Lysias (0540) & On the Murder of Eratosthenes (001) & 18.1 & σοὶ οὖν ἔφην ἔξεστι δυοῖν ὁπότερον βούλει ἑλέσθαι, ἢ μαστιγωθεῖσαν εἰς μύλωνα ἐμπεσεῖν καὶ μηδέποτε παύσασθαι κακοῖς τοιούτοις συνεχομένην, ἢ κατειποῦσαν ἅπαντα τἀληθῆ μηδὲν παθεῖν κακόν, ἀλλὰ συγγνώμης παρ’ ἐμοῦ τυχεῖν τῶν ἡμαρτημένων.\\
\addlinespace
Lysias (0540) & Defense Against A Charge Of Taking Bribes (021) & 15.1 & ἄξιον δέ ἐστιν ἐνθυμηθῆναι ὅτι πολὺ μᾶλλον ὑμῖν προσήκει τῶν ὑμετέρων ἐμοὶ διδόναι ἢ τῶν ἐμῶν ἐμοὶ ἀμφισβητῆσαι, καὶ πένητα γενόμενον ἐλεῆσαι μᾶλλον ἢ πλουτοῦντι φθονῆσαι, καὶ τοῖς θεοῖς εὔχεσθαι τοὺς ἄλλους εἶναι τοιούτους πολίτας, ἵνα τῶν μὲν ὑμετέρων μὴ ἐπιθυμήσωσι, τὰ δὲ σφέτερα αὐτῶν εἰς ὑμᾶς ἀναλίσκωσιν.\\
Lysias (0540) & Funeral Oration (002) & 17.1 & πολλὰ μὲν ὑπῆρχε τοῖς ἡμετέροις προγόνοις μιᾷ γνώμῃ χρωμένοις περὶ τοῦ δικαίου διαμάχεσθαι·\\
Lysias (0540) & Defense in the Matter of the Olive Stump (007) & t.2 & πρότερον μέν, ὦ βουλή, ἐνόμιζον ἐξεῖναι τῷ βουλομένῳ, ἡσυχίαν ἄγοντι, μήτε δίκας ἔχειν μήτε πράγματα·\\
Lysias (0540) & Against Alcibiades 2 (015) & t.1 & ἐγὼ μέν, ὦ ἄνδρες δικασταί, καὶ ὑμᾶς αἰτοῦμαι τὰ δίκαια ψηφίσασθαι, καὶ τῶν στρατηγῶν δέομαι, ἐπεὶ καὶ ἐν τῇ ἄλλῃ ἀρχῇ πολλοῦ ἄξιοι τῇ πόλει γεγόνασι, καὶ τῶν τῆς ἀστρατείας γραφῶν κοινοὺς εἶναι τῷ τε διώκοντι καὶ τῷ φεύγοντι, καὶ μὴ βοηθοῦντας ᾧ ἂν βούλωνται πᾶσαν προθυμίαν ἔχειν παρὰ τὸ δίκαιον ὑμᾶς ψηφίσασθαι, ἐνθυμουμένους ὅτι σφόδρ’ ἂν ἠγανακτεῖτε, εἰ ἐν τῇ ὑμετέρᾳ δοκιμασίᾳ οἱ θεσμοθέται ἀναβάντες ὑμῶν ἐδέοντο καταψηφίσασθαι, ἡγούμενοι δεινὸν εἶναι εἰ οἱ τιθέντες τὸν ἀγῶνα καὶ τὴν ψῆφον διδόντες παρακελεύσονται τῶν μὲν μὴ καταψηφίζεσθαι τῶν δὲ καταψηφίζεσθαι.\\
Lysias (0540) & Against Theomnestus 1 (010) & 21.1 & εἰ δὲ μή, δέομαι ὑμῶν, ὦ ἄνδρες δικασταί, τὰ δίκαια ψηφίσασθαι, ἐνθυμουμένους ὅτι πολὺ μεῖζον κακόν ἐστιν ἀκοῦσαί τινα τὸν πατέρα ἀπεκτονέναι ἢ τὴν ἀσπίδα ἀποβεβληκέναι.\\
\addlinespace
Lysias (0540) & Against Theomnestus 1 (010) & 31.2 & ἐγὼ δ’ ὑμῶν δέομαι καταψηφίσασθαι Θεομνήστου, ἐνθυμουμένους ὅτι οὐκ ἂν γένοιτο τούτου μείζων ἀγών μοι.\\
Lysias (0540) & Defense Against A Charge Of Taking Bribes (021) & 19.1 & δέομαι οὖν ὑμῶν, ὦ ἄνδρες δικασταί, τὴν αὐτὴν νῦν περὶ ἐμοῦ γνώμην ἔχειν ἥνπερ καὶ ἐν τῷ τέως χρόνῳ, καὶ μὴ μόνον τῶν δημοσίων λῃτουργιῶν μεμνῆσθαι, ἀλλὰ τῶν ἰδίων ἐπιτηδευμάτων ἐνθυμεῖσθαι, ἡγουμένους ταύτην εἶναι λῃτουργίαν ἐπιπονωτάτην, διὰ τέλους τὸν πάντα χρόνον κόσμιον εἶναι καὶ σώφρονα καὶ μήθ’ ὑφ’ ἡδονῆς ἡττηθῆναι μήθ’ ὑπὸ κέρδους ἐπαρθῆναι, ἀλλὰ τοιοῦτον παρασχεῖν ἑαυτὸν ὥστε μηδένα τῶν πολιτῶν μήτε μέμψασθαι μήτε δίκην τολμῆσαι προσκαλέσασθαι.\\
Lysias (0540) & Defense in the Matter of the Olive Stump (007) & 29.1 & ἐγὼ τοίνυν δέομαι ὑμῶν μὴ τοὺς τοιούτους λόγους πιστοτέρους ἡγήσασθαι τῶν ἔργων, μηδὲ περὶ ὧν αὐτοὶ σύνιστε, τοιαῦτ’ ἀνασχέσθαι τῶν ἐμῶν ἐχθρῶν λεγόντων, ἐνθυμουμένους καὶ ἐκ τῶν εἰρημένων καὶ ἐκ τῆς ἄλλης πολιτείας.\\
Lysias (0540) & On the Property of Aristophanes (019) & 11.6 & δέομαι δ’ ὑμῶν πάσῃ τέχνῃ καὶ μηχανῇ μετ’ εὐνοίας ἀκροασαμένους ἡμῶν διὰ τέλους ὅ τι ἂν ὑμῖν ἄριστον καὶ εὐορκότατον νομίζητε εἶναι, τοῦτο ψηφίσασθαι.\\
Lysias (0540) & On the Property of Aristophanes (019) & 64.1 & δέομαι οὖν ὑμῶν, ὦ ἄνδρες δικασταί, καὶ τούτων καὶ τῶν ἄλλων μεμνημένους ἁπάντων τῶν εἰρημένων βοηθεῖν ἡμῖν καὶ μὴ περιιδεῖν ὑπὸ τῶν ἐχθρῶν ἀναιρεθέντας.\\
\addlinespace
Lysias (0540) & Against Agoratus (013) & 94.3 & καὶ οὕτως ἂν δεινότατα πάντων πάθοιεν, εἰ οἷς ἐπέσκηπτον ἐκεῖνοι ὡς φίλοις οὖσι τιμωρεῖν ὑπὲρ αὑτῶν, οὗτοι ὁμόψηφοι κατ’ ἐκείνων τῶν ἀνδρῶν τοῖς τριάκοντα γενήσονται.\\
Lysias (0540) & Against The Corn-Dealers (022) & 8.3 & καὶ οἱ μὲν δύο οὐδὲν ἔφασαν εἰδέναι τοῦ πράγματος, Ἄνυτος δ’ ἔλεγεν ὡς τοῦ προτέρου χειμῶνος, ἐπειδὴ τίμιος ἦν ὁ σῖτος, τούτων ὑπερβαλλόντων ἀλλήλους καὶ πρὸς σφᾶς αὐτοὺς μαχομένων συμβουλεύσειεν αὐτοῖς παύσασθαι φιλονικοῦσιν, ἡγούμενος συμφέρειν ὑμῖν τοῖς παρὰ τούτων ὠνουμένοις ὡς ἀξιώτατον τούτους πρίασθαι·\\
Plato (0059) & Gorgias (023) & 485.a.6 & ὅταν μὲν γὰρ παιδίον ἴδω, ᾧ ἔτι προσήκει διαλέγεσθαι οὕτω, ψελλιζόμενον καὶ παῖζον, χαίρω τε καὶ χαρίεν μοι φαίνεται καὶ ἐλευθέριον καὶ πρέπον τῇ τοῦ παιδίου ἡλικίᾳ, ὅταν δὲ σαφῶς διαλεγομένου παιδαρίου ἀκούσω, πικρόν τί μοι δοκεῖ χρῆμα εἶναι καὶ ἀνιᾷ μου τὰ ὦτα καί μοι δοκεῖ δουλοπρεπές τι εἶναι·\\
Plato (0059) & Laches (019) & 181.d.3 & δικαιότατον μέντοι μοι δοκεῖ εἶναι ἐμὲ νεώτερον ὄντα τῶνδε καὶ ἀπειρότερον τούτων ἀκούειν πρότερον τί λέγουσιν καὶ μανθάνειν παρ’ αὐτῶν·\\
Plato (0059) & Laws (034) & 683.d.2 & τὸ δὲ δὴ μετὰ τοῦτο ἔδοξεν αὐτοῖς, ὥς γε λέγεται τὸ τοῦ μύθου, τριχῇ τὸ στράτευμα διανείμαντας, τρεῖς πόλεις κατοικίζειν, Ἄργος, Μεσσήνην, Λακεδαίμονα.\\
\addlinespace
Plato (0059) & Lysis (020) & 213.e.3 & ἀλλὰ ταύτῃ μὲν μηκέτι ἴωμεν—καὶ γὰρ χαλεπή τίς μοι φαίνεται ὥσπερ ὁδὸς ἡ σκέψις—ᾗ δὲ ἐτράπημεν, δοκεῖ μοι χρῆναι ἰέναι, σκοποῦντα κατὰ τοὺς ποιητάς·\\
Plato (0059) & Menexenus (028) & 239.c.3 & ὧν δὲ οὔτε ποιητής πω δόξαν ἀξίαν ἐπ’ ἀξίοις λαβὼν ἔχει ἔτι τέ ἐστιν ἐν ἀμνηστίᾳ, τούτων πέρι μοι δοκεῖ χρῆναι ἐπιμνησθῆναι ἐπαινοῦντά τε καὶ προμνώμενον ἄλλοις ἐς ᾠδάς τε καὶ τὴν ἄλλην ποίησιν αὐτὰ θεῖναι πρεπόντως τῶν πραξάντων.\\
Plato (0059) & Phaedo (004) & 86.d.7 & δοκεῖ μέντοι μοι χρῆναι πρὸ τῆς ἀποκρίσεως ἔτι πρότερον κέβητος ἀκοῦσαι τί αὖ ὅδε ἐγκαλεῖ τῷ λόγῳ, ἵνα χρόνου ἐγγενομένου βουλευσώμεθα τί ἐροῦμεν, ἔπειτα ἀκούσαντας ἢ συγχωρεῖν αὐτοῖς ἐάν τι δοκῶσι προσᾴδειν, ἐὰν δὲ μή, οὕτως ἤδη ὑπερδικεῖν τοῦ λόγου.\\
Plato (0059) & Phaedo (004) & 98.d.6 & καὶ αὖ περὶ τοῦ διαλέγεσθαι ὑμῖν ἑτέρας τοιαύτας αἰτίας λέγοι, φωνάς τε καὶ ἀέρας καὶ ἀκοὰς καὶ ἄλλα μυρία τοιαῦτα αἰτιώμενος, ἀμελήσας τὰς ὡς ἀληθῶς αἰτίας λέγειν, ὅτι, ἐπειδὴ ἔδοξε βέλτιον εἶναι ἐμοῦ καταψηφίσασθαι, διὰ ταῦτα δὴ καὶ ἐμοὶ βέλτιον αὖ δέδοκται ἐνθάδε καθῆσθαι, καὶ δικαιότερον παραμένοντα ὑπέχειν τὴν δίκην ἣν ἂν κελεύσωσιν·\\
Plato (0059) & Phaedo (004) & 99.e.4 & ἔδοξε δή μοι χρῆναι εἰς τοὺς λόγους καταφυγόντα ἐν ἐκείνοις σκοπεῖν τῶν ὄντων τὴν ἀλήθειαν.\\
\addlinespace
Plato (0059) & Republic (030) & 360.b.3 & εἰ οὖν δύο τοιούτω δακτυλίω γενοίσθην, καὶ τὸν μὲν ὁ δίκαιος περιθεῖτο, τὸν δὲ ὁ ἄδικος, οὐδεὶς ἂν γένοιτο, ὡς δόξειεν, οὕτως ἀδαμάντινος, ὃς ἂν μείνειεν ἐν τῇ δικαιοσύνῃ καὶ τολμήσειεν ἀπέχεσθαι τῶν ἀλλοτρίων καὶ μὴ ἅπτεσθαι, ἐξὸν αὐτῷ καὶ ἐκ τῆς ἀγορᾶς ἀδεῶς ὅτι βούλοιτο λαμβάνειν, καὶ εἰσιόντι εἰς τὰς οἰκίας συγγίγνεσθαι ὅτῳ βούλοιτο, καὶ ἀποκτεινύναι καὶ ἐκ δεσμῶν λύειν οὕστινας βούλοιτο, καὶ τἆλλα πράττειν ἐν τοῖς ἀνθρώποις ἰσόθεον ὄντα.\\
Plato (0059) & Republic (030) & 578.c.9 & δοκεῖ γάρ μοι δεῖν ἐννοῆσαι ἐκ τῶνδε περὶ αὐτοῦ σκοποῦντας.\\
Plato (0059) & Protagoras (022) & 321.d.5 & τῷ δὲ Προμηθεῖ εἰς μὲν τὴν ἀκρόπολιν τὴν τοῦ Διὸς οἴκησιν οὐκέτι ἐνεχώρει εἰσελθεῖν—πρὸς δὲ καὶ αἱ Διὸς φυλακαὶ φοβεραὶ ἦσαν—εἰς δὲ τὸ τῆς Ἀθηνᾶς καὶ Ἡφαίστου οἴκημα τὸ κοινόν, ἐν ᾧ ἐφιλοτεχνείτην, λαθὼν εἰσέρχεται, καὶ κλέψας τήν τε ἔμπυρον τέχνην τὴν τοῦ Ἡφαίστου καὶ τὴν ἄλλην τὴν τῆς Ἀθηνᾶς δίδωσιν ἀνθρώπῳ, καὶ ἐκ τούτου εὐπορία μὲν ἀνθρώπῳ τοῦ βίου γίγνεται, Προμηθέα δὲ δι’ Ἐπιμηθέα ὕστερον, ᾗπερ λέγεται, κλοπῆς δίκη μετῆλθεν.\\
Plato (0059) & Protagoras (022) & 344.d.5 & τῷ μὲν γὰρ ἐσθλῷ ἐγχωρεῖ κακῷ γενέσθαι, ὥσπερ καὶ παρ’ ἄλλου ποιητοῦ μαρτυρεῖται τοῦ εἰπόντος—αὐτὰρ ἀνὴρ ἀγαθὸς τοτὲ μὲν κακός, ἄλλοτε δ’ ἐσθλός·\\
Plato (0059) & Republic (030) & 403.e.4 & παντὶ γάρ που μᾶλλον ἐγχωρεῖ ἢ φύλακι μεθυσθέντι μὴ εἰδέναι ὅπου γῆς ἐστιν.\\
\addlinespace
Plato (0059) & Republic (030) & 408.e.1 & οὐ γὰρ οἶμαι σώματι σῶμα θεραπεύουσιν—οὐ γὰρ ἂν αὐτὰ ἐνεχώρει κακὰ εἶναί ποτε καὶ γενέσθαι—ἀλλὰ ψυχῇ σῶμα, ᾗ οὐκ ἐγχωρεῖ κακὴν γενομένην τε καὶ οὖσαν εὖ τι θεραπεύειν.\\
Plato (0059) & Republic (030) & 494.d.4 & τῷ δὴ οὕτω διατιθεμένῳ ἐάν τις ἠρέμα προσελθὼν τἀληθῆ λέγῃ, ὅτι νοῦς οὐκ ἔνεστιν αὐτῷ, δεῖται δέ, τὸ δὲ οὐ κτητὸν μὴ δουλεύσαντι τῇ κτήσει αὐτοῦ, ἆρ’ εὐπετὲς οἴει εἶναι εἰσακοῦσαι διὰ τοσούτων κακῶν;\\
Plato (0059) & Alcibiades 2 (014) & 141.a.5 & ἐγὼ μὲν γὰρ οἶμαί σε πρῶτον, εἴ σοι ἐμφανὴς γενόμενος ὁ θεὸς πρὸς ὃν τυγχάνεις πορευόμενος, ἐρωτήσειεν, πρὶν ὁτιοῦν εὔξασθαί σε, εἰ ἐξαρκέσει σοι τύραννον γενέσθαι τῆς Ἀθηναίων πόλεως·\\
Plato (0059) & Republic (030) & 341.e.1 & ὥσπερ, ἔφην ἐγώ, εἴ με ἔροιο εἰ ἐξαρκεῖ σώματι εἶναι σώματι ἢ προσδεῖταί τινος, εἴποιμ’ ἂν ὅτι παντάπασι μὲν οὖν προσδεῖται.\\
Plato (0059) & Republic (030) & 341.e.4 & διὰ ταῦτα καὶ ἡ τέχνη ἐστὶν ἡ ἰατρικὴ νῦν ηὑρημένη, ὅτι σῶμά ἐστιν πονηρὸν καὶ οὐκ ἐξαρκεῖ αὐτῷ τοιούτῳ εἶναι.\\
\addlinespace
Plato (0059) & Symposium (011) & 192.b.3 & ἀλλ’ ἐξαρκεῖ αὐτοῖς μετ’ ἀλλήλων καταζῆν ἀγάμοις.\\
Plato (0059) & Epinomis (035) & 982.d.7 & ἐξῆν δὲ ἀνθρώπῳ γε ἐπὶ τὰ καλλίω καὶ βελτίω καὶ φίλα τιθεμένῳ λαμβάνειν ὡς διὰ τοῦτο αὐτὸ ἔμφρον δεῖ νομίζειν τὸ κατὰ ταὐτὰ καὶ ὡσαύτως καὶ διὰ ταὐτὰ πρᾶττον ἀεί, τοῦτο δ’ εἶναι τὴν τῶν ἄστρων φύσιν, ἰδεῖν μὲν καλλίστην, πορείαν δὲ καὶ χορείαν πάντων χορῶν καλλίστην καὶ μεγαλοπρεπεστάτην χορεύοντα πᾶσι τοῖς ζῶσι τὸ δέον ἀποτελεῖν.\\
Plato (0059) & Laws (034) & 858.b.2 & ἡμῖν δ’—εἰπεῖν σὺν θεῷ—ἔξεστι, καθάπερ ἢ λιθολόγοις ἢ καί τινος ἑτέρας ἀρχομένοις συστάσεως, παραφορήσασθαι χύδην ἐξ ὧν ἐκλεξόμεθα τὰ πρόσφορα τῇ μελλούσῃ γενήσεσθαι συστάσει, καὶ δὴ καὶ κατὰ σχολὴν ἐκλέξασθαι.\\
Plato (0059) & Protagoras (022) & 350.e.3 & ταῦτα δὲ ἐμοῦ ὁμολογήσαντος ἐξείη ἄν σοι, χρωμένῳ τοῖς αὐτοῖς τεκμηρίοις τούτοις, λέγειν ὡς κατὰ τὴν ἐμὴν ὁμολογίαν ἡ σοφία ἐστὶν ἰσχύς.\\
Plato (0059) & Republic (030) & 422.b.10 & οὐδ’ εἰ ἐξείη, ἦν δ’ ἐγώ, ὑποφεύγοντι τὸν πρότερον ἀεὶ προσφερόμενον ἀναστρέφοντα κρούειν, καὶ τοῦτο ποιοῖ πολλάκις ἐν ἡλίῳ τε καὶ πνίγει;\\
\addlinespace
Plato (0059) & Republic (030) & 579.b.5 & λίχνῳ δὲ ὄντι αὐτῷ τὴν ψυχὴν μόνῳ τῶν ἐν τῇ πόλει οὔτε ἀποδημῆσαι ἔξεστιν οὐδαμόσε, οὔτε θεωρῆσαι ὅσων δὴ καὶ οἱ ἄλλοι ἐλεύθεροι ἐπιθυμηταί εἰσιν, καταδεδυκὼς δὲ ἐν τῇ οἰκίᾳ τὰ πολλὰ ὡς γυνὴ ζῇ, φθονῶν καὶ τοῖς ἄλλοις πολίταις, ἐάν τις ἔξω ἀποδημῇ καί τι ἀγαθὸν ὁρᾷ;\\
Plato (0059) & Laws (034) & 906.a.7 & ψυχαὶ δέ τινες ἐπὶ γῆς οἰκοῦσαι καὶ ἄδικον λῆμμα κεκτημέναι δῆλον ὅτι θηριώδεις, πρὸς τὰς τῶν φυλάκων ψυχὰς ἄρα κυνῶν ἢ τὰς τῶν νομέων ἢ πρὸς τὰς τῶν παντάπασιν ἀκροτάτων δεσποτῶν προσπίπτουσαι, πείθουσιν θωπείαις λόγων καὶ ἐν εὐκταίαις τισὶν ἐπῳδαῖς, ὡς αἱ φῆμαί φασιν αἱ τῶν κακῶν, ἐξεῖναι πλεονεκτοῦσιν σφίσιν ἐν ἀνθρώποις πάσχειν μηδὲν χαλεπόν·\\
Plato (0059) & Apology (002) & 31.b.1 & οὐ γὰρ ἀνθρωπίνῳ ἔοικε τὸ ἐμὲ τῶν μὲν ἐμαυτοῦ πάντων ἠμεληκέναι καὶ ἀνέχεσθαι τῶν οἰκείων ἀμελουμένων τοσαῦτα ἤδη ἔτη, τὸ δὲ ὑμέτερον πράττειν ἀεί, ἰδίᾳ ἑκάστῳ προσιόντα ὥσπερ πατέρα ἢ ἀδελφὸν πρεσβύτερον πείθοντα ἐπιμελεῖσθαι ἀρετῆς.\\
Plato (0059) & Laws (034) & 879.c.3 & ἔοικεν δὲ νέῳ παντὶ ὑπὸ γέροντος πληγέντι ῥᾳθύμως ὀργὴν ὑποφέρειν, αὑτῷ τιθεμένῳ τιμὴν ταύτην εἰς γῆρας.\\
Plato (0059) & Cratylus (005) & 407.d.2 & εἰ δ’ αὖ κατὰ τὸ σκληρόν τε καὶ ἀμετάστροφον, ὃ δὴ ἄρρατον καλεῖται, καὶ ταύτῃ ἂν πανταχῇ πολεμικῷ θεῷ πρέποι Ἄρη καλεῖσθαι.\\
\addlinespace
Plato (0059) & Epistles (036) & 356.c.8 & τούτους δ’ ἐλθόντας νόμους μὲν πρῶτον θεῖναι καὶ πολιτείαν τοιαύτην, ἐν ᾗ βασιλέας ἁρμόττει γίγνεσθαι κυρίους ἱερῶν τε καὶ ὅσων ἄλλων πρέπει τοῖς γενομένοις ποτὲ εὐεργέταις, πολέμου δὲ καὶ εἰρήνης ἄρχοντας νομοφύλακας ποιήσασθαι ἀριθμὸν τριάκοντα καὶ πέντε μετά τε δήμου καὶ βουλῆς.\\
Plato (0059) & Hippias Major (025) & 291.a.5 & σοὶ μὲν γὰρ οὐκ ἂν πρέποι τοιούτων ὀνομάτων ἀναπίμπλασθαι, καλῶς μὲν οὑτωσὶ ἀμπεχομένῳ, καλῶς δὲ ὑποδεδεμένῳ, εὐδοκιμοῦντι δὲ ἐπὶ σοφίᾳ ἐν πᾶσι τοῖς Ἕλλησιν.\\
Plato (0059) & Hippias Major (025) & 293.d.6 & ὦ δαιμόνιε, φησί, Σώκρατες, τὰ μὲν τοιαῦτα ἀποκρινόμενος καὶ οὕτω παῦσαι—λίαν γὰρ εὐήθη τε καὶ εὐεξέλεγκτά ἐστιν—ἀλλὰ τὸ τοιόνδε σκόπει εἴ σοι δοκεῖ καλὸν εἶναι, οὗ καὶ νυνδὴ ἐπελαβόμεθα ἐν τῇ ἀποκρίσει, ἡνίκ’ ἔφαμεν τὸν χρυσὸν οἷς μὲν πρέπει καλὸν εἶναι, οἷς δὲ μή, οὔ, καὶ τἆλλα πάντα οἷς ἂν τοῦτο προσῇ·\\
Plato (0059) & Ion (027) & 540.c.4 & οἷον βουκόλῳ λέγεις δούλῳ ἃ πρέπει εἰπεῖν ἀγριαινουσῶν βοῶν παραμυθουμένῳ, ὁ ῥαψῳδὸς γνώσεται ἀλλ’ οὐχ ὁ βουκόλος;\\
Plato (0059) & Ion (027) & 540.d.1 & ἀλλ’ οἷα ἀνδρὶ πρέπει εἰπεῖν γνώσεται στρατηγῷ στρατιώταις παραινοῦντι;\\
\addlinespace
Plato (0059) & Phaedo (004) & 92.c.4 & οὐδαμῶς, ἔφη ὁ καὶ μήν, ἦ δ’ ὅς, πρέπει γε εἴπερ τῳ ἄλλῳ λόγῳ συνῳδῷ εἶναι καὶ τῷ περὶ ἁρμονίας.\\
Plato (0059) & Gorgias (023) & 471.a.4 & ᾧ γε προσῆκε μὲν τῆς ἀρχῆς οὐδὲν ἣν νῦν ἔχει, ὄντι ἐκ γυναικὸς ἣ ἦν δούλη Ἀλκέτου τοῦ Περδίκκου ἀδελφοῦ, καὶ κατὰ μὲν τὸ δίκαιον δοῦλος ἦν Ἀλκέτου, καὶ εἰ ἐβούλετο τὰ δίκαια ποιεῖν, ἐδούλευεν ἂν Ἀλκέτῃ καὶ ἦν εὐδαίμων κατὰ τὸν σὸν λόγον.\\
Plato (0059) & Gorgias (023) & 525.b.1 & προσήκει δὲ παντὶ τῷ ἐν τιμωρίᾳ ὄντι, ὑπ’ ἄλλου ὀρθῶς τιμωρουμένῳ, ἢ βελτίονι γίγνεσθαι καὶ ὀνίνασθαι ἢ παραδείγματι τοῖς ἄλλοις γίγνεσθαι, ἵνα ἄλλοι ὁρῶντες πάσχοντα ἃ ἂν πάσχῃ φοβούμενοι βελτίους γίγνωνται.\\
Plato (0059) & Laches (019) & 199.d.7 & καὶ τοῦτον οἴει ἂν σὺ ἐνδεᾶ εἶναι σωφροσύνης ἢ δικαιοσύνης τε καὶ ὁσιότητος, ᾧ γε μόνῳ προσήκει καὶ περὶ θεοὺς καὶ περὶ ἀνθρώπους ἐξευλαβεῖσθαί τε τὰ δεινὰ καὶ τὰ μή, καὶ τἀγαθὰ πορίζεσθαι, ἐπισταμένῳ ὀρθῶς προσομιλεῖν;\\
Plato (0059) & Laches (019) & 200.a.4 & εὖ γε, ὦ Λάχης, ὅτι οὐδὲν οἴει σὺ ἔτι πρᾶγμα εἶναι ὅτι αὐτὸς ἄρτι ἐφάνης ἀνδρείας πέρι οὐδὲν εἰδώς, ἀλλ’ εἰ καὶ ἐγὼ ἕτερος τοιοῦτος ἀναφανήσομαι, πρὸς τοῦτο βλέπεις, καὶ οὐδὲν ἔτι διοίσει, ὡς ἔοικε, σοὶ μετ’ ἐμοῦ μηδὲν εἰδέναι ὧν προσήκει ἐπιστήμην ἔχειν ἀνδρὶ οἰομένῳ τὶ εἶναι.\\
\addlinespace
Plato (0059) & Laws (034) & 672.b.7 & ἐγὼ δὲ τὰ μὲν τοιαῦτα τοῖς ἀσφαλὲς ἡγουμένοις εἶναι λέγειν περὶ θεῶν ἀφίημι λέγειν, τὸ δὲ τοσόνδε οἶδα, ὅτι πᾶν ζῷον, ὅσον αὐτῷ προσήκει νοῦν ἔχειν τελεωθέντι, τοῦτον καὶ τοσοῦτον οὐδὲν ἔχον ποτὲ φύεται·\\
Plato (0059) & Laws (034) & 758.b.2 & πλῆθος δὲ οὐ δυνατὸν ὀξέως οὐδέποτε οὐδὲν τούτων πράττειν, ἀναγκαῖον δὲ τοὺς μὲν πολλοὺς τῶν βουλευτῶν ἐπὶ τὸν πλεῖστον τοῦ χρόνου ἐᾶν ἐπὶ τοῖς αὑτῶν ἰδίοισι μένοντας εὐθημονεῖσθαι τὰ κατὰ τὰς αὑτῶν οἰκήσεις, τὸ δὲ δωδέκατον μέρος αὐτῶν ἐπὶ δώδεκα μῆνας νείμαντας, ἓν ἐφ’ ἑνὶ παρέχειν αὐτοὺς φύλακας ἰόντι τέ τινί ποθεν ἄλλοθεν εἴτε καὶ ἐξ αὐτῆς τῆς πόλεως ἑτοίμως ἐπιτυχεῖν, ἄντε ἀγγέλλειν βούληταί τις ἐάντ’ αὖ πυνθάνεσθαί τι τῶν ὧν προσήκει πόλει πρὸς πόλεις ἄλλας ἀποκρίνεσθαί τε, καὶ ἐρωτήσασαν ἑτέρας, ἀποδέξασθαι τὰς ἀποκρίσεις, καὶ δὴ καὶ τῶν κατὰ πόλιν ἑκάστοτε νεωτερισμῶν ἕνεκα παντοδαπῶν εἰωθότων ἀεὶ γίγνεσθαι, ὅπως ἂν μάλιστα μὲν μὴ γίγνωνται, γενομένων δέ, ὅτι τάχιστα αἰσθομένης τῆς πόλεως ἰαθῇ τὸ γενόμενον·\\
Plato (0059) & Laws (034) & 891.b.4 & νόμοις οὖν διαφθειρομένοις τοῖς μεγίστοις ὑπὸ κακῶν ἀνθρώπων τίνα καὶ μᾶλλον προσήκει βοηθεῖν ἢ νομοθέτην;\\
Plato (0059) & Parmenides (009) & 139.c.4 & οὐ γὰρ ἑνὶ προσήκει ἑτέρῳ τινὸς εἶναι, ἀλλὰ μόνῳ ἑτέρῳ ἑτέρου, ἄλλῳ δὲ οὐδενί.\\
Plato (0059) & Phaedo (004) & 80.b.8 & τούτων οὕτως ἐχόντων ἆρ’ οὐχὶ σώματι μὲν ταχὺ διαλύεσθαι προσήκει, ψυχῇ δὲ αὖ τὸ παράπαν ἀδιαλύτῳ εἶναι ἢ ἐγγύς τι τούτου;\\
\addlinespace
Plato (0059) & Phaedo (004) & 88.a.1 & εἰ γάρ τις καὶ πλέον ἔτι τῷ λέγοντι ἢ ἃ σὺ λέγεις συγχωρήσειεν, δοὺς αὐτῷ μὴ μόνον ἐν τῷ πρὶν καὶ γενέσθαι ἡμᾶς χρόνῳ εἶναι ἡμῶν τὰς ψυχάς, ἀλλὰ μηδὲν κωλύειν καὶ ἐπειδὰν ἀποθάνωμεν ἐνίων ἔτι εἶναι καὶ ἔσεσθαι καὶ πολλάκις γενήσεσθαι καὶ ἀποθανεῖσθαι αὖθις—οὕτω γὰρ αὐτὸ φύσει ἰσχυρὸν εἶναι, ὥστε πολλάκις γιγνομένην ψυχὴν ἀντέχειν—δοὺς δὲ ταῦτα ἐκεῖνο μηκέτι συγχωροῖ, μὴ οὐ πονεῖν αὐτὴν ἐν ταῖς πολλαῖς γενέσεσιν καὶ τελευτῶσάν γε ἔν τινι τῶν θανάτων παντάπασιν ἀπόλλυσθαι, τοῦτον δὲ τὸν θάνατον καὶ ταύτην τὴν διάλυσιν τοῦ σώματος ἣ τῇ ψυχῇ φέρει ὄλεθρον μηδένα φαίη εἰδέναι—ἀδύνατον γὰρ εἶναι ὁτῳοῦν αἰσθέσθαι ἡμῶν—εἰ δὲ τοῦτο οὕτως ἔχει, οὐδενὶ προσήκει θάνατον θαρροῦντι μὴ οὐκ ἀνοήτως θαρρεῖν, ὃς ἂν μὴ ἔχῃ ἀποδεῖξαι ὅτι ἔστι ψυχὴ παντάπασιν ἀθάνατόν τε καὶ ἀνώλεθρον·\\
Plato (0059) & Phaedo (004) & 95.d.6 & προσήκει γὰρ φοβεῖσθαι, εἰ μὴ ἀνόητος εἴη, τῷ μὴ εἰδότι μηδὲ ἔχοντι λόγον διδόναι ὡς ἀθάνατόν ἐστι.\\
Plato (0059) & Phaedrus (012) & 233.a.4 & καὶ μὲν δὴ βελτίονί σοι προσήκει γενέσθαι ἐμοὶ πειθομένῳ ἢ ἐραστῇ.\\
Plato (0059) & Phaedrus (012) & 233.e.5 & ἀλλ’ ἴσως προσήκει οὐ τοῖς σφόδρα δεομένοις χαρίζεσθαι, ἀλλὰ τοῖς μάλιστα ἀποδοῦναι χάριν δυναμένοις·\\
Plato (0059) & Republic (030) & 441.e.4 & οὐκοῦν τῷ μὲν λογιστικῷ ἄρχειν προσήκει, σοφῷ ὄντι καὶ ἔχοντι τὴν ὑπὲρ ἁπάσης τῆς ψυχῆς προμήθειαν, τῷ δὲ θυμοειδεῖ ὑπηκόῳ εἶναι καὶ συμμάχῳ τούτου;\\
\addlinespace
Plato (0059) & Statesman (008) & 260.a.4 & τούτῳ δέ γε οἶμαι προσήκει κρίναντι μὴ τέλος ἔχειν μηδ’ ἀπηλλάχθαι, καθάπερ ὁ λογιστὴς ἀπήλλακτο, προστάττειν δὲ ἑκάστοις τῶν ἐργατῶν τό γε πρόσφορον ἕως ἂν ἀπεργάσωνται τὸ προσταχθέν.\\
Plato (0059) & Statesman (008) & 311.b.7 & τοῦτο δὴ τέλος ὑφάσματος εὐθυπλοκίᾳ συμπλακὲν γίγνεσθαι φῶμεν πολιτικῆς πράξεως τὸ τῶν ἀνδρείων καὶ σωφρόνων ἀνθρώπων ἦθος, ὁπόταν ὁμονοίᾳ καὶ φιλίᾳ κοινὸν συναγαγοῦσα αὐτῶν τὸν βίον ἡ βασιλικὴ τέχνη, πάντων μεγαλοπρεπέστατον ὑφασμάτων καὶ ἄριστον ἀποτελέσασα τούς τ’ ἄλλους ἐν ταῖς πόλεσι πάντας δούλους καὶ ἐλευθέρους ἀμπίσχουσα, συνέχῃ τούτῳ τῷ πλέγματι, καὶ καθ’ ὅσον εὐδαίμονι προσήκει γίγνεσθαι πόλει τούτου μηδαμῇ μηδὲν ἐλλείπουσα ἄρχῃ τε καὶ ἐπιστατῇ.\\
Plato (0059) & Timaeus (031) & 29.b.3 & τοῦ μὲν οὖν μονίμου καὶ βεβαίου καὶ μετὰ νοῦ καταφανοῦς μονίμους καὶ ἀμεταπτώτους—καθ’ ὅσον οἷόν τε καὶ ἀνελέγκτοις προσήκει λόγοις εἶναι καὶ ἀνικήτοις, τούτου δεῖ μηδὲν ἐλλείπειν—τοὺς δὲ τοῦ πρὸς μὲν ἐκεῖνο ἀπεικασθέντος, ὄντος δὲ εἰκόνος εἰκότας ἀνὰ λόγον τε ἐκείνων ὄντας·\\
Plato (0059) & Charmides (018) & 173.b.4 & ἐκ δὴ τούτων οὕτως ἐχόντων ἄλλο ἂν ἡμῖν τι συμβαίνοι ἢ ὑγιέσιν τε τὰ σώματα εἶναι μᾶλλον ἢ νῦν, καὶ ἐν τῇ θαλάττῃ κινδυνεύοντας καὶ ἐν πολέμῳ σῴζεσθαι, καὶ τὰ σκεύη καὶ τὴν ἀμπεχόνην καὶ ὑπόδεσιν πᾶσαν καὶ τὰ χρήματα πάντα τεχνικῶς ἡμῖν εἰργασμένα εἶναι καὶ ἄλλα πολλὰ διὰ τὸ ἀληθινοῖς δημιουργοῖς χρῆσθαι;\\
Plato (0059) & Laws (034) & 635.a.7 & οὐ γὰρ τό γε γνῶναί τι τῶν μὴ καλῶν ἄτιμον, ἀλλὰ ἴασιν ἐξ αὐτοῦ συμβαίνει γίγνεσθαι τῷ μὴ φθόνῳ τὰ λεγόμενα ἀλλ’ εὐνοίᾳ δεχομένῳ.\\
\addlinespace
Plato (0059) & Laws (034) & 850.b.6 & ἐὰν δ’ ἐν τοῖς ἔτεσι τούτοις αὐτῷ συμβῇ λόγου ἀξίῳ πρὸς εὐεργεσίαν τῆς πόλεως γεγονέναι τινὰ ἱκανήν, καὶ πιστεύῃ πείσειν βουλὴν καὶ ἐκκλησίαν, ἤ τινα ἀναβολὴν τῆς ἐξοικήσεως ἀξιῶν αὑτῷ γίγνεσθαι κυρίως, ἢ καὶ τὸ παράπαν διὰ βίου τινὰ μονήν, ἐπελθὼν καὶ πείσας τὴν πόλιν, ἅπερ ἂν πείσῃ, ταῦτα αὐτῷ τέλεα γιγνέσθω.\\
Plato (0059) & Phaedo (004) & 68.e.5 & καίτοι καλοῦσί γε ἀκολασίαν τὸ ὑπὸ τῶν ἡδονῶν ἄρχεσθαι, ἀλλ’ ὅμως συμβαίνει αὐτοῖς κρατουμένοις ὑφ’ ἡδονῶν κρατεῖν ἄλλων ἡδονῶν.\\
Plato (0059) & Protagoras (022) & 346.a.3 & τοὺς μὲν οὖν πονηρούς, ὅταν τοιοῦτόν τι αὐτοῖς συμβῇ, ὥσπερ ἁσμένους ὁρᾶν καὶ ψέγοντας ἐπιδεικνύναι καὶ κατηγορεῖν τὴν πονηρίαν τῶν γονέων ἢ πατρίδος, ἵνα αὐτοῖς ἀμελοῦσιν αὐτῶν μὴ ἐγκαλῶσιν οἱ ἄνθρωποι μηδ’ ὀνειδίζωσιν ὅτι ἀμελοῦσιν, ὥστε ἔτι μᾶλλον ψέγειν τε αὐτοὺς καὶ ἔχθρας ἑκουσίους πρὸς ταῖς ἀναγκαίαις προστίθεσθαι·\\
Plato (0059) & Theaetetus (006) & 183.a.2 & καλὸν ἂν ἡμῖν συμβαίνοι τὸ ἐπανόρθωμα τῆς ἀποκρίσεως, προθυμηθεῖσιν ἀποδεῖξαι ὅτι πάντα κινεῖται, ἵνα δὴ ἐκείνη ἡ ἀπόκρισις ὀρθὴ φανῇ.\\
Plato (0059) & Republic (030) & 344.a.2 & τοῦτον οὖν σκόπει, εἴπερ βούλει κρίνειν ὅσῳ μᾶλλον συμφέρει ἰδίᾳ αὑτῷ ἄδικον εἶναι ἢ τὸ δίκαιον.\\
\addlinespace
Plato (0059) & Gorgias (023) & 485.d.3 & ὃ γὰρ νυνδὴ ἔλεγον, ὑπάρχει τούτῳ τῷ ἀνθρώπῳ, κἂν πάνυ εὐφυὴς ᾖ, ἀνάνδρῳ γενέσθαι φεύγοντι τὰ μέσα τῆς πόλεως καὶ τὰς ἀγοράς, ἐν αἷς ἔφη ὁ ποιητὴς τοὺς ἄνδρας ἀριπρεπεῖς γίγνεσθαι, καταδεδυκότι δὲ τὸν λοιπὸν βίον βιῶναι μετὰ μειρακίων ἐν γωνίᾳ τριῶν ἢ τεττάρων ψιθυρίζοντα, ἐλεύθερον δὲ καὶ μέγα καὶ ἱκανὸν μηδέποτε φθέγξασθαι.\\
Plato (0059) & Gorgias (023) & 492.a.4 & ἐπεὶ ὅσοις ἐξ ἀρχῆς ὑπῆρξεν ἢ βασιλέων ὑέσιν εἶναι ἢ αὐτοὺς τῇ φύσει ἱκανοὺς ἐκπορίσασθαι ἀρχήν τινα ἢ τυραννίδα ἢ δυναστείαν, τί ἂν τῇ ἀληθείᾳ αἴσχιον καὶ κάκιον εἴη σωφροσύνης καὶ δικαιοσύνης τούτοις τοῖς ἀνθρώποις, οἷς ἐξὸν ἀπολαύειν τῶν ἀγαθῶν καὶ μηδενὸς ἐμποδὼν ὄντος, αὐτοὶ ἑαυτοῖς δεσπότην ἐπαγάγοιντο τὸν τῶν πολλῶν ἀνθρώπων νόμον τε καὶ λόγον καὶ ψόγον;\\
Plato (0059) & Hippias Major (025) & 292.c.9 & οὐχ οἷός τ’ εἶ μεμνῆσθαι ὅτι τὸ καλὸν αὐτὸ ἠρώτων, ὃ παντὶ ᾧ ἂν προσγένηται, ὑπάρχει ἐκείνῳ καλῷ εἶναι, καὶ λίθῳ καὶ ξύλῳ καὶ ἀνθρώπῳ καὶ θεῷ καὶ πάσῃ πράξει καὶ παντὶ μαθήματι;\\
Plato (0059) & Ion (027) & 542.b.3 & τοῦτο τοίνυν τὸ κάλλιον ὑπάρχει σοι παρ’ ἡμῖν, ὦ Ἴων, θεῖον εἶναι καὶ μὴ τεχνικὸν περὶ Ὁμήρου ἐπαινέτην.\\
Plato (0059) & Laches (019) & 186.b.5 & εἰ δὲ μηδὲν ἡμῖν τούτων ὑπάρχει, ἄλλους κελεύειν ζητεῖν καὶ μὴ ἐν ἑταίρων ἀνδρῶν ὑέσιν κινδυνεύειν διαφθείροντας τὴν μεγίστην αἰτίαν ἔχειν ὑπὸ τῶν οἰκειοτάτων.\\
\addlinespace
Plato (0059) & Phaedo (004) & 81.a.4 & οὐκοῦν οὕτω μὲν ἔχουσα εἰς τὸ ὅμοιον αὐτῇ τὸ ἀιδὲς ἀπέρχεται, τὸ θεῖόν τε καὶ ἀθάνατον καὶ φρόνιμον, οἷ ἀφικομένῃ ὑπάρχει αὐτῇ εὐδαίμονι εἶναι, πλάνης καὶ ἀνοίας καὶ φόβων καὶ ἀγρίων ἐρώτων καὶ τῶν ἄλλων κακῶν τῶν ἀνθρωπείων ἀπηλλαγμένῃ, ὥσπερ δὲ λέγεται κατὰ τῶν μεμυημένων, ὡς ἀληθῶς τὸν λοιπὸν χρόνον μετὰ θεῶν διάγουσα;\\
Plato (0059) & Phaedrus (012) & 240.a.9 & ἔστι μὲν δὴ καὶ ἄλλα κακά, ἀλλά τις δαίμων ἔμειξε τοῖς πλείστοις ἐν τῷ παραυτίκα ἡδονήν, οἷον κόλακι, δεινῷ θηρίῳ καὶ βλάβῃ μεγάλῃ, ὅμως ἐπέμειξεν ἡ φύσις ἡδονήν τινα οὐκ ἄμουσον, καί τις ἑταίραν ὡς βλαβερὸν ψέξειεν ἄν, καὶ ἄλλα πολλὰ τῶν τοιουτοτρόπων θρεμμάτων τε καὶ ἐπιτηδευμάτων, οἷς τό γε καθ’ ἡμέραν ἡδίστοισιν εἶναι ὑπάρχει·\\
Plato (0059) & Phaedrus (012) & 269.d.2 & εἰ μέν σοι ὑπάρχει φύσει ῥητορικῷ εἶναι, ἔσῃ ῥήτωρ ἐλλόγιμος, προσλαβὼν ἐπιστήμην τε καὶ μελέτην, ὅτου δ’ ἂν ἐλλείπῃς τούτων, ταύτῃ ἀτελὴς ἔσῃ.\\
Plato (0059) & Protagoras (022) & 345.a.5 & δῆλον ὅτι ᾧ πρῶτον μὲν ὑπάρχει ἰατρῷ εἶναι, ἔπειτα ἀγαθῷ ἰατρῷ—οὗτος γὰρ ἂν καὶ κακὸς γένοιτο—ἡμεῖς δὲ οἱ ἰατρικῆς ἰδιῶται οὐκ ἄν ποτε γενοίμεθα κακῶς πράξαντες οὔτε ἰατροὶ οὔτε τέκτονες οὔτε ἄλλο οὐδὲν τῶν τοιούτων·\\
Plato (0059) & Republic (030) & 366.b.7 & ἐκ δὴ πάντων τῶν εἰρημένων τίς μηχανή, ὦ Σώκρατες, δικαιοσύνην τιμᾶν ἐθέλειν ᾧ τις δύναμις ὑπάρχει ψυχῆς ἢ χρημάτων ἢ σώματος ἢ γένους, ἀλλὰ μὴ γελᾶν ἐπαινουμένης ἀκούοντα;\\
\addlinespace
Plato (0059) & Republic (030) & 464.d.9 & ὅθεν δὴ ὑπάρχει τούτοις ἀστασιάστοις εἶναι, ὅσα γε διὰ χρημάτων ἢ παίδων καὶ συγγενῶν κτῆσιν ἄνθρωποι στασιάζουσιν;\\
Plato (0059) & Republic (030) & 586.e.4 & τῷ φιλοσόφῳ ἄρα ἑπομένης ἁπάσης τῆς ψυχῆς καὶ μὴ στασιαζούσης ἑκάστῳ τῷ μέρει ὑπάρχει εἴς τε τἆλλα τὰ ἑαυτοῦ πράττειν καὶ δικαίῳ εἶναι, καὶ δὴ καὶ τὰς ἡδονὰς τὰς ἑαυτοῦ ἕκαστον καὶ τὰς βελτίστας καὶ εἰς τὸ δυνατὸν τὰς ἀληθεστάτας καρποῦσθαι.\\
Plato (0059) & Symposium (011) & 212.a.5 & τεκόντι δὲ ἀρετὴν ἀληθῆ καὶ θρεψαμένῳ ὑπάρχει θεοφιλεῖ γενέσθαι, καὶ εἴπέρ τῳ ἄλλῳ ἀνθρώπων ἀθανάτῳ καὶ ἐκείνῳ;\\
Plato (0059) & Republic (030) & 360.b.3 & εἰ οὖν δύο τοιούτω δακτυλίω γενοίσθην, καὶ τὸν μὲν ὁ δίκαιος περιθεῖτο, τὸν δὲ ὁ ἄδικος, οὐδεὶς ἂν γένοιτο, ὡς δόξειεν, οὕτως ἀδαμάντινος, ὃς ἂν μείνειεν ἐν τῇ δικαιοσύνῃ καὶ τολμήσειεν ἀπέχεσθαι τῶν ἀλλοτρίων καὶ μὴ ἅπτεσθαι, ἐξὸν αὐτῷ καὶ ἐκ τῆς ἀγορᾶς ἀδεῶς ὅτι βούλοιτο λαμβάνειν, καὶ εἰσιόντι εἰς τὰς οἰκίας συγγίγνεσθαι ὅτῳ βούλοιτο, καὶ ἀποκτεινύναι καὶ ἐκ δεσμῶν λύειν οὕστινας βούλοιτο, καὶ τἆλλα πράττειν ἐν τοῖς ἀνθρώποις ἰσόθεον ὄντα.\\
Plato (0059) & Republic (030) & 405.a.6 & τῆς δὲ κακῆς τε καὶ αἰσχρᾶς παιδείας ἐν πόλει ἆρα μή τι μεῖζον ἕξεις λαβεῖν τεκμήριον ἢ τὸ δεῖσθαι ἰατρῶν καὶ δικαστῶν ἄκρων μὴ μόνον τοὺς φαύλους τε καὶ χειροτέχνας, ἀλλὰ καὶ τοὺς ἐν ἐλευθέρῳ σχήματι προσποιουμένους τεθράφθαι;\\
\addlinespace
Plato (0059) & Republic (030) & 416.d.7 & τὰ δ’ ἐπιτήδεια, ὅσων δέονται ἄνδρες ἀθληταὶ πολέμου σώφρονές τε καὶ ἀνδρεῖοι, ταξαμένους παρὰ τῶν ἄλλων πολιτῶν δέχεσθαι μισθὸν τῆς φυλακῆς τοσοῦτον ὅσον μήτε περιεῖναι αὐτοῖς εἰς τὸν ἐνιαυτὸν μήτε ἐνδεῖν·\\
Plato (0059) & Laws (034) & 925.e.6 & ἔστω τοίνυν εἰρημένον ὑπέρ τε νομοθέτου καὶ ὑπὲρ νομοθετουμένου σχεδὸν οἷον κοινὸν προοίμιον, συγγνώμην μὲν τῷ νομοθέτῃ τοὺς ἐπιταττομένους δεόμενον ἔχειν, ὅτι τῶν κοινῶν ἐπιμελούμενος οὐκ ἄν ποτε δύναιτο διοικεῖν ἅμα καὶ τὰς ἰδίας ἑκάστῳ γιγνομένας συμφοράς, συγγνώμην δ’ αὖ καὶ τοῖς νομοθετουμένοις, ὡς τὰ τοῦ νομοθετοῦντος εἰκότως ἐνίοτε οὐ δύνανται προστάγματα τελεῖν, ἃ μὴ γιγνώσκων προστάττει.\\
Plato (0059) & Euthydemus (021) & 282.a.7 & —καὶ παρὰ πατρός γε δήπου τοῦτο οἰόμενον δεῖν παραλαμβάνειν πολὺ μᾶλλον ἢ χρήματα, καὶ παρ’ ἐπιτρόπων καὶ φίλων τῶν τε ἄλλων καὶ τῶν φασκόντων ἐραστῶν εἶναι, καὶ ξένων καὶ πολιτῶν, δεόμενον καὶ ἱκετεύοντα σοφίας μεταδιδόναι, οὐδὲν αἰσχρόν, ὦ Κλεινία, οὐδὲ νεμεσητὸν ἕνεκα τούτου ὑπηρετεῖν καὶ δουλεύειν καὶ ἐραστῇ καὶ παντὶ ἀνθρώπῳ, ὁτιοῦν ἐθέλοντα ὑπηρετεῖν τῶν καλῶν ὑπηρετημάτων, προθυμούμενον σοφὸν γενέσθαι·\\
Plato (0059) & Gorgias (023) & 527.d.5 & αἰσχρὸν γὰρ ἔχοντάς γε ὡς νῦν φαινόμεθα ἔχειν, ἔπειτα νεανιεύεσθαι ὡς τὶ ὄντας, οἷς οὐδέποτε ταὐτὰ δοκεῖ περὶ τῶν αὐτῶν, καὶ ταῦτα περὶ τῶν μεγίστων—εἰς τοσοῦτον ἥκομεν ἀπαιδευσίας—ὥσπερ οὖν ἡγεμόνι τῷ λόγῳ χρησώμεθα τῷ νῦν παραφανέντι, ὃς ἡμῖν σημαίνει ὅτι οὗτος ὁ τρόπος ἄριστος τοῦ βίου, καὶ τὴν δικαιοσύνην καὶ τὴν ἄλλην ἀρετὴν ἀσκοῦντας καὶ ζῆν καὶ τεθνάναι.\\
Plato (0059) & Charmides (018) & 176.a.2 & εἰ γὰρ ἔχεις, μᾶλλον ἂν ἔγωγέ σοι συμβουλεύσαιμι ἐμὲ μὲν λῆρον ἡγεῖσθαι εἶναι καὶ ἀδύνατον λόγῳ ὁτιοῦν ζητεῖν, σεαυτὸν δέ, ὅσῳπερ σωφρονέστερος εἶ, τοσούτῳ εἶναι καὶ εὐδαιμονέστερον.\\
\addlinespace
Plato (0059) & Epistles (036) & 323.a.5 & ἔχεσθαι δὴ παντὶ συμβουλεύω δικαίῳ τρόπῳ τούτων τῶν ἀνδρῶν, μὴ πάρεργον ἡγουμένῳ.\\
Plato (0059) & Epistles (036) & 332.c.6 & ἃ δὴ καὶ Διονυσίῳ συνεβουλεύομεν ἐγὼ καὶ δίων, ἐπειδὴ τὰ παρὰ τοῦ πατρὸς αὐτῷ συνεβεβήκει οὕτως, ἀνομιλήτῳ μὲν παιδείας, ἀνομιλήτῳ δὲ συνουσιῶν τῶν προσηκουσῶν γεγονέναι, πρῶτον ἔπειτα ταύτῃ ὁρμήσαντα φίλους ἄλλους αὑτῷ τῶν οἰκείων ἅμα καὶ ἡλικιωτῶν καὶ συμφώνους πρὸς ἀρετὴν κτήσασθαι, μάλιστα δ’ αὐτὸν αὑτῷ, τούτου γὰρ αὐτὸν θαυμαστῶς ἐνδεᾶ γεγονέναι, λέγοντες οὐκ ἐναργῶς οὕτως—οὐ γὰρ ἦν ἀσφαλές—αἰνιττόμενοι δὲ καὶ διαμαχόμενοι τοῖς λόγοις ὡς οὕτω μὲν πᾶς ἀνὴρ αὑτόν τε καὶ ἐκείνους ὧν ἂν ἡγεμὼν γίγνηται σώσει, μὴ ταύτῃ δὲ τραπόμενος τἀναντία πάντα ἀποτελεῖ·\\
Plato (0059) & Republic (030) & 390.e.4 & οὐδὲ τὸν τοῦ Ἀχιλλέως παιδαγωγὸν Φοίνικα ἐπαινετέον ὡς μετρίως ἔλεγε συμβουλεύων αὐτῷ δῶρα μὲν λαβόντι ἐπαμύνειν τοῖς Ἀχαιοῖς, ἄνευ δὲ δώρων μὴ ἀπαλλάττεσθαι τῆς μήνιος.\\
Plato (0059) & Symposium (011) & 176.d.2 & καὶ οὔτε αὐτὸς ἑκὼν εἶναι πόρρω ἐθελήσαιμι ἂν πιεῖν οὔτε ἄλλῳ συμβουλεύσαιμι, ἄλλως τε καὶ κραιπαλῶντα ἔτι ἐκ τῆς προτεραίας.\\
Sophocles (0011) & Electra (005) & 335 & νῦν δ’ ἐν κακοῖς μοι πλεῖν ὑφειμένῃ δοκεῖ, καὶ μὴ δοκεῖν μὲν δρᾶν τι, πημαίνειν δὲ μή·\\
\addlinespace
Sophocles (0011) & Oedipus Tyrannus (004) & 404 & ἡμῖν μὲν εἰκάζουσι καὶ τὰ τοῦδ’ ἔπη ὀργῇ λελέχθαι καὶ τά σ’, Οἰδίπους, δοκεῖ, δεῖ δ’ οὐ τοιούτων, ἀλλ’ ὅπως τὰ τοῦ θεοῦ μαντεῖ’ ἄριστα λύσομεν, τόδε σκοπεῖν.\\
Sophocles (0011) & Philoctetes (006) & 1273 & βούλομαι δέ σου κλύειν, πότερα δέδοκταί σοι μένοντι καρτερεῖν ἢ πλεῖν μεθ’ ἡμῶν;\\
Sophocles (0011) & Electra (005) & 1 & ὦ τοῦ στρατηγήσαντος ἐν Τροίᾳ ποτὲ Ἀγαμέμνονος παῖ, νῦν ἐκεῖν’ ἔξεστί σοι παρόντι λεύσσειν, ὧν πρόθυμος ἦσθ’ ἀεί.\\
Thucydides (0003) & History (001) & 1.31.2.1 & πυνθανόμενοι δὲ οἱ Κερκυραῖοι τὴν παρασκευὴν αὐτῶν ἐφοβοῦντο, καί (ἦσαν γὰρ οὐδενὸς Ἑλλήνων ἔνσπονδοι οὐδὲ ἐσεγράψαντο ἑαυτοὺς οὔτε ἐς τὰς Ἀθηναίων σπονδὰς οὔτε ἐς τὰς Λακεδαιμονίων) ἔδοξεν αὐτοῖς ἐλθοῦσιν ὡς τοὺς Ἀθηναίους ξυμμάχους γενέσθαι καὶ ὠφελίαν τινὰ πειρᾶσθαι ἀπ’ αὐτῶν εὑρίσκεσθαι.\\
Thucydides (0003) & History (001) & 1.36.1.1 & καὶ ὅτῳ τάδε ξυμφέροντα μὲν δοκεῖ λέγεσθαι, φοβεῖται δὲ μὴ δι’ αὐτὰ πειθόμενος τὰς σπονδὰς λύσῃ, γνώτω τὸ μὲν δεδιὸς αὐτοῦ ἰσχὺν ἔχον τοὺς ἐναντίους μᾶλλον φοβῆσον, τὸ δὲ θαρσοῦν μὴ δεξαμένου ἀσθενὲς ὂν πρὸς ἰσχύοντας τοὺς ἐχθροὺς ἀδεέστερον ἐσόμενον, καὶ ἅμα οὐ περὶ τῆς Κερκύρας νῦν τὸ πλέον ἢ καὶ τῶν Ἀθηνῶν βουλευόμενος, καὶ οὐ τὰ κράτιστα αὐταῖς προνοῶν, ὅταν ἐς τὸν μέλλοντα καὶ ὅσον οὐ παρόντα πόλεμον τὸ αὐτίκα περισκοπῶν ἐνδοιάζῃ χωρίον προσλαβεῖν ὃ μετὰ μεγίστων καιρῶν οἰκειοῦταί τε καὶ πολεμοῦται.\\
\addlinespace
Thucydides (0003) & History (001) & 1.72.1.1 & τῶν δὲ Ἀθηναίων ἔτυχε γὰρ πρεσβεία πρότερον ἐν τῇ Λακεδαίμονι περὶ ἄλλων παροῦσα, καὶ ὡς ᾔσθοντο τῶν λόγων, ἔδοξεν αὐτοῖς παριτητέα ἐς τοὺς Λακεδαιμονίους εἶναι, τῶν μὲν ἐγκλημάτων πέρι μηδὲν ἀπολογησομένους ὧν αἱ πόλεις ἐνεκάλουν, δηλῶσαι δὲ περὶ τοῦ παντὸς ὡς οὐ ταχέως αὐτοῖς βουλευτέον εἴη, ἀλλ’ ἐν πλέονι σκεπτέον.\\
Thucydides (0003) & History (001) & 1.75.4.1 & καὶ οὐκ ἀσφαλὲς ἔτι ἐδόκει εἶναι τοῖς πολλοῖς ἀπηχθημένους, καί τινων καὶ ἤδη ἀποστάντων κατεστραμμένων, ὑμῶν τε ἡμῖν οὐκέτι ὁμοίως φίλων, ἀλλ’ ὑπόπτων καὶ διαφόρων ὄντων, ἀνέντας κινδυνεύειν·\\
Thucydides (0003) & History (001) & 1.107.4.1 & ἔδοξε δ’ αὐτοῖς ἐν Βοιωτοῖς περιμείνασι σκέψασθαι ὅτῳ τρόπῳ ἀσφαλέστατα διαπορεύσονται.\\
Thucydides (0003) & History (001) & 1.125.1.3 & δεδογμένον δὲ αὐτοῖς εὐθὺς μὲν ἀδύνατα ἦν ἐπιχειρεῖν ἀπαρασκεύοις οὖσιν, ἐκπορίζεσθαι δὲ ἐδόκει ἑκάστοις ἃ πρόσφορα ἦν καὶ μὴ εἶναι μέλλησιν.\\
Thucydides (0003) & History (001) & 2.35.1.3 & ἐμοὶ δὲ ἀρκοῦν ἂν ἐδόκει εἶναι ἀνδρῶν ἀγαθῶν ἔργῳ γενομένων ἔργῳ καὶ δηλοῦσθαι τὰς τιμάς, οἷα καὶ νῦν περὶ τὸν τάφον τόνδε δημοσίᾳ παρασκευασθέντα ὁρᾶτε, καὶ μὴ ἐν ἑνὶ ἀνδρὶ πολλῶν ἀρετὰς κινδυνεύεσθαι εὖ τε καὶ χεῖρον εἰπόντι πιστευθῆναι.\\
\addlinespace
Thucydides (0003) & History (001) & 4.15.1.1 & ἐς δὲ τὴν Σπάρτην ὡς ἠγγέλθη τὰ γεγενημένα περὶ Πύλον, ἔδοξεν αὐτοῖς ὡς ἐπὶ ξυμφορᾷ μεγάλῃ τὰ τέλη καταβάντας ἐς τὸ στρατόπεδον βουλεύειν παραχρῆμα ὁρῶντας ὅτι ἂν δοκῇ.\\
Thucydides (0003) & History (001) & 4.15.1.1 & καὶ ὡς εἶδον ἀδύνατον ὂν τιμωρεῖν τοῖς ἀνδράσι καὶ κινδυνεύειν οὐκ ἐβούλοντο ἢ ὑπὸ λιμοῦ τι παθεῖν αὐτοὺς ἢ ὑπὸ πλήθους βιασθέντας κρατηθῆναι, ἔδοξεν αὐτοῖς πρὸς τοὺς στρατηγοὺς τῶν Ἀθηναίων, ἢν ἐθέλωσι, σπονδὰς ποιησαμένους τὰ περὶ Πύλον ἀποστεῖλαι ἐς τὰς Ἀθήνας πρέσβεις περὶ ξυμβάσεως καὶ τοὺς ἄνδρας ὡς τάχιστα πειρᾶσθαι κομίσασθαι.\\
Thucydides (0003) & History (001) & 4.22.3.1 & ὁρῶντες δὲ οἱ Λακεδαιμόνιοι οὔτε σφίσιν οἷόν τε ὂν ἐν πλήθει εἰπεῖν, εἴ τι καὶ ὑπὸ τῆς ξυμφορᾶς ἐδόκει αὐτοῖς ξυγχωρεῖν, μὴ ἐς τοὺς ξυμμάχους διαβληθῶσιν εἰπόντες καὶ οὐ τυχόντες, οὔτε τοὺς Ἀθηναίους ἐπὶ μετρίοις ποιήσοντας ἃ προυκαλοῦντο, ἀνεχώρησαν ἐκ τῶν Ἀθηνῶν ἄπρακτοι.\\
Thucydides (0003) & History (001) & 4.71.1.1 & αἱ δὲ τῶν Μεγαρέων στάσεις φοβούμεναι, οἱ μὲν μὴ τοὺς φεύγοντας σφίσιν ἐσαγαγὼν αὐτοὺς ἐκβάλῃ, οἱ δὲ μὴ αὐτὸ τοῦτο ὁ δῆμος δείσας ἐπίθηται σφίσι καὶ ἡ πόλις ἐν μάχῃ καθ’ αὑτὴν οὖσα ἐγγὺς ἐφεδρευόντων Ἀθηναίων ἀπόληται, οὐκ ἐδέξαντο, ἀλλ’ ἀμφοτέροις ἐδόκει ἡσυχάσασι τὸ μέλλον περιιδεῖν.\\
Thucydides (0003) & History (001) & 4.92.2.1 & νυνὶ δ’ εἴ τῳ καὶ ἀσφαλέστερον ἔδοξεν εἶναι, μεταγνώτω.\\
\addlinespace
Thucydides (0003) & History (001) & 4.118.4.2 & τάδε δὲ ἔδοξε Λακεδαιμονίοις καὶ τοῖς ἄλλοις ξυμμάχοις ἐὰν σπονδὰς ποιῶνται οἱ Ἀθηναῖοι, ἐπὶ τῆς αὐτῶν μένειν ἑκατέρους ἔχοντας ἅπερ νῦν ἔχομεν, τοὺς μὲν ἐν τῷ Κορυφασίῳ ἐντὸς τῆς Βουφράδος καὶ τοῦ Τομέως μένοντας, τοὺς δὲ ἐν Κυθήροις μὴ ἐπιμισγομένους ἐς τὴν ξυμμαχίαν, μήτε ἡμᾶς πρὸς αὐτοὺς μήτε αὐτοὺς πρὸς ἡμᾶς, τοὺς δ’ ἐν Νισαίᾳ καὶ Μινῴᾳ μὴ ὑπερβαίνοντας τὴν ὁδὸν τὴν ἀπὸ τῶν πυλῶν τῶν παρὰ τοῦ Νίσου ἐπὶ τὸ Ποσειδώνιον, ἀπὸ δὲ τοῦ Ποσειδωνίου εὐθὺς ἐπὶ τὴν γέφυραν τὴν ἐς Μινῴαν (μηδὲ Μεγαρέας καὶ τοὺς ξυμμάχους ὑπερβαίνειν τὴν ὁδὸν ταύτην) καὶ τὴν νῆσον, ἥνπερ ἔλαβον οἱ Ἀθηναῖοι, ἔχοντας, μηδὲ ἐπιμισγομένους μηδετέρους μηδετέρωσε, καὶ τὰ ἐν Τροιζῆνι, ὅσαπερ νῦν ἔχουσι, καθ’ ἃ ξυνέθεντο πρὸς Ἀθηναίους·\\
Thucydides (0003) & History (001) & 5.77.1.1 & καττάδε δοκεῖ τᾷ ἐκκλησίᾳ τῶν Λακεδαιμονίων ξυμβαλέσθαι ποττὼς Ἀργείως, ἀποδιδόντας τὼς παῖδας τοῖς Ὀρχομενίοις καὶ τὼς ἄνδρας τοῖς Μαιναλίοις, καὶ τὼς ἄνδρας τὼς ἐν Μαντινείᾳ τοῖς Λακεδαιμονίοις ἀποδιδόντας, καὶ ἐξ Ἐπιδαύρω ἐκβῶντας καὶ τὸ τεῖχος ἀναιρίοντας.\\
Thucydides (0003) & History (001) & 6.9.1.2 & ἐμοὶ μέντοι δοκεῖ καὶ περὶ αὐτοῦ τούτου ἔτι χρῆναι σκέψασθαι, εἰ ἄμεινόν ἐστιν ἐκπέμπειν τὰς ναῦς, καὶ μὴ οὕτω βραχείᾳ βουλῇ περὶ μεγάλων πραγμάτων ἀνδράσιν ἀλλοφύλοις πειθομένους πόλεμον οὐ προσήκοντα ἄρασθαι.\\
Thucydides (0003) & History (001) & 7.40.4.2 & ἔπειτα οὐκ ἐδόκει τοῖς Ἀθηναίοις ὑπὸ σφῶν αὐτῶν διαμέλλοντας κόπῳ ἁλίσκεσθαι, ἀλλ’ ἐπιχειρεῖν ὅτι τάχιστα, καὶ ἐπιφερόμενοι ἐκ παρακελεύσεως ἐναυμάχουν.\\
Thucydides (0003) & History (001) & 7.80.1.1 & τῆς δὲ νυκτὸς τῷ Νικίᾳ καὶ Δημοσθένει ἐδόκει, ἐπειδὴ κακῶς σφίσι τὸ στράτευμα εἶχε τῶν τε ἐπιτηδείων πάντων ἀπορίᾳ ἤδη, καὶ κατατετραυματισμένοι ἦσαν πολλοὶ ἐν πολλαῖς προσβολαῖς τῶν πολεμίων γεγενημέναις, πυρὰ καύσαντας ὡς πλεῖστα ἀπάγειν τὴν στρατιάν, μηκέτι τὴν αὐτὴν ὁδὸν ᾗ διενοήθησαν, ἀλλὰ τοὐναντίον ἢ οἱ Συρακόσιοι ἐτήρουν, πρὸς τὴν θάλασσαν.\\
\addlinespace
Thucydides (0003) & History (001) & 8.8.2.1 & ὁ δὲ Ἆγις ἐπειδὴ ἑώρα τοὺς Λακεδαιμονίους ἐς τὴν Χίον πρῶτον ὡρμημένους, οὐδ’ αὐτὸς ἄλλο τι ἐγίγνωσκεν, ἀλλὰ ξυνελθόντες ἐς Κόρινθον οἱ ξύμμαχοι ἐβουλεύοντο, καὶ ἔδοξε πρῶτον ἐς Χίον αὐτοῖς πλεῖν ἄρχοντα ἔχοντας Χαλκιδέα, ὃς ἐν τῇ Λακωνικῇ τὰς πέντε ναῦς παρεσκεύαζεν, ἔπειτα ἐς Λέσβον καὶ Ἀλκαμένη ἄρχοντα, ὅνπερ καὶ Ἆγις διενοεῖτο, τὸ τελευταῖον δὲ ἐς τὸν Ἑλλήσποντον ἀφικέσθαι (προσετέτακτο δὲ ἐς αὐτὸν ἄρχων Κλέαρχος ὁ Ῥαμφίου), διαφέρειν δὲ τὸν Ἰσθμὸν τὰς ἡμισείας τῶν νεῶν πρῶτον, καὶ εὐθὺς ταύτας ἀποπλεῖν, ὅπως μὴ οἱ Ἀθηναῖοι πρὸς τὰς ἀφορμωμένας μᾶλλον τὸν νοῦν ἔχωσιν ἢ τὰς ὕστερον ἐπιδιαφερομένας.\\
Thucydides (0003) & History (001) & 8.11.2.3 & καὶ ὁρῶντες τὴν φυλακὴν ἐν χωρίῳ ἐρήμῳ ἐπίπονον οὖσαν ἠπόρουν, καὶ ἐπενόησαν μὲν κατακαῦσαι τὰς ναῦς, ἔπειτα δὲ ἔδοξεν αὐτοῖς ἀνελκύσαι καὶ τῷ πεζῷ προσκαθημένους φυλακὴν ἔχειν, ἕως ἄν τις παρατύχῃ διαφυγὴ ἐπιτηδεία.\\
Thucydides (0003) & History (001) & 1.37.5.1 & καίτοι εἰ ἦσαν ἄνδρες, ὥσπερ φασίν, ἀγαθοί, ὅσῳ ἀληπτότεροι ἦσαν τοῖς πέλας, τόσῳ δὲ φανερωτέραν ἐξῆν αὐτοῖς τὴν ἀρετὴν διδοῦσι καὶ δεχομένοις τὰ δίκαια δεικνύναι.\\
Thucydides (0003) & History (001) & 3.39.6.3 & πάντες γὰρ ὑμῖν γε ὁμοίως ἐπέθεντο, οἷς γ’ ἐξῆν ὡς ἡμᾶς τραπομένοις νῦν πάλιν ἐν τῇ πόλει εἶναι·\\
Thucydides (0003) & History (001) & 4.17.4.1 & ὑμῖν γὰρ εὐτυχίαν τὴν παροῦσαν ἔξεστι καλῶς θέσθαι, ἔχουσι μὲν ὧν κρατεῖτε, προσλαβοῦσι δὲ τιμὴν καὶ δόξαν, καὶ μὴ παθεῖν ὅπερ οἱ ἀήθως τι ἀγαθὸν λαμβάνοντες τῶν ἀνθρώπων·\\
\addlinespace
Thucydides (0003) & History (001) & 4.20.3.1 & ἤν τε γνῶτε, Λακεδαιμονίοις ἔξεστιν ὑμῖν φίλους γενέσθαι βεβαίως, αὐτῶν τε προκαλεσαμένων χαρισαμένοις τε μᾶλλον ἢ βιασαμένοις.\\
Thucydides (0003) & History (001) & 8.53.1.1 & οἱ δὲ μετὰ τοῦ Πεισάνδρου πρέσβεις τῶν Ἀθηναίων ἀποσταλέντες ἐκ τῆς Σάμου ἀφικόμενοι ἐς τὰς Ἀθήνας λόγους ἐποιοῦντο ἐν τῷ δήμῳ κεφαλαιοῦντες ἐκ πολλῶν, μάλιστα δὲ ὡς ἐξείη αὐτοῖς Ἀλκιβιάδην καταγαγοῦσι καὶ μὴ τὸν αὐτὸν τρόπον δημοκρατουμένοις βασιλέα τε ξύμμαχον ἔχειν καὶ Πελοποννησίων περιγενέσθαι.\\
Thucydides (0003) & History (001) & 3.58.1.1 & καίτοι ἀξιοῦμέν γε καὶ θεῶν ἕνεκα τῶν ξυμμαχικῶν ποτὲ γενομένων καὶ τῆς ἀρετῆς τῆς ἐς τοὺς Ἕλληνας καμφθῆναι ὑμᾶς καὶ μεταγνῶναι εἴ τι ὑπὸ Θηβαίων ἐπείσθητε, τήν τε δωρεὰν ἀνταπαιτῆσαι αὐτοὺς μὴ κτείνειν οὓς μὴ ὑμῖν πρέπει, σώφρονά τε ἀντὶ αἰσχρᾶς κομίσασθαι χάριν, καὶ μὴ ἡδονὴν δόντας ἄλλοις κακίαν αὐτοὺς ἀντιλαβεῖν·\\
Thucydides (0003) & History (001) & 4.126.2.1 & ἀγαθοῖς γὰρ εἶναι ὑμῖν προσήκει τὰ πολέμια οὐ διὰ ξυμμάχων παρουσίαν ἑκάστοτε, ἀλλὰ δι’ οἰκείαν ἀρετήν, καὶ μηδὲν πλῆθος πεφοβῆσθαι ἑτέρων, οἵ γε μηδὲ ἀπὸ πολιτειῶν τοιούτων ἥκετε, ἐν αἷς οὐ πολλοὶ ὀλίγων ἄρχουσιν, ἀλλὰ πλεόνων μᾶλλον ἐλάσσους, οὐκ ἄλλῳ τινὶ κτησάμενοι τὴν δυναστείαν ἢ τῷ μαχόμενοι κρατεῖν.\\
Thucydides (0003) & History (001) & 1.1.3.1 & τὰ γὰρ πρὸ αὐτῶν καὶ τὰ ἔτι παλαίτερα σαφῶς μὲν εὑρεῖν διὰ χρόνου πλῆθος ἀδύνατα ἦν, ἐκ δὲ τεκμηρίων ὧν ἐπὶ μακρότατον σκοποῦντί μοι πιστεῦσαι ξυμβαίνει οὐ μεγάλα νομίζω γενέσθαι οὔτε κατὰ τοὺς πολέμους οὔτε ἐς τὰ ἄλλα.\\
\addlinespace
Thucydides (0003) & History (001) & 1.29.5.3 & τῇ δὲ αὐτῇ ἡμέρᾳ αὐτοῖς ξυνέβη καὶ τοὺς τὴν Ἐπίδαμνον πολιορκοῦντας παραστήσασθαι ὁμολογίᾳ ὥστε τοὺς μὲν ἐπήλυδας ἀποδόσθαι, Κορινθίους δὲ δήσαντας ἔχειν ἕως ἂν ἄλλο τι δόξῃ.\\
Thucydides (0003) & History (001) & 5.26.5.3 & καὶ ξυνέβη μοι φεύγειν τὴν ἐμαυτοῦ ἔτη εἴκοσι μετὰ τὴν ἐς Ἀμφίπολιν στρατηγίαν, καὶ γενομένῳ παρ’ ἀμφοτέροις τοῖς πράγμασι, καὶ οὐχ ἧσσον τοῖς Πελοποννησίων διὰ τὴν φυγήν, καθ’ ἡσυχίαν τι αὐτῶν μᾶλλον αἰσθέσθαι.\\
Thucydides (0003) & History (001) & 5.72.1.1 & ξυνέβη οὖν αὐτῷ ἅτε ἐν αὐτῇ τῇ ἐφόδῳ καὶ ἐξ ὀλίγου παραγγείλαντι τόν τε Ἀριστοκλέα καὶ τὸν Ἱππονοΐδαν μὴ ’θελῆσαι παρελθεῖν, ἀλλὰ καὶ διὰ τοῦτο τὸ αἰτίαμα ὕστερον φεύγειν ἐκ Σπάρτης δόξαντας μαλακισθῆναι, καὶ τοὺς πολεμίους φθάσαι τῇ προσμείξει, καὶ κελεύσαντος αὐτοῦ, ἐπὶ τοὺς Σκιρίτας ὡς οὐ παρῆλθον οἱ λόχοι, πάλιν αὖ σφίσι προσμεῖξαι, μὴ δυνηθῆναι ἔτι μηδὲ τούτους ξυγκλῇσαι.\\
Thucydides (0003) & History (001) & 6.55.4.1 & Ἱππάρχῳ δὲ ξυνέβη τοῦ πάθους τῇ δυστυχίᾳ ὀνομασθέντα καὶ τὴν δόξαν τῆς τυραννίδος ἐς τὰ ἔπειτα προσλαβεῖν.\\
Thucydides (0003) & History (001) & 7.57.9.9 & ξυνέβη δὲ τοῖς Κρησὶ τὴν Γέλαν Ῥοδίοις ξυγκτίσαντας μὴ ξὺν τοῖς ἀποίκοις, ἀλλ’ ἐπὶ τοὺς ἀποίκους ἑκόντας μετὰ μισθοῦ ἐλθεῖν.\\
\addlinespace
Thucydides (0003) & History (001) & 7.13.1.1 & ἡμῖν δ’ ἐκ πολλῆς ἂν περιουσίας νεῶν μόλις τοῦτο ὑπῆρχε καὶ μὴ ἀναγκαζομένοις ὥσπερ νῦν πάσαις φυλάσσειν·\\
Thucydides (0003) & History (001) & 7.63.3.1 & τοῖς δὲ ναύταις παραινῶ καὶ ἐν τῷ αὐτῷ τῷδε καὶ δέομαι μὴ ἐκπεπλῆχθαί τι ταῖς ξυμφοραῖς ἄγαν, τήν τε παρασκευὴν ἀπὸ τῶν καταστρωμάτων βελτίω νῦν ἔχοντας καὶ τὰς ναῦς πλείους, ἐκείνην τε τὴν ἡδονὴν ἐνθυμεῖσθαι ὡς ἀξία ἐστὶ διασώσασθαι, οἳ τέως Ἀθηναῖοι νομιζόμενοι καὶ μὴ ὄντες ἡμῶν τῆς τε φωνῆς τῇ ἐπιστήμῃ καὶ τῶν τρόπων τῇ μιμήσει ἐθαυμάζεσθε κατὰ τὴν Ἑλλάδα, καὶ τῆς ἀρχῆς τῆς ἡμετέρας οὐκ ἔλασσον κατὰ τὸ ὠφελεῖσθαι ἔς τε τὸ φοβερὸν τοῖς ὑπηκόοις καὶ τὸ μὴ ἀδικεῖσθαι πολὺ πλέον μετείχετε.\\
Thucydides (0003) & History (001) & 2.7.2.1 & καὶ Λακεδαιμονίοις μὲν πρὸς ταῖς αὐτοῦ ὑπαρχούσαις ἐξ Ἰταλίας καὶ Σικελίας τοῖς τἀκείνων ἑλομένοις ναῦς ἐπετάχθη ποιεῖσθαι κατὰ μέγεθος τῶν πόλεων, ὡς ἐς τὸν πάντα ἀριθμὸν πεντακοσίων νεῶν ἐσομένων, καὶ ἀργύριον ῥητὸν ἑτοιμάζειν, τά τε ἄλλα ἡσυχάζοντας καὶ Ἀθηναίους δεχομένους μιᾷ νηὶ ἕως ἂν ταῦτα παρασκευασθῇ.\\
Thucydides (0003) & History (001) & 5.71.2.1 & δείσας δὲ Ἆγις μὴ σφῶν κυκλωθῇ τὸ εὐώνυμον, καὶ νομίσας ἄγαν περιέχειν τοὺς Μαντινέας, τοῖς μὲν Σκιρίταις καὶ Βρασιδείοις ἐσήμηνεν ἐπεξαγαγόντας ἀπὸ σφῶν ἐξισῶσαι τοῖς Μαντινεῦσιν, ἐς δὲ τὸ διάκενον τοῦτο παρήγγελλεν ἀπὸ τοῦ δεξιοῦ κέρως δύο λόχους τῶν πολεμάρχων Ἱππονοΐδᾳ καὶ Ἀριστοκλεῖ ἔχουσι παρελθεῖν καὶ ἐσβαλόντας πληρῶσαι, νομίζων τῷ θ’ ἑαυτῶν δεξιῷ ἔτι περιουσίαν ἔσεσθαι καὶ τὸ κατὰ τοὺς Μαντινέας βεβαιότερον τετάξεσθαι.\\
Thucydides (0003) & History (001) & 5.71.2.1 & δείσας δὲ Ἆγις μὴ σφῶν κυκλωθῇ τὸ εὐώνυμον, καὶ νομίσας ἄγαν περιέχειν τοὺς Μαντινέας, τοῖς μὲν Σκιρίταις καὶ Βρασιδείοις ἐσήμηνεν ἐπεξαγαγόντας ἀπὸ σφῶν ἐξισῶσαι τοῖς Μαντινεῦσιν, ἐς δὲ τὸ διάκενον τοῦτο παρήγγελλεν ἀπὸ τοῦ δεξιοῦ κέρως δύο λόχους τῶν πολεμάρχων Ἱππονοΐδᾳ καὶ Ἀριστοκλεῖ ἔχουσι παρελθεῖν καὶ ἐσβαλόντας πληρῶσαι, νομίζων τῷ θ’ ἑαυτῶν δεξιῷ ἔτι περιουσίαν ἔσεσθαι καὶ τὸ κατὰ τοὺς Μαντινέας βεβαιότερον τετάξεσθαι.\\
\addlinespace
Xenophon (0032) & Anabasis (006) & 2.1.2.3 & ἔδοξεν οὖν αὐτοῖς συσκευασαμένοις ἃ εἶχον καὶ ἐξοπλισαμένοις προϊέναι εἰς τὸ πρόσθεν, ἕως Κύρῳ συμμείξειαν.\\
Xenophon (0032) & Anabasis (006) & 3.2.1.1 & ἐπεὶ δὲ ᾕρηντο, ἡμέρα τε σχεδὸν ὑπέφαινε καὶ εἰς τὸ μέσον ἧκον οἱ ἄρχοντες, καὶ ἔδοξεν αὐτοῖς προφυλακὰς καταστήσαντας συγκαλεῖν τοὺς στρατιώτας.\\
Xenophon (0032) & Anabasis (006) & 4.1.12.1 & ἅμα δὲ τῇ ἡμέρᾳ συνελθοῦσι τοῖς στρατηγοῖς καὶ λοχαγοῖς τῶν Ἑλλήνων ἔδοξε τῶν τε ὑποζυγίων τὰ ἀναγκαῖα καὶ δυνατώτατα ἔχοντας πορεύεσθαι, καταλιπόντας τἆλλα, καὶ ὅσα ἦν νεωστὶ αἰχμάλωτα ἀνδράποδα ἐν τῇ στρατιᾷ πάντα ἀφεῖναι.\\
Xenophon (0032) & Anabasis (006) & 4.8.9.4 & ἔπειτα δὲ ἔδοξε τοῖς στρατηγοῖς βουλεύσασθαι συλλεγεῖσιν ὅπως ὡς κάλλιστα ἀγωνιοῦνται.\\
Xenophon (0032) & Anabasis (006) & 4.8.12.1 & ἀλλά μοι δοκεῖ ὀρθίους τοὺς λόχους ποιησαμένους τοσοῦτον χωρίον κατασχεῖν διαλιπόντας τοῖς λόχοις ὅσον ἔξω τοὺς ἐσχάτους λόχους γενέσθαι τῶν πολεμίων κεράτων·\\
\addlinespace
Xenophon (0032) & Anabasis (006) & 5.6.1.2 & καὶ ἐδόκει αὐτοῖς περὶ τῆς λοιπῆς πορείας παρακαλέσαντας τοὺς Σινωπέας βουλεύεσθαι.\\
Xenophon (0032) & Anabasis (006) & 6.6.30.1 & ἐκ τούτου ἔδοξεν αὐτοῖς πέμψαντας στρατηγοὺς καὶ λοχαγοὺς καὶ Δρακόντιον τὸν Σπαρτιάτην καὶ τῶν ἄλλων οἳ ἐδόκουν ἐπιτήδειοι εἶναι δεῖσθαι Κλεάνδρου κατὰ πάντα τρόπον ἀφεῖναι τὼ ἄνδρε.\\
Xenophon (0032) & Anabasis (006) & 6.6.38.1 & ἐπεὶ δὲ οὐδενὶ ἐνέτυχον πορευόμενοι τὴν ὀρθὴν ὁδόν, ὥστε ἔχοντές τι εἰς τὴν φιλίαν ἐλθεῖν, ἔδοξεν αὐτοῖς τοὔμπαλιν ὑποστρέψαντας ἐλθεῖν μίαν ἡμέραν καὶ νύκτα.\\
Xenophon (0032) & Anabasis (006) & 7.1.31.1 & καὶ νῦν μοι δοκεῖ πέμψαντας Ἀναξιβίῳ εἰπεῖν ὅτι ἡμεῖς οὐδὲν βίαιον ποιήσοντες παρεληλύθαμεν εἰς τὴν πόλιν, ἀλλ’ ἢν μὲν δυνώμεθα παρ’ ὑμῶν ἀγαθόν τι εὑρίσκεσθαι, εἰ δὲ μή, ἀλλὰ δηλώσοντες ὅτι οὐκ ἐξαπατώμενοι ἀλλὰ πειθόμενοι ἐξερχόμεθα.\\
Xenophon (0032) & Cyropaedia (007) & 3.3.14.2 & ἡμῖν γὰρ δοκεῖ πᾶσιν, ἐπείπερ παρεσκευάσμεθα, μὴ ἐπειδὰν ἐμβάλωσιν οἱ πολέμιοι εἰς τὴν σὴν χώραν, τότε μάχεσθαι, μηδ’ ἐν τῇ φιλίᾳ καθημένους ἡμᾶς ὑπομένειν, ἀλλ’ ἰέναι ὡς τάχιστα εἰς τὴν πολεμίαν.\\
\addlinespace
Xenophon (0032) & Hellenica (001) & 3.2.14.1 & ἐπεὶ δ’ ἐκεῖ ἦσαν, ἔδοξεν αὐτοῖς ἱκανὰς φυλακὰς εἰς τὰ ἐρύματα καταστήσαντας διαβαίνειν πάλιν ἐπὶ τὴν Ἰωνίαν.\\
Xenophon (0032) & Hellenica (001) & 3.5.23.9 & διὰ οὖν πάντα ταῦτα ἔδοξεν αὐτοῖς τοὺς νεκροὺς ὑποσπόνδους ἀναιρεῖσθαι.\\
Xenophon (0032) & Memorabilia (002) & 4.3.13.1 & ὅτι δέ γε ἀληθῆ λέγω, καὶ σὺ γνώσῃ, ἂν μὴ ἀναμένῃς ἕως ἂν τὰς μορφὰς τῶν θεῶν ἴδῃς, ἀλλ’ ἐξαρκῇ σοι τὰ ἔργα αὐτῶν ὁρῶντι σέβεσθαι καὶ τιμᾶν τοὺς θεούς.\\
Xenophon (0032) & Anabasis (006) & 2.3.26.1 & καὶ νῦν ἔξεστιν ὑμῖν πιστὰ λαβεῖν παρ’ ἡμῶν ἦ μὴν φιλίαν παρέξειν ὑμῖν τὴν χώραν καὶ ἀδόλως ἀπάξειν εἰς τὴν Ἑλλάδα ἀγορὰν παρέχοντας·\\
Xenophon (0032) & Anabasis (006) & 2.5.18.2 & οὐ τοσαῦτα μὲν πεδία ἃ ὑμεῖς φίλια ὄντα σὺν πολλῷ πόνῳ διαπορεύεσθε, τοσαῦτα δὲ ὄρη ὁρᾶτε ὑμῖν ὄντα πορευτέα, ἃ ἡμῖν ἔξεστι προκαταλαβοῦσιν ἄπορα ὑμῖν παρέχειν, τοσοῦτοι δ’ εἰσὶ ποταμοὶ ἐφ’ ὧν ἔξεστιν ἡμῖν ταμιεύεσθαι ὁπόσοις ἂν ὑμῶν βουλώμεθα μάχεσθαι;\\
\addlinespace
Xenophon (0032) & Anabasis (006) & 4.3.10.2 & ᾔδεσαν γὰρ πάντες ὅτι ἐξείη αὐτῷ καὶ ἀριστῶντι καὶ δειπνοῦντι προσελθεῖν καὶ εἰ καθεύδοι ἐπεγείραντα εἰπεῖν, εἴ τίς τι ἔχοι τῶν πρὸς τὸν πόλεμον.\\
Xenophon (0032) & Anabasis (006) & 5.8.17.1 & καὶ γὰρ οὖν νῦν ἔξεστιν αὐτοῖς σωθεῖσιν, εἴ τι ὑπ’ ἐμοῦ ἔπαθον παρὰ τὸ δίκαιον, δίκην λαβεῖν.\\
Xenophon (0032) & Anabasis (006) & 7.1.21.3 & νῦν σοι ἔξεστιν, ὦ Ξενοφῶν, ἀνδρὶ γενέσθαι.\\
Xenophon (0032) & Constitution of the Lacedaemonians (010) & 13.9.1 & ἔξεστι δὲ τῷ νέῳ καὶ κεχριμένῳ εἰς μάχην συνιέναι καὶ φαιδρὸν εἶναι καὶ εὐδόκιμον.\\
Xenophon (0032) & Cyropaedia (007) & 2.1.15.7 & ἔξεστι δ’ ὑμῖν, εἰ βούλεσθε, λαβόντας ὅπλα οἷάπερ ἡμεῖς ἔχομεν εἰς τὸν αὐτὸν ἡμῖν κίνδυνον ἐμβαίνειν, καὶ ἄν τι ἐκ τούτων καλὸν κἀγαθὸν γίγνηται, τῶν ὁμοίων ἡμῖν ἀξιοῦσθαι.\\
\addlinespace
Xenophon (0032) & Cyropaedia (007) & 4.3.14.5 & ὅπου γὰρ ἂν βουλώμεθα, ἐξέσται ἡμῖν πεζοῖς εὐθὺς μάχεσθαι·\\
Xenophon (0032) & Cyropaedia (007) & 7.2.14.1 & ἐξέσται δέ σοι ἰδόντι ταῦτα ἐλθόντα ἔτι καὶ περὶ τῆς ἁρπαγῆς βουλεύσασθαι.\\
Xenophon (0032) & Cyropaedia (007) & 8.3.50.1 & οὕτω δὴ ὅ τε Φεραύλας ὑπερήδετο ὅτι ἐξέσοιτο αὐτῷ ἀπαλλαγέντι τῆς τῶν ἄλλων κτημάτων ἐπιμελείας ἀμφὶ τοὺς φίλους ἔχειν, ὅ τε Σάκας ὅτι ἔμελλε πολλὰ ἔχων πολλοῖς χρήσεσθαι.\\
Xenophon (0032) & Hellenica (001) & 2.3.41.8 & ἐξῆν γὰρ αὐτοῖς, εἰ τούτου γε δέοιντο, καὶ μηδένα λιπεῖν ὀλίγον ἔτι χρόνον τῷ λιμῷ πιέσαντας.\\
Xenophon (0032) & Hellenica (001) & 4.1.35.3 & νῦν δὲ ἔξεστί σοι μεθ’ ἡμῶν γενομένῳ μηδένα προσκυνοῦντα μηδὲ δεσπότην ἔχοντα ζῆν καρπούμενον τὰ σαυτοῦ.\\
\addlinespace
Xenophon (0032) & Hellenica (001) & 4.8.4.1 & ὦ ἄνδρες, νῦν ἔξεστιν ὑμῖν καὶ πρόσθεν φίλοις οὖσι τῇ πόλει ἡμῶν εὐεργέτας φανῆναι τῶν Λακεδαιμονίων.\\
Xenophon (0032) & Hellenica (001) & 5.4.26.2 & ἔξεστί σοι, ὦ υἱέ, σῶσαι τὸν πατέρα, δεηθέντι Ἀρχιδάμου εὐμενῆ Ἀγησίλαον ἐμοὶ εἰς τὴν κρίσιν παρασχεῖν.\\
Xenophon (0032) & Hellenica (001) & 7.1.17.1 & ὡς δὲ οἱ σωθέντες ἐκ τοῦ πράγματος ἀπέφυγον ἐπὶ τὸν ἐγγύτατα λόφον, ἐξὸν τῷ Λακεδαιμονίων πολεμάρχῳ λαβόντι ὁπόσους μὲν ἐβούλετο τῶν συμμάχων ὁπλίτας, ὁπόσους δὲ πελταστάς, κατέχειν τὸ χωρίον, καὶ γὰρ τὰ ἐπιτήδεια ἐξῆν ἀσφαλῶς ἐκ Κεγχρειῶν κομίζεσθαι, οὐκ ἐποίησε ταῦτα, ἀλλὰ μάλα ἀπορούντων τῶν Θηβαίων πῶς χρὴ ἐκ τοῦ πρὸς Σικυῶνα βλέποντος καταβῆναι ἢ πάλιν ἀπελθεῖν, σπονδὰς ποιησάμενος, ὡς τοῖς πλείστοις ἐδόκει, πρὸς Θηβαίων μᾶλλον ἢ πρὸς ἑαυτῶν, οὕτως ἀπῆλθε καὶ τοὺς μεθ’ αὑτοῦ ἀπήγαγεν.\\
Xenophon (0032) & Anabasis (006) & 3.2.11.1 & ἔπειτα δὲ ἀναμνήσω γὰρ ὑμᾶς καὶ τοὺς τῶν προγόνων τῶν ἡμετέρων κινδύνους, ἵνα εἰδῆτε ὡς ἀγαθοῖς τε ὑμῖν προσήκει εἶναι σῴζονταί τε σὺν τοῖς θεοῖς καὶ ἐκ πάνυ δεινῶν οἱ ἀγαθοί.\\
Xenophon (0032) & Apology (005) & 24.5 & ἐμοὶ δὲ τί προσήκει νῦν μεῖον φρονεῖν ἢ πρὶν κατακριθῆναι, μηδὲν ἐλεγχθέντι ὡς πεποίηκά τι ὧν ἐγράψαντό με;\\
\addlinespace
Xenophon (0032) & On the Cavalry Commander (012) & 7.1.1 & παντὶ μὲν οὖν προσήκει ἄρχοντι φρονίμῳ εἶναι·\\
Xenophon (0032) & On the Art of Horsemanship (013) & 9.1.4 & καιρὸς δ’ ἴσως γράψαι καί, εἴ ποτε συμβαίη θυμοειδεστέρῳ ἵππῳ τοῦ καιρίου χρῆσθαι ἢ βλακωδεστέρῳ, ὡς ἂν ὀρθότατα ἑκατέρῳ χρῷτο.\\
Xenophon (0032) & Economics (003) & 11.23.3 & ἢ γὰρ κατηγοροῦντός τινος τῶν οἰκετῶν ἢ ἀπολογουμένου ἀκούσας ἐλέγχειν πειρῶμαι, ἢ μέμφομαί τινα πρὸς τοὺς φίλους ἢ ἐπαινῶ, ἢ διαλλάττω τινὰς τῶν ἐπιτηδείων πειρώμενος διδάσκειν ὡς συμφέρει αὐτοῖς φίλους εἶναι μᾶλλον ἢ πολεμίους, ἢ ἐπιτιμῶμέν τινι στρατηγῷ συμπαρόντες, ἢ ἀπολογούμεθα ὑπέρ του, εἴ τις ἀδίκως αἰτίαν ἔχει, ἢ κατηγοροῦμεν πρὸς ἀλλήλους, εἴ τις ἀδίκως τιμᾶται.\\
Xenophon (0032) & Cyropaedia (007) & 8.8.20.1 & οἷς ἐν μὲν τῷ παρελθόντι χρόνῳ ἐπιχώριον εἶναι ὑπῆρχε τοὺς μὲν τὴν γῆν ἔχοντας ἀπὸ ταύτης ἱππότας παρέχεσθαι, οἳ δὴ καὶ ἐστρατεύοντο εἰ δέοι στρατεύεσθαι, τοὺς δὲ φρουροῦντας πρὸ τῆς χώρας μισθοφόρους εἶναι·\\
Xenophon (0032) & Hiero (008) & 11.9.1 & περίβλεπτος δὲ ὢν οὐχ ὑπὸ ἰδιωτῶν μόνον ἀλλὰ καὶ ὑπὸ πολλῶν πόλεων ἀγαπῷο ἄν, καὶ θαυμαστὸς οὐκ ἰδίᾳ μόνον ἀλλὰ καὶ δημοσίᾳ παρὰ πᾶσιν ἂν εἴης, καὶ ἐξείη μὲν ἄν σοι ἕνεκεν ἀσφαλείας, εἴ ποι βούλοιο, θεωρήσοντι πορεύεσθαι, ἐξείη δ’ ἂν αὐτοῦ μένοντι τοῦτο πράττειν.\\
\addlinespace
Xenophon (0032) & Memorabilia (002) & 2.1.33.12 & τοιαῦτά σοι, ὦ παῖ τοκέων ἀγαθῶν Ἡράκλεις, ἔξεστι διαπονησαμένῳ τὴν μακαριστοτάτην εὐδαιμονίαν κεκτῆσθαι.\\
Xenophon (0032) & Memorabilia (002) & 2.6.26.1 & ἀλλὰ καὶ ἐν τοῖς γυμνικοῖς ἀγῶσι δῆλόν ἐστιν, ὅτι, εἰ ἐξῆν τοῖς κρατίστοις συνθεμένους ἐπὶ τοὺς χείρους ἰέναι, πάντας ἂν τοὺς ἀγῶνας οὗτοι ἐνίκων καὶ πάντα τὰ ἆθλα οὗτοι ἐλάμβανον.\\
Xenophon (0032) & Memorabilia (002) & 4.5.10.7 & τῷ γὰρ ἂν ἧττον φήσαιμεν τῶν τοιούτων προσήκειν, ἢ ᾧ ἥκιστα ἔξεστι ταῦτα πράττειν, κατεχομένῳ ἐπὶ τῷ σπουδάζειν περὶ τὰς ἐγγυτάτω ἡδονάς;\\
Xenophon (0032) & Memorabilia (002) & 4.5.11.7 & ἀλλὰ τοῖς ἐγκρατέσι μόνοις ἔξεστι σκοπεῖν τὰ κράτιστα τῶν πραγμάτων, καὶ λόγῳ καὶ ἔργῳ διαλέγοντας κατὰ γένη τὰ μὲν ἀγαθὰ προαιρεῖσθαι, τῶν δὲ κακῶν ἀπέχεσθαι.\\
Xenophon (0032) & On the Cavalry Commander (012) & 4.12.1 & ἔτι δὲ τῷ κρυπτὰς ἔχοντι φυλακὰς ἐξέσται μὲν φανεροῖς ὀλίγοις ἔμπροσθεν τῶν κρυπτῶν φυλάττοντα πειρᾶσθαι τοὺς πολεμίους εἰς ἐνέδρας ὑπάγειν·\\
\addlinespace
Xenophon (0032) & Symposium (004) & 4.31.1 & νῦν δ’ ἐπειδὴ τῶν ὑπερορίων στέρομαι καὶ τὰ ἔγγεια οὐ καρποῦμαι καὶ τὰ ἐκ τῆς οἰκίας πέπραται, ἡδέως μὲν καθεύδω ἐκτεταμένος, πιστὸς δὲ τῇ πόλει γεγένημαι, οὐκέτι δὲ ἀπειλοῦμαι, ἀλλ’ ἤδη ἀπειλῶ ἄλλοις, ὡς ἐλευθέρῳ τε ἔξεστί μοι καὶ ἀποδημεῖν καὶ ἐπιδημεῖν·\\
Xenophon (0032) & Hellenica (001) & 5.2.30.4 & ἐγὼ δὲ τοῦ νόμου κελεύοντος ἐξεῖναι πολεμάρχῳ λαβεῖν, εἴ τις δοκεῖ ἄξια θανάτου ποιεῖν, λαμβάνω τουτονὶ Ἰσμηνίαν, ὡς πολεμοποιοῦντα.\\
Xenophon (0032) & Hellenica (001) & 5.4.60.3 & ἐξεῖναι γὰρ σφίσι ναῦς πληρώσαντας πολὺ πλείους τῶν Ἀθηναίων ἑλεῖν λιμῷ τὴν πόλιν αὐτῶν·\\
Xenophon (0032) & Memorabilia (002) & 3.9.9.4 & ἐξεῖναι γὰρ αὐτοῖς ἰέναι πράξοντας τὰ βελτίω τούτων.\\
Xenophon (0032) & Memorabilia (002) & 2.6.36.7 & ἃ δὴ καὶ ἐγὼ πεισθεὶς ὀρθῶς ἔχειν ἡγοῦμαι οὐκ ἐξεῖναί μοι περὶ σοῦ λέγειν ἐπαινοῦντι οὐδὲν ὅ τι ἂν μὴ ἀληθεύω.\\
\addlinespace
Xenophon (0032) & Hiero (008) & 6.11.3 & πιστὸν δὲ ἕνα πολὺ χαλεπώτερον εὑρεῖν ἢ πάνυ πολλοὺς ἐργάτας ὁποίου βούλει ἔργου, ἄλλως τε καὶ ὁπόταν χρημάτων μὲν ἕνεκα παρῶσιν οἱ φυλάττοντες, ἐξῇ δ’ αὐτοῖς ἐν ὀλίγῳ χρόνῳ πολὺ πλείω λαβεῖν ἀποκτείνασι τὸν τύραννον ἢ ὅσα πολὺν χρόνον φυλάττοντες παρὰ τοῦ τυράννου λαμβάνουσιν.\\
Xenophon (0032) & Agesilaus (009) & 2.12.3 & ἐξὸν γὰρ αὐτῷ παρέντι τοὺς διαπίπτοντας ἑπομένῳ χειροῦσθαι τοὺς ὄπισθεν οὐκ ἐποίησε τοῦτο, ἀλλ’ ἀντιμέτωπος συνέρραξε τοῖς Θηβαίοις.\\
Xenophon (0032) & Anabasis (006) & 5.6.3.1 & ἀναστὰς δὲ Ἑκατώνυμος πρῶτον μὲν ἀπελογήσατο περὶ οὗ εἶπεν ὡς τὸν Παφλαγόνα φίλον ποιήσοιντο, ὅτι οὐχ ὡς τοῖς Ἕλλησι πολεμησόντων σφῶν εἴποι, ἀλλ’ ὅτι ἐξὸν τοῖς βαρβάροις φίλους εἶναι τοὺς Ἕλληνας αἱρήσονται.\\
Xenophon (0032) & Hellenica (001) & 4.3.19.3 & ἐξὸν γὰρ αὐτῷ παρέντι τοὺς διαπίπτοντας ἀκολουθοῦντι χειροῦσθαι τοὺς ὄπισθεν, οὐκ ἐποίησε τοῦτο, ἀλλ’ ἀντιμέτωπος συνέρραξε τοῖς Θηβαίοις·\\
Xenophon (0032) & Hellenica (001) & 7.1.17.1 & ὡς δὲ οἱ σωθέντες ἐκ τοῦ πράγματος ἀπέφυγον ἐπὶ τὸν ἐγγύτατα λόφον, ἐξὸν τῷ Λακεδαιμονίων πολεμάρχῳ λαβόντι ὁπόσους μὲν ἐβούλετο τῶν συμμάχων ὁπλίτας, ὁπόσους δὲ πελταστάς, κατέχειν τὸ χωρίον, καὶ γὰρ τὰ ἐπιτήδεια ἐξῆν ἀσφαλῶς ἐκ Κεγχρειῶν κομίζεσθαι, οὐκ ἐποίησε ταῦτα, ἀλλὰ μάλα ἀπορούντων τῶν Θηβαίων πῶς χρὴ ἐκ τοῦ πρὸς Σικυῶνα βλέποντος καταβῆναι ἢ πάλιν ἀπελθεῖν, σπονδὰς ποιησάμενος, ὡς τοῖς πλείστοις ἐδόκει, πρὸς Θηβαίων μᾶλλον ἢ πρὸς ἑαυτῶν, οὕτως ἀπῆλθε καὶ τοὺς μεθ’ αὑτοῦ ἀπήγαγεν.\\
\addlinespace
Xenophon (0032) & Hellenica (001) & 6.1.13.5 & ὁ δ’ ἐπαινέσας με καὶ εἰπὼν ὅτι μᾶλλον ἑκτέον μου εἴη, ὅτι τοιοῦτος εἴην, ἀφῆκέ μοι ἐλθόντι πρὸς ὑμᾶς λέγειν τἀληθῆ, ὅτι διανοοῖτο στρατεύειν ἐπὶ Φαρσαλίους, εἰ μὴ πεισοίμεθα.\\
Xenophon (0032) & Anabasis (006) & 6.6.33.1 & δέονται δέ σου καὶ τοῦτο, παραγενόμενον καὶ ἄρξαντα ἑαυτῶν πεῖραν λαβεῖν καὶ Δεξίππου καὶ σφῶν τῶν ἄλλων οἷος ἕκαστός ἐστι, καὶ τὴν ἀξίαν ἑκάστοις νεῖμαι.\\
Xenophon (0032) & Cyropaedia (007) & 6.4.9.3 & ἀλλ’, ὦ Ζεῦ μέγιστε, δός μοι φανῆναι ἀξίῳ μὲν Πανθείας ἀνδρί, ἀξίῳ δὲ Κύρου φίλῳ τοῦ ἡμᾶς τιμήσαντος.\\
Xenophon (0032) & Anabasis (006) & 6.6.15.1 & ἐγὼ μὲν οὖν (καὶ γὰρ ἀκούω Δέξιππον λέγειν πρὸς Κλέανδρον ὡς οὐκ ἂν ἐποίησεν Ἀγασίας ταῦτα, εἰ μὴ ἐγὼ αὐτὸν ἐκέλευσα), ἐγὼ μὲν οὖν ἀπολύω καὶ ὑμᾶς τῆς αἰτίας καὶ Ἀγασίαν, ἂν αὐτὸς Ἀγασίας φήσῃ ἐμέ τι τούτων αἴτιον εἶναι, καὶ καταδικάζω ἐμαυτοῦ, εἰ ἐγὼ πετροβολίας ἢ ἄλλου τινὸς βιαίου ἐξάρχω, τῆς ἐσχάτης δίκης ἄξιος εἶναι, καὶ ὑφέξω τὴν δίκην.\\
Xenophon (0032) & Anabasis (006) & 1.2.1.3 & ἐνταῦθα καὶ παραγγέλλει τῷ τε Κλεάρχῳ λαβόντι ἥκειν ὅσον ἦν αὐτῷ στράτευμα καὶ τῷ Ἀριστίππῳ συναλλαγέντι πρὸς τοὺς οἴκοι ἀποπέμψαι πρὸς ἑαυτὸν ὃ εἶχε στράτευμα·\\
\addlinespace
Xenophon (0032) & Anabasis (006) & 1.2.1.7 & καὶ Ξενίᾳ τῷ Ἀρκάδι, ὃς αὐτῷ προειστήκει τοῦ ἐν ταῖς πόλεσι ξενικοῦ, ἥκειν παραγγέλλει λαβόντα τοὺς ἄλλους πλὴν ὁπόσοι ἱκανοὶ ἦσαν τὰς ἀκροπόλεις φυλάττειν.\\
Xenophon (0032) & Anabasis (006) & 4.3.26.1 & καὶ τὰ σκευοφόρα τῶν Ἑλλήνων καὶ ὁ ὄχλος ἀκμὴν διέβαινε, Ξενοφῶν δὲ στρέψας πρὸς τοὺς Καρδούχους ἀντία τὰ ὅπλα ἔθετο, καὶ παρήγγειλε τοῖς λοχαγοῖς κατ’ ἐνωμοτίας ποιήσασθαι ἕκαστον τὸν ἑαυτοῦ λόχον, παρ’ ἀσπίδα παραγαγόντας τὴν ἐνωμοτίαν ἐπὶ φάλαγγος·\\
Xenophon (0032) & Anabasis (006) & 4.3.29.1 & τοῖς δὲ παρ’ ἑαυτῷ παρήγγειλεν, ἐπειδὰν σφενδόνη ἐξικνῆται καὶ ἀσπὶς ψοφῇ, παιανίσαντας θεῖν εἰς τοὺς πολεμίους, ἐπειδὰν δ’ ἀναστρέψωσιν οἱ πολέμιοι καὶ ἐκ τοῦ ποταμοῦ ὁ σαλπικτὴς σημήνῃ τὸ πολεμικόν, ἀναστρέψαντας ἐπὶ δόρυ ἡγεῖσθαι μὲν τοὺς οὐραγούς, θεῖν δὲ πάντας καὶ διαβαίνειν ὅτι τάχιστα ᾗ ἕκαστος τὴν τάξιν εἶχεν, ὡς μὴ ἐμποδίζειν ἀλλήλους·\\
Xenophon (0032) & Anabasis (006) & 5.2.12.1 & ὁ δὲ τοῖς πελτασταῖς πᾶσι παρήγγειλε διηγκυλωμένους ἰέναι, ὡς, ὁπόταν σημήνῃ, ἀκοντίζειν, καὶ τοὺς τοξότας ἐπιβεβλῆσθαι ἐπὶ ταῖς νευραῖς, ὡς, ὁπόταν σημήνῃ, τοξεύειν δεῆσον, καὶ τοὺς γυμνῆτας λίθων ἔχειν μεστὰς τὰς διφθέρας·\\
Xenophon (0032) & Cyropaedia (007) & 2.4.2.1 & ἀκούσας δὲ ταῦτα ὁ Κῦρος παρήγγειλε τῷ πρώτῳ τεταγμένῳ ταξιάρχῳ εἰς μέτωπον στῆναι, ἐφ’ ἑνὸς ἄγοντα τὴν τάξιν, ἐν δεξιᾷ ἔχοντα ἑαυτόν, καὶ τῷ δευτέρῳ ἐκέλευσε ταὐτὸ τοῦτο παραγγεῖλαι, καὶ διὰ πάντων οὕτω παραδιδόναι ἐκέλευσεν.\\
\addlinespace
Xenophon (0032) & Cyropaedia (007) & 3.3.34.1 & τῇ δ’ ὑστεραίᾳ πρῲ Κῦρος μὲν ἐστεφανωμένος ἔθυε, παρήγγειλε δὲ καὶ τοῖς ἄλλοις ὁμοτίμοις ἐστεφανωμένοις πρὸς τὰ ἱερὰ παρεῖναι.\\
Xenophon (0032) & Cyropaedia (007) & 6.3.21.11 & παραγγείλατε δὲ τοῖς ταξιάρχοις καὶ λοχαγοῖς ἐπὶ φάλαγγος καθίστασθαι εἰς δύο ἔχοντας ἕκαστον τὸν λόχον.\\
Xenophon (0032) & Hellenica (001) & 1.6.37.1 & αὐτὸς δ’, ἐπειδὴ ἐκεῖνοι κατέπλεον, ἔθυε τὰ εὐαγγέλια, καὶ τοῖς στρατιώταις παρήγγειλε δειπνοποιεῖσθαι, καὶ τοῖς ἐμπόροις τὰ χρήματα σιωπῇ ἐνθεμένους εἰς τὰ πλοῖα ἀποπλεῖν εἰς Χίον (ἦν δὲ τὸ πνεῦμα οὔριον) καὶ τὰς τριήρεις τὴν ταχίστην.\\
Xenophon (0032) & Hellenica (001) & 2.3.23.4 & καὶ παραγγείλαντες νεανίσκοις οἳ ἐδόκουν αὐτοῖς θρασύτατοι εἶναι ξιφίδια ὑπὸ μάλης ἔχοντας παραγενέσθαι, συνέλεξαν τὴν βουλήν.\\
Xenophon (0032) & Cyropaedia (007) & 3.3.55.1 & ἐπεὶ ἔγωγ’, ἔφη, οὐδ’ ἂν τούτοις ἐπίστευον ἐμμόνοις ἔσεσθαι οὓς νῦν ἔχοντες παρ’ ἡμῖν αὐτοῖς ἠσκοῦμεν, εἰ μὴ καὶ ὑμᾶς ἑώρων παρόντας, οἳ καὶ παράδειγμα αὐτοῖς ἔσεσθε οἵους χρὴ εἶναι καὶ ὑποβαλεῖν δυνήσεσθε, ἤν τι ἐπιλανθάνωνται.\\
\addlinespace
Xenophon (0032) & Anabasis (006) & 3.1.5.1 & καὶ ὁ Σωκράτης ὑποπτεύσας μή τι πρὸς τῆς πόλεως ὑπαίτιον εἴη Κύρῳ φίλον γενέσθαι, ὅτι ἐδόκει ὁ Κῦρος προθύμως τοῖς Λακεδαιμονίοις ἐπὶ τὰς Ἀθήνας συμπολεμῆσαι, συμβουλεύει τῷ Ξενοφῶντι ἐλθόντα εἰς Δελφοὺς ἀνακοινῶσαι τῷ θεῷ περὶ τῆς πορείας.\\
Xenophon (0032) & Anabasis (006) & 7.1.30.3 & καὶ ὑμῖν δὲ συμβουλεύω Ἕλληνας ὄντας τοῖς τῶν Ἑλλήνων προεστηκόσι πειθομένους πειρᾶσθαι τῶν δικαίων τυγχάνειν.\\
Xenophon (0032) & Cyropaedia (007) & 4.5.32.1 & συμβουλεύω δέ σοι καίπερ νεώτερος ὢν μὴ ἀφαιρεῖσθαι ἃ ἂν δῷς, ἵνα μή σοι ἀντὶ χαρίτων ἔχθραι ὀφείλωνται, μηδ’ ὅταν τινὰ βούλῃ πρὸς σὲ ταχὺ ἐλθεῖν, ἀπειλοῦντα μεταπέμπεσθαι, μηδὲ φάσκοντα ἔρημον εἶναι ἅμα πολλοῖς ἀπειλεῖν, ἵνα μὴ διδάσκῃς αὐτοὺς σοῦ μὴ φροντίζειν.\\
Xenophon (0032) & Memorabilia (002) & 1.3.6.4 & τοῖς δὲ μὴ δυναμένοις τοῦτο ποιεῖν συνεβούλευε φυλάττεσθαι τὰ πείθοντα μὴ πεινῶντας ἐσθίειν μηδὲ διψῶντας πίνειν·\\
Xenophon (0032) & On the Cavalry Commander (012) & 1.18.3 & συγκαλέσαντα δὲ χρὴ τοὺς ἱππέας συμβουλεῦσαι αὐτοῖς μελετᾶν, καὶ ὅταν εἰς χώραν ἐλαύνωσι καὶ ὅταν ἄλλοσέ ποι, ἐκβιβάζοντας τῶν ὁδῶν καὶ ταχὺ ἐλαύνοντας ἐν τόποις παντοδαποῖς.\\
\addlinespace
Isocrates (0010) & Plataicus (012) & 39.1 & εἰ δ’ οὖν καὶ τἀναντία μέλλοιεν ἅπαντα πράξειν, οὐδ’ οὕτως ἡγοῦμαι προσήκειν ὑμῖν τῆς Θηβαίων πόλεως πλείω ποιήσασθαι λόγον ἢ τῶν ὅρκων καὶ τῶν συνθηκῶν, ἐνθυμουμένους πρῶτον μὲν ὡς οὐ τοὺς κινδύνους, ἀλλὰ τὰς ἀδοξίας καὶ τὰς αἰσχύνας φοβεῖσθαι πάτριον ὑμῖν ἐστιν, ἔπειθ’ ὅτι συμβαίνει κρατεῖν ἐν τοῖς πολέμοις οὐ τοὺς βίᾳ τὰς πόλεις καταστρεφομένους, ἀλλὰ τοὺς ὁσιώτερον καὶ πραότερον τὴν Ἑλλάδα διοικοῦντας.\\
\bottomrule
\end{longtable}
}
