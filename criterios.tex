\hypertarget{crituxe9rios-adicionais-de-busca-e-seleuxe7uxe3o-de-dados}{%
\section{Critérios adicionais de busca e seleção de
dados}\label{crituxe9rios-adicionais-de-busca-e-seleuxe7uxe3o-de-dados}}

\hypertarget{crituxe9rios-adicionais-de-busca}{%
\subsection{Critérios adicionais de
busca}\label{crituxe9rios-adicionais-de-busca}}

As regras abaixo foram incluídas no processo de busca de dados para
garantir que formas analisadas de maneira errônea no processo de
anotação automática realizada pelos compiladores do banco de dados
Diorisis não fossem ignoradas.

\begin{itemize}
\tightlist
\item
  δοκέω:

  \begin{itemize}
  \tightlist
  \item
    Incluir a forma pura δόξαι
  \end{itemize}
\item
  δέομαι:

  \begin{itemize}
  \tightlist
  \item
    Incluir passiva de δέω
  \end{itemize}
\item
  ἐξαρκέω:

  \begin{itemize}
  \tightlist
  \item
    Incluir forma pura ἐξαρκεῖ
  \end{itemize}
\item
  ἔξεστι:

  \begin{itemize}
  \tightlist
  \item
    Incluir forma pura ἐξεῖναι
  \item
    Incluir forma pura ἐξεῖναί
  \item
    Incluir forma pura ἐξόν
  \end{itemize}
\item
  συμπίπτω:

  \begin{itemize}
  \tightlist
  \item
    Substituir por συμπίτνω
  \end{itemize}
\end{itemize}

\hypertarget{crituxe9rios-adicionais-de-seleuxe7uxe3o}{%
\subsection{Critérios adicionais de
seleção}\label{crituxe9rios-adicionais-de-seleuxe7uxe3o}}

\begin{enumerate}
\def\labelenumi{\arabic{enumi}.}
\tightlist
\item
  Critério de seleção para pronomes + particípios quando o verbo
  principal é \emph{impessoal}:

  \begin{itemize}
  \tightlist
  \item
    pronome + artigo + particípio = 1 constituinte \(\rightarrow\) Não
    entra

    \begin{itemize}
    \tightlist
    \item
      Demosthenes, Agains Aphobus 1, 60.1
    \end{itemize}
  \item
    particípio + pronome ou pronome + particípio = 2 constituintes
    \(\rightarrow\) Entra

    \begin{itemize}
    \tightlist
    \item
      Demosthenes, Agains Boeotus 2, 5.1
    \end{itemize}
  \end{itemize}
\item
  Critério de seleção para pronomes + particípios quando o verbo
  principal é \emph{pessoal}: incluir a opção mais provável e recomendar
  checagem posterior na coluna \texttt{CHECK}.
\item
  Ignorar sentenças com construções formulares tais como ``hōs
  emoi\textbackslash(ge) dokei= / e)/oike''.
\item
  Ignorar resultados do verbo di/dwmi salvo nos casos com formas
  imperativas, a construção é extremamente rara perto da quantidade de
  exemplos produzidos por ruído.
\item
  Ignorar resultados do verbo pisteu/w, \emph{idem}.
\item
  Excluir sentenças regidas por dokei= com substantivos no nominativo
  nas imediações.
\item
  Excluir sentenças cuja captura da busca seja um particípio neutro
  precedido por artigo.
\end{enumerate}
